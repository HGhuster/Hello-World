\documentclass{article}
\usepackage{geometry}
\usepackage[dvipsnames,svgnames]{xcolor}
\usepackage{tcolorbox}
\usepackage{mathrsfs}
\usepackage{amsfonts,amssymb}
\usepackage{euscript}
\usepackage{amsmath}
\usepackage{color}
\usepackage{soul}
\usepackage{CJKutf8}

\title{Exercise 1}

\definecolor{highlight}{rgb}{1,0,0}

\newtcolorbox{defi}[2][]{colback=Salmon!20, colframe=Salmon!90!Black,title=\textbf{#2}}

\newtcolorbox{theo}[2][]{colback=JungleGreen!10!Cerulean!15,colframe=CornflowerBlue!60!Black,title=\textbf{#2}}

\newtcolorbox{prop}[2][]{colback=Emerald!10,colframe=cyan!40!black,title=\textbf{#2}}

\newtcolorbox{com}[2][]{colback=OliveGreen!10,colframe=Green!70,title=\textbf{#2}}
%colback=OliveGreen!10,colframe=Green!70




\begin{document}
\maketitle
\begin{CJK}{UTF8}{gbsn}
练习1:

无限的$\sigma$代数意味着可测空间有无限个元素(有限个元素的power set的基数为有限),若m的不相交元素个数为有限个,则m有限,若m的不相交元素为无限个则其交并补构成一个power set,这个power set的基数和R基数一样

\vspace{0.2cm}

练习2:

略

\vspace{0.2cm}

练习3:

要证:$\forall V=(\alpha,\infty),\alpha\in \mathbb{R}$有$ f^{-1}(V)=\{x: f(x)>\alpha\}\in\mathfrak{M}$

设:对任意$\alpha$选取$\alpha<r_n<\alpha+1/n,r_n\in\mathbb{Q}$,$B_n=\{x:f(x)\ge r_n\}\in\mathfrak{M}$

$ f^{-1}(V)=\bigcup_{i=1}^{\infty}B_n\rightarrow f^{-1}(V)\in\mathfrak{M}$

根据1.12(c)的证明,f是可测的








\end{CJK}




\end{document}