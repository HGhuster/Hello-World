\documentclass{article}
\usepackage{geometry}
\usepackage[dvipsnames,svgnames]{xcolor}
\usepackage{tcolorbox}
\usepackage{mathrsfs}
\usepackage{amsfonts,amssymb}
\usepackage{euscript}
\usepackage{amsmath}
\usepackage{color}
\usepackage{soul}
\usepackage{CJKutf8}
\title{POSITIVE BOREL MEASURES}

\definecolor{highlight}{rgb}{1,0,0}

\newtcolorbox{defi}[2][]{colback=Salmon!20, colframe=Salmon!90!Black,title=\textbf{#2}}

\newtcolorbox{theo}[2][]{colback=JungleGreen!10!Cerulean!15,colframe=CornflowerBlue!60!Black,title=\textbf{#2}}

\newtcolorbox{prop}[2][]{colback=Emerald!10,colframe=cyan!40!black,title=\textbf{#2}}

\newtcolorbox{com}[2][]{colback=OliveGreen!10,colframe=Green!70,title=\textbf{#2}}
%colback=OliveGreen!10,colframe=Green!70




\begin{document}
\maketitle

\section*{Vector Spaces}
\begin{defi}{2.1 Definition}
	A complex vector space  is a set \textit{V}, whose elements are called vectors and in which two operations,
called addition and scalar multiplication, are defined, with the following
familiar algebraic properties:

\vspace{0.2 cm}

To every pair of vectors x and y there corresponds a vector $x + y$, in
such a way that $x + y = y + x$ and $x + (y + z) = (x + y) + z$; \textit{V} contains a
unique vector \textit{0} (\textit{the zero vector or origin of V}) such that \textit{x + 0 = x} for every
$x \in V$; and to each $x \in V$ there corresponds a unique vector $-x$ such that
$x + (-x) = 0$.

\vspace{0.2 cm}

To each pair $(\alpha x, x)$, where $x \in V$ and ex is a scalar, there is associated a vector exx E V, in
such a way that 1x = x, ex(px) = (exP)x, and such that the two distributive laws

\begin{displaymath}
  \alpha(x+y)=\alpha x+\alpha y, (\alpha+\beta)x=\alpha x+\beta x
\end{displaymath}

hold.

\vspace{0.2 cm}

A linear transformation of a vector space V into a vector space Vi is a mapping A of V into Vi such that


\end{defi}




\end{document}