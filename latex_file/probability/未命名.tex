\documentclass{article}
\author{Dawei Wang}
\usepackage{geometry}
\usepackage[dvipsnames,svgnames]{xcolor}
\usepackage{tcolorbox}
\usepackage{mathrsfs}
\usepackage{amsfonts,amssymb}
\usepackage{euscript}
\usepackage{amsmath}
\usepackage{color}
\usepackage{soul}

\title{Measure Theory}

\definecolor{highlight}{rgb}{1,0,0}

\newtcolorbox{defi}[2][]{colback=Salmon!20, colframe=Salmon!90!Black,title=\textbf{#2}}

\newtcolorbox{theo}[2][]{colback=JungleGreen!10!Cerulean!15,colframe=CornflowerBlue!60!Black,title=\textbf{#2}}

\newtcolorbox{prop}[2][]{colback=Emerald!10,colframe=cyan!40!black,title=\textbf{#2}}

\newtcolorbox{example}[2][]{colback=OliveGreen!10,colframe=Green!70,title=\textbf{#2}}
%colback=OliveGreen!10,colframe=Green!70




\begin{document}
\maketitle

\section*{Probability Spaces}

A probablity space is a triple $(\Omega,\mathcal{F},P)$, where $\Omega$ is a set of "outcomes," $\mathcal{F}$ is a set of "events," and $P:\mathcal \rightarrow$ is a funtion that assigns probabilities to events. We assume that $\mathcal{F}$ is a $\sigma-algebra$.

Without P, ($\Omega,\mathcal{F}$) is called a measurable space, i.e., it is a space on which we can put
a measure. A (positive) measure is a nonnegative countably additive set function.

If $\mu(\Omega)=1$, we call $\mu$ a probability measure. In this book, probability measures are usually denoted by P.

\begin{theo}{Theorem 1.1.1}

Let $\mu$ be a measure on ($\Omega,\mathcal{F}$)

\begin{enumerate}
  \item [(i)] monotonicity. If $A\in B$, then $\mu(A)\le \mu(B)$.
  \item [(ii)] subadditivity. If $A\in \cup_{m=1}^\infty A_m$, then $\mu(A)\le \sum^\infty_{m=1}\mu(A_m)$.
  \item [(iii)] continuity from below. If $A_i\uparrow A$(i.e., $A_1\subset A_2\subset ...$ and $\cup_i A_i=A$), then $\mu(A_i)\uparrow\mu(A)$.
  \item [(iv)] continuity from above. If $A_i\downarrow A$(i.e., $A_1\supset A_2\supset ...$ and $\cap_i A_i=A$), with $\mu(A_1)<\infty$, then $\mu(A_i)\downarrow\mu(A)$.
\end{enumerate}
	
\end{theo}

\begin{example}{Discrete probability spaces}
Let $\Omega$ = a countable set. Let $\mathcal{F}$ = the set of all subsets of $\Omega$. Let	
\begin{displaymath}
  P(A)=\sum_{w\in A}p(w),\enspace where\enspace p(w)>0\enspace and\enspace\sum_{w\in\Omega}p(w)=1
\end{displaymath}

\end{example}

If we are given a set $\Omega$ and a collection $\mathcal{A}$ of subsets of $\Omega$, then there is a smallest $\sigma$-field containing $\mathcal{A}$. We will call this the $\sigma$-field
generated by $\mathcal{A}$ and denote it by $\sigma(\mathcal{A})$.

Let $\textbf{R}^d$ be the set of vectors $(x_1, . . . x_d)$ of real numbers and $\mathcal{R}^d$ be the Borel sets.

\begin{example}{Measures on the real line}
Measures on $(\textbf{R},\mathcal{R})$ are defined by giving a \textbf{Stieltjes measure function} with the following properties:	
\begin{enumerate}
  \item [(i)] F is nodecresing.
  \item [(ii)] F is right continuous, i.e., $lim_{y\downarrow x}F(y)=F(x)$
\end{enumerate}

\end{example}

\begin{theo}{Theorem 1.1.4}
	Associated with each \textbf{Stieltjes measure function} F there is a unique measure $\mu$ on $(\textbf{R},\mathcal{R})$ with $\mu((a,b]) = F(b)-F(a)$
	
	When $F(x)=x$, the resulting measure is called \textbf{Lebesgue measure}.
\end{theo}

A collection $\mathcal{S}$ of sets is said to be a \textbf{semialgebra} if (i) it is closed under intersection,
i.e., S, $T \in S$ implies $S\cap T\in S$, and (ii) if $S \in \mathcal{S}$, then $S^c$ is a finite disjoint union of sets
in $\mathcal{S}$.

A collection $\mathcal{A}$ of subsets of $\Omega$ is called an \textbf{algebra} (or \textbf{field}) if $A,B\in \mathcal{A}$ implies $A^c$ and $A\cup B$ are in A. Since $A\cap B=(A^c\cup B^c)^c$, it follows that $A\cap B\in A$. An algebra is closed under finite unions.

\begin{theo}{Lemma 1.1.7}
	If $\mathcal{S}$ is a semialgebra, then $\overline{\mathcal{S}}$=\{finite disjoint unions of sets in $\mathcal{S}$\} is an
algebra, called the algebra generated by $\mathcal{S}$ .
\end{theo}

By a measure on an algebra $\mathcal{A}$, we mean a set function $\mu $ with

\begin{enumerate}

  \item [(i)] $\mu(A)\ge\mu(\emptyset)$ for all $A\in \mathcal{A}$

  \item [(ii)] if $A_i\in \mathcal{A}$ are disjoint and their union is in $\mathcal{A}$, then
\end{enumerate}

\begin{displaymath}
  \mu(\cup_{i=1}^\infty A_i)=\sum^\infty_{i=1}\mu(A_i)
\end{displaymath}

$\mu$ is said to be $\sigma-$finite if there is a sequence of sets $A_n\in\mathcal{A}$ so that $\mu(A_n)<\infty$ and $\cup_n A_n=\Omega$.

\begin{theo}{Theorem 1.1.9}
Let $\mathcal{S}$ be a semialgebra and let $\mu$ defined on $\mathcal{S}$ have $\mu(\emptyset)=0$. Suppose (i) if $S\in\mathcal{S}$, is a finite disjoint union of set $S_i\in \mathcal{S}$, then $\mu(S)= \sum_i \mu(S_i)$, and (ii) if $S_i,S\in\mathcal{S}$ with $S=+_{i\ge 1}S_i$, then $\mu(S)\le\sum_{i\ge 1}\mu(S_i)$. Then μ has a unique extension $\overline\mu $ that is a measure on $\overline{\mathcal{S}}$, the algebra generated by $\mathcal{S}$. If $\overline\mu $ is sigma-finite, then there is a unique extension $\nu$ that is a measure on $\sigma(\mathcal{S})$.	
\end{theo}

\begin{theo}{Lemma 1.1.10}
Suppose only that (i) holds

\begin{enumerate}
  \item [(i)] If $A,B^i\in \overline{\mathcal{S}}$ with $A=+^n_{i=1}B_i$, then $\overline{\mu}(A)=\sum_i\overline{\mu}(B_i)$
  \item [(ii)] If $A,B_i\in\overline{\mathcal{S}}$ with $A\in\cup^n_{i=1}$, then $\overline{\mu}(A)\le\sum_{i}\overline{\mu}(B_i)$

\end{enumerate}

\end{theo}




$ln(x^a)=ln(e^{ln(x)a})=aln(x)$

$e^{ln(x)}=x$
$\frac{-b\pm\sqrt{b^2-4ac}}{2a}$

\end{document}
