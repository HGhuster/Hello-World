\documentclass{article}
\title{Blanchard Ch.8}
\author{Dawei Wang}
\date{\today}
\usepackage{ctex}
\usepackage{amsmath}
\usepackage{amssymb}
\usepackage{graphicx} %插入图片的宏包
\usepackage{float} %设置图片浮动位置的宏包
\usepackage{subfigure} %插入多图时用子图显示的宏包
\begin{document}
	\maketitle
	菲利普斯曲线(Phillips curve)——描述通货膨胀率和失业率之间的关系。它的发现意味着各国政府可以选择不同失业率和通货膨胀率的组合。如果一国愿意忍受较高的通货膨胀水平,它就可以实现较低的失业水平;或者,如果一国愿意忍受较高的失业水平,它就可以实现价格稳定的零通货膨胀。

\section{通货膨胀、预期通货膨胀和失业}

工资设定等式:

\[
W=P^eF(u,z)
\]

工资设定者选择的名义工资W取决于预期价格水平$ P^e $、失业率u和可能影响工资的其他变量z。

价格决定等式:

\[
P=(1+m)W
\]

之后我们利用这两个关系式,以及实际价格等于预期价格水平这一额外假设,推导出了自然失业率。

放弃实际价格水平等于预期价格水平这一假设:

\[
P=P^e(1+m)F(u,z)
\]

预期价格水平上升将导致名义工资上升,进而使得企业提高价格,导致价格水平上升。失业率上升导致名义工资下降,进而使企业降低价格导致价格水平下降。

假设函数F的特殊形式:

\[
F(u,z)=1-\alpha u+z
\]

得到:

\[
P=P^e(1+m)(1-\alpha u+z)
\]	

通货膨胀率、预期通货膨胀和失业率之间的关系:

\[
\pi=\pi^e+(m+z)-\alpha u
\]

解释:

预期通货膨胀率$ \pi^e $上升导致实际通货膨胀率上升($ \pi^e $的增加$\Rightarrow$$ \pi $的增加):

如果工资设定者预期较高的价格水平(较高的预期通胀率),他们就会设定较高的名义工资水平,从而导致价格水平上升(较高的实际通胀率)。

\hspace*{\fill}

给定预期通货膨胀率$ \pi^e $,价格加成m的增加,或影响工资决定的因素z的增加,导致实际通货膨胀率$ \pi $的增加。

$ m $或z的增加$ \Rightarrow $$ \pi $的增加。

\hspace*{\fill}

给定预期通货膨胀率$ \pi^e $,失业率u的下降导致更高的名义工资,从而导致更高的价格水平P,实际通货膨胀率$ \pi $的上升。

$ u $的下降$ \Rightarrow\pi $的增加。

\hspace*{\fill}

t年的情况:

\[
\pi_t=\pi^e_t+(m+z)-\alpha u_t
\]

\section{菲利普斯曲线及其突变}

早期形式:

假设不同年份的通货膨胀围绕某个值$ \overline{\pi} $波动,并假设通货膨胀不具有持续性,从而本年的通货膨胀并不能很好地预测下一年的通货膨胀。在这种情况下,工资设定者完全有理由假设,不管上一年的通货膨胀水平,本年的通货膨胀都等于$ \overline{\pi} $,即$ \pi^e_t=\overline{\pi} $。

\[
\pi_t=\overline{\pi}+(m+z)-\alpha u_t
\]

这个等式明确揭示了失业和通货膨胀的负相关关系。

\hspace*{\fill}

当上述结论被发表后,早期政府只需在失业率和通胀之间权衡,但随着工资设定者改变了他们形成通货膨胀预期的方式,原始的菲利普斯曲线消失了。这种变化进而带来了通货膨胀行为的变化。通货膨胀变得更加具有持续性。若本年是高通货膨胀,那么下一年更加可能出现高通货膨胀。

假设预期通货膨胀是根据下式形成的:

\[
\pi^e_t=(1-\theta)\overline{\pi}+\theta\pi_{t-1}
\]

$ \theta $值越高,上一年的通货膨胀率就越会促使工人和企业对本年的预期通货膨胀率做出更大修正,预期通货膨胀率也就越高。

通货膨胀持续性——受上一年度通胀水平影响的性质。持续性越强$ \theta $值越大。

\hspace*{\fill}

当$ \theta $等于0,我们可以得到原始的菲利普斯曲线,即失业率和通货膨胀率之间的关系:

\[
\pi_t=\overline{\pi}+(m+z)-\alpha u_t
\]

当$ \theta $为正,通货膨胀率不仅取决于失业率,而且还取决于上一年的通货膨胀率:

\[
\pi_t=[(1-\theta)\overline{\pi}+(m+z)]+\theta\pi_{t-1}-\alpha u_t
\]

当$ \theta=1  $,它们之间的关系变为

\[
\pi_t-\pi_{t-1}=(m+z)-\alpha u_t
\]

因此,当$ \theta $等于1,失业率影响的不是通货膨胀率,而是通货膨胀率的变化:高失业导致通货膨胀下降,低失业导致通货膨胀上升。

通常将上式称作修正的菲利普斯曲线(modified Phillips curve),或者附加预期的菲利普斯曲线(expectations-augmented Phillips curve,以说明$ \pi_{t-1} $表示预期通货膨胀),或者加速的菲利普斯曲线(accelerationist Phillips curve,以说明低的就业率将导致通货膨胀率的增加,由此加速了价格水平的上升)。我们简单地将该式称为菲利普斯曲线,而将早先的形式称为原始的菲利普斯曲线。

\section{菲利普斯曲线和自然失业率}

原始的菲利普斯曲线意味着根本不存在自然失业率这回事,即如果政策制定者愿意承受较高的通货膨胀率,那么就能长时间维持较低的失业率。

弗里德曼和菲利普斯认为,只有在工资设定者系统地低估预期通货膨胀的前提下,这种均衡关系才能存在,但这种系统性的低估是不可维系的,如果政府试图通货接受较高的通货膨胀率来维持较低的失业率,那么这种权衡最终会消失,失业率不可能一直低于“自然失业水平”。

\hspace*{\fill}

根据定义,自然失业率是实际价格水平等于预期价格水平时的失业率。同理,自然失业率也可以是实际通货膨胀率等于预期通货膨胀率$ \pi=\pi^e $时的失业率。

\[
0=(m+z)-\alpha u_n
\]

解得自然失业率$ u_n $:

\[
u_n=\frac{m+z}{\alpha}
\]

将$ \pi_t=\pi^e_t+(m+z)-\alpha u_t $改写为

\[
\pi_t-\pi_t^e=-\alpha(u_t-\frac{m+z}{\alpha})
\]

\[
\pi_t-\pi^e=-\alpha(u_t-u_n)
\]

如果预期通货膨胀率$ \pi^e_t $非常接近上一年的通货膨胀率$ \pi_{t-1} $,那么等式最终变为:

\[
\pi_t-\pi_{t-1}=-\alpha(u_t-u_n)
\]

因此可以将菲利普斯曲线看作实际失业率u、自然失业率$ u_n $和通货膨胀率变化$ \pi_t-\pi_{t-1} $之间的关系。

通货膨胀率的变化取决于实际失业率和自然失业率之间的差异。当实际失业率高于自然失业率时,通货膨胀率下降;反之,当实际失业率低于自然失业率时,通货膨胀率上升。

同时也给了我们另一种思考自然失业率的方式:

自然失业率是使通货膨胀率保持不变的失业率,这就是为什么将自然失业率称为无附加通货膨胀的失业率(nonaccelerating inflation rate of unemployment, NAIRU)的原因。

\section{总结与警示}

通货膨胀和失业在不同国家与不同时期可能有不同的表现。

什么能解释欧洲的失业:

慷慨的失业保险系统;

高度的就业保护;

最低工资;

议价规则。

%


\subsection{不同时期自然失业率的变化}

我们完全有理由相信$ m+z $可能随时间变化而变化。自然失业率随时间的变化是难以衡量的,原因在于我们无法观察到自然失业率,我们只能观察到实际失业率。但是可以通过比较平均失业率来大体衡量它的变化。自然失业率由多种因素决定,我们可以识别出其中一些因素,但是了解它们各自的作用机制并获得政策经验就不那么简单了。

\hspace*{\fill}

美国自然失业率的变化:

全球化的加剧以及美国企业和外国企业之间的竞争加剧可能导致其垄断势力和价格加成的下降。同样地,企业更容易将其部分运营业务转移到国外去,这使企业在与工人谈判时更具优势。美国工会势力的弱化也是不争的事实。

劳动力市场的本质发生了变化,通过临时救助机构获得工作的比例上升,基于互联网的招聘网站的作用日益提升也使工人和工作机会之间的匹配变得更加容易。

人口的老龄化:年轻人(更爱跳槽)减少,失业率下降;囚犯增加;残疾工人数量增加。

\subsection{高通货膨胀和菲利普斯曲线关系}

普遍的经验是:通货膨胀和失业的关系很可能随着通货膨胀水平及其持续时间的变化而变化。不仅工人和企业形成预期的方式会发生变化,制度安排也会发生变化。

当通货膨胀率升高,通货膨胀也变得更加多变。结果是,工人和企业更加不愿意签订决定长时间名义工资的劳动合同。鉴于这个原因,工资协议的期限应随通货膨胀水平的变化而变化。名义工资的时间更短,从一年下降到一个月甚至更短。工资指数化是一种工资自动随通货膨胀增加的制度安排。

假如一个经济有两种类型的劳动合同,比例$\lambda$的劳动合同是指数化的,即这部分合同的名义工资随实际价格水平的变化而变化;而比例1-$\lambda$的劳动合同没有经过指数化,即名义工资的设定基于预期的通货膨胀。

\[
\pi_t=[\lambda\pi_t+(1-\lambda)\pi_t^e]-\alpha(u_t-u_n)
\]

假设本年的预期通货膨胀等于上一年的实际通货膨胀,即$ \pi^e_t=\pi_{t-1} $,那么我们可以得到:

\[
\pi_t=[\lambda\pi_t+(1-\lambda)\pi_{t-1}]-\alpha(u_t-u_n)
\]

整理可得:

\[
\pi_t-\pi_{t-1}=-\frac{\alpha}{1-\lambda}(u_t-u_n)
\]

工资指数化加剧了失业对通货膨胀的影响。指数化的工资合同比例越高,即$\lambda$越高,失业率对通货膨胀的变化的影响越大,即系数$ \alpha/(1-\lambda) $越大。如果$\lambda$接近于1,即几乎所有工资合同都允许工资指数化,失业的小幅波动就会导致通货膨胀的大幅变动。高通货膨胀国家的通货膨胀和失业之间的关系变得越来越弱,并且最终会消失。

\subsection{通货紧缩和菲利普斯曲线的关系}

高失业率和高通胀并存的可能原因:

1. 伴随着大萧条出现的不仅是实际失业率的增加,而且是自然失业率的上升。

2. 当经济开始经历通货紧缩,菲利普斯曲线关系便瓦解了。一个可能的原因是工人不愿意接受名义工资的下降。




\end{document}