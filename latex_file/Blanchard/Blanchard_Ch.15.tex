\documentclass{article}
\title{Blanchard Ch.15}
\author{Dawei Wang}
\date{\today}
\usepackage{ctex}
\usepackage{amsmath}
\usepackage{amssymb}
\usepackage{graphicx} %插入图片的宏包
\usepackage{float} %设置图片浮动位置的宏包
\usepackage{subfigure} %插入多图时用子图显示的宏包
\begin{document}
	\maketitle

\section{消费}

预期会影响消费决策。Milton Friedman——消费的持久收入理论;Franco Modigliani 消费的生命周期理论。

Friedman的的“持久收入”强调消费者不仅仅考虑当前收入,Modigliani的“生命周期”强调消费者自然的消费计划范围是其一生。

\subsection{深谋远虑的消费者}

消费者把所有的股票和债券价值、支票和储蓄账户的价值、房产的价值加总,再减去未到期的抵押价值。这可以使他明确自己的金融财富和房产财富(不仅包括房产,也包括消费者可能拥有的其他商品,从汽车到字画等。)

他也会估计工作年限内获得的税后收入,并计算其预期贴现值。这个就是经济学家所谓的人力财富(human wealth),区别于非人力财富(nonhuman wealth)。非人力财富就是金融财富和房产财富的加总。

人力财富+非人力财富=总财富。

假设消费者将总财富的一定比例用于支出,他决定的总财富的支出比例可以使其再生命周期内保持基本相同的消费水平。如果该消费水平高于当期收入,差额部分将选择借贷消费;如果低于当期收入,差额部分储蓄。

消费决策的形式:

\[
C_t=C(total\enspace wealth_t)
\]

此时$ C_t $表示t时期的消费,$ C_t=C(total\enspace wealth_t) $表示t时期人力财富(t时期当前和未来税后劳动收入的预期现值)和非人力财富(金融财富和房产财富)的加总。

注:对于一个普通的消费者来说,这样的假设太过算计并要求富有远见。

\subsection{更现实的描述}

1. 可能并不希望再生命周期内保持固定的消费水平;

2. 我们的决策水平可能不够;

3. 总财富计算是基于对将来可能发生变化的预期;

4. 可能很难找到银行借款。

我们在做消费决策时可能仅仅考虑了当前收入,而没有考虑自己的财富。用税后劳动收入度量当前收入。

\[
C_t=C(Total\enspace wealth_t,Y_{Lt}-T_t)
\]

消费是关于总财富和当前税后劳动收入的增函数。

消费在多大程度上取决于总财富取决于消费者本身,大多数消费者是富有远见的。但一些消费者若暂时收入较低、资信比较差,无论他预期未来会发生什么,都只能消费当前收入。

\subsection{综合考虑:当前收入、预期和消费}

预期从下面两个方面影响消费:

预期通过人力财富直接影响消费:为了计算人力财富,消费者必须对未来劳动收入、实际利率以及税款形成预期;

预期通过非人力财富间接影响消费。

未来更高的预期如何影响现在的消费:

预期未来产出上升$ \Rightarrow $预期未来劳务收入增加$ \Rightarrow $人力财富增加$ \Rightarrow $消费增加。

预期未来产出增加$ \Rightarrow $预期未来股息增加$ \Rightarrow $股票价格上升$ \Rightarrow $非人力财富增加$ \Rightarrow $消费增加。


消费和收入间的关系:

消费的波动往往比当期收入波动幅度要小;

即使当前收入不变,消费也可能会发生变化。

短期$ C=c_0+c_1Y $。这意味着当收入提高的时候,消费也提高,但幅度较小(C/Y下降)。

考虑长期,假定$ S=sY $,或者等价于$ C=(1-s)Y $。这意味着当收入提高的时候,消费等比例提高。当我们关注长期变动时,这是恰当的。

\section{投资}

投资不仅取决于当前利率和当前销售水平,同时还取决于未来的预期。

\subsection{投资和预期利润}

折旧:

假设一台机器每年损失的可用性比率为$ \delta $。今年的一台机器在今年仅值$ (1-\delta) $,两年后值$ (1-\delta)^2 $,依次类推。

预期利润的现值

假设在t年购入的机器投入运转,开始折旧需要在一年之后。每年以实际值计算的机器利润记为$ \Pi $。该机器会在t+1年产生第一笔利润。这笔利润记作$ \Pi $。

如果公司在t年购买机器,该机器会在$ t+1 $年产生第一笔利润。这笔预期利润记作$ \Pi^e_{t+1} $。该预期利润在t年的现值由下式表示:

\[
\frac{1}{1+r_t}\Pi^e_{t+1}
\]

这台机器在t+2年产生的预期利润记作$ \Pi^e_{t+2} $。由于存在折旧,机器在t+2年的价值仅剩$ (1-\delta) $。因此,这台机器的预期利润现值为:

\[
\frac{1}{(1+r_t)(1+r^e_{t+1})}(1-\delta)\Pi^e_{t+2}
\]

则在t年购买一台机器获得的预期利润的现值为:

\[
V(\Pi^e_t)=\frac{1}{1+r_t}\Pi^e_{t+1}+\frac{1}{(1+r_t)(1+r^e_{t+1})}(1-\delta)\Pi^e_{t+2}+\cdots
\]

投资决策

令$ I_t $表示总投资

将经济中每单位资本的利润记作一个整体,记作$ \Pi_t $。

每单位资本的预期利润现值记作$ V(\Pi^e_t) $,则投资函数具有以下形式:

\[
I_t=I[V(\Pi^e_t)]
\]

总结:投资与未来利润的预期现值成正比。当期或预期利润越高,预期现值和投资水平越高。当期或预期实际利率越高,预期现值就越低,从而投资水平就越低。

公司面临一个简单问题:将每增加一单位资本所需花费的成本与市场愿意支付的股票价格做比较。如果股票价格超过购买价格,公司应该投资,否则不应该。

托宾q理论:

q=总市值/资本存量的重置成本

这个比率有效地告诉我们每单位资本相对于当前购买价格的相对价值,直观上讲,q值越高,当前买价相对于资本价值越高,就越应该投资。

托宾q值和股票价格存在联系,这是因为投资决策和股票价格在很大程度上都取决于相同因素——未来预期利润和未来预期利率。

\hspace*{\fill}

假设公司预期未来的利润和利率保持与今天同样的水平,因此:

\[
\Pi^e_{t+1}=\Pi^e_{t+2}=\cdots=\Pi^e_{t},and\enspace r^e_{t+1}=r^e_{t+2}=\cdots=r^e_{t}
\]

经济学家称这样的预期值,也就是未来值等于当前值为静态预期,此时:

\[
V(\Pi^e_t)=\frac{\Pi_t}{r_t+\delta}
\]

投资等于:

\[
I_t=I(\frac{\Pi_t}{r_t+\delta})
\]

实际利率与折旧率之和被称作使用者成本(user cost),或者资本的租贷成本(rental cost)。

即使公司在一般情况下不会租贷它们所使用的机器,$ (r_t+\delta) $仍旧代表了公司使用机器一年所暗含的成本,有时称为影子成本。

投资取决于利润和使用者成本的比值,利润越高,投资水平越高。使用者成本越低,投资水平越低。

\subsection{当前利润和预期利润}

现实中投资变化和当期利润变化显著正相关。如果公司预期未来利润与当期利润变动非常一致,那么这些未来利润的现值将会于当期利润非常一致,从而投资也就与之一致。

当期利润在投资决策中起作用的原因:

\hspace*{\fill}

1. 如果当期利润较低,一家公司需要通过借钱来筹集购买新机器的资金。公司也许并不愿意借钱:即使预期利润看起来很高,但一旦情况变糟糕,该公司就无法支付债务。

如果公司的当前利润很高,公司就可以通过保留一部分收益进行筹款,而不必去借款。总的来说,更高的当前收益可能导致公司进行更多投资。

\hspace*{\fill}

2. 即使公司想要投资,也可能存在借款困难的问题。如果公司有大量的当前利润,它便可以不必借钱,也不必去说服潜在的贷款人。公司可以按照其意愿进行投资,并倾向于这样做。

\hspace*{\fill}

总的来说,为了与我们实际观察到的投资行为相符合,投资等式最好表示为:

\[
I_t=I[V(\Pi^e_t),\Pi_t]
\]

总结:投资即与未来利润的预期现值有关(获利能力),也与当前的利润水平(现金流)有关。

\subsection{利润和现金流}

决定每单位资本的利润的主要有两方面:销售水平以及现存的资本存量。如果销售额水平相对于资本存量比较低,那么每单位资本的利润也就比较低。

忽略销售额和产出之间的区别,令$ Y_t $表示产出,等价地就是销售额。令$ K_t $表示t时期的资本存量。我们的讨论说明了以下关系:

\[
\Pi_t=\Pi(\frac{Y_t}{K_t})
\]

单位资本利润的变动与产出资本比值的变动之间存在紧密联系。如果大多数产出资本比值的变化都源自产出的变动,那么大多数单位资本利润的年变化源自利润的变动。

产出和利润之间存在的联系。一方面是当前产出和预期未来产出,另一方面是投资。当前产出影响当前利润,预期未来产出影响预期未来利润;当前利润和预期未来利润影响投资。

高预期产出$ \Rightarrow $高预期利润$ \Rightarrow $今天的高投资率。

\section{消费和投资的波动性}

消费者意识到收入的变动是短暂的还是永久的,将影响他们的消费决策。他们预期收入的提高持续时间越短,就越难以提高自己的消费水平。

同样地,一家公司预期当前销售额的变动是暂时还是永久的,将影响它的投资决策。公司预期销售额的提高持续时间越是短暂,就越难以改变对利润现值的估计,从而就越不会购买新的机器或者建造新的工厂。

\hspace*{\fill}

消费决策和投资决策之间的区别:

拓展的消费理论意味着,当消费者面对收入的永久性上升时,他们最多只是提高等量的消费。收入的永久性提高意味着他们能支付现在和将来的收入增长等量的消费量。如果消费的提高量超过收入的增长,那么将来就要削减消费。

然而一家公司面临销售额的永久增长时,预期利润的现值增长,会带来投资的上升。然而,与消费相比较,这并不简单地意味着最多只是增加与销售额的提高等量的投资。一旦公司认为销售额的提高有利于做出购买新机器或者建设新工厂的决策,该公司就会尽快去实施,从而导致投资支出发生较大但是短暂的提高。投资支出的提高可能会超出销售额的提高量。

这些差别说明了投资的波动性比消费更大。

消费和投资通常都是同步变动的。在经济衰退时,投资和消费会同时下降。

投资的波动范围比消费的波动大得多。

投资对产出波动的贡献大致相同。








	
\end{document}