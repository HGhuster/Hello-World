\documentclass{article}
\title{Blanchard Ch.20}
\author{Dawei Wang}
\date{\today}
\usepackage{ctex}
\usepackage{amsmath}
\usepackage{amssymb}
\usepackage{graphicx} %插入图片的宏包
\usepackage{float} %设置图片浮动位置的宏包
\usepackage{subfigure} %插入多图时用子图显示的宏包
\begin{document}
	\maketitle

\section{中期}

在浮动汇率制度下,要实现汇率的实际贬值,可以利用扩张性的货币政策来实现利率降低和汇率降低。

在固定汇率制下,调整汇率于利率的工具失灵。按照固定汇率的定义,名义汇率必须是固定的,这样该国就不能调整名义汇率。同时,固定汇率和利率平价条件意味着该国也不能调整其利率,国内利率必须和外国利率保持相等。

尽管我们关于短期的结论是正确的,但是在中期这两种制度间的差异会消失,无论是采取固定汇率制度还是浮动汇率制度,经济在中期都将达到相同的实际汇率和相同的产出水平。

\[
\epsilon=\frac{EP}{P^*}
\]

因此存在两种途径来实现对实际汇率的调整。

1. 通过名义汇率E来改变,从定义上来看,这只有在浮动汇率制下才能实施。如果我们假设国内价格水平P和外国价格水平$ P^* $在短期内不发生改变的话,改变名义汇率是短期内唯一可以变动实际汇率的途径;

2. 通过国内价格水平P相对于外国价格水平$ P^* $的改变,在中期,这个选择甚至对于那些实行固定汇率的国家也是可行的,并且这确实是固定汇率制下所发生的情况:实际汇率的调整是通过价格水平而不是名义汇率来进行的。

\subsection{固定汇率下的IS关系}

在一个实行固定汇率的开放经济中,我们可以将IS关系写为:

\[
Y=Y(\frac{\overline{E}P}{P^*},G,T,i^*-\pi^e,Y^*)
\]

产出与

实际汇率$ \frac{\overline{E}P}{P^*} $负相关;

正向取决于政府支出G和税收T;

负向取决于国内实际利率,它等于国内名义利率减去预期通胀率。在利率平价条件和固定汇率制度下,国内名义利率等于国外名义利率$ i^* $,所以国内实际利率为$ i^*-\pi^e $。

正向取决于国外产出$ Y^* $,通过对出口的影响。

\subsection{短期均衡和中期均衡}

考虑一个实际汇率过高的经济体。其结果是,贸易平衡出现赤字,产出则低于潜在产出水平。

在短期,浮动汇率制下可以通过降低利率来实现名义贬值。给定国内外价格水平,名义贬值就意味着实际贬值,贸易平衡改善,并且产出增加。

在固定汇率制下,中央银行不能改变国内利率,所以短期贸易赤字仍然存在,国内经济也处于衰退中。

在中期,价格是可以调整的。

\[
\pi-\pi^e=(\alpha/L)(Y-Y_n)
\]

当产出高于潜在产出水平的时候,通货膨胀率高于预期。当产出低于潜在产出水平时,通货膨胀率低于预期。

假设预期通货膨胀率保持不变,所以菲利普斯曲线关系可以表示为:

\[
\pi-\overline{\pi}=(\alpha/L)(Y-Y_n)
\]

假设外国通胀率为$ \pi^* $。假设产出等于潜在产出,国内外通胀率相等,并且都等于$ \overline{\pi} $,所以$ \pi=\pi^*=\overline{\pi} $

假设产出低于潜在产出,这就意味着国内通货膨胀率低于潜在产出时的通货膨胀率,因而也就低于外国通货膨胀率。换言之,国内价格水平相比于外国价格水平而言增加得更慢。这意味着在名义汇率固定的情况下,实际汇率下降。结果净出口和产出都随着时间增加。在中期,产出恢复到潜在产出水平,国内通货膨胀率恢复到$ \overline{\pi} $,因此与外国通货膨胀率相等。因为国内外通货膨胀率相等,所以实际汇率保持不变。

\hspace*{\fill}

总结:

1. 在短期,固定名义汇率意味着固定实际汇率;

2. 在中期,即使名义汇率是固定的,实际汇率也能调整。这个调整是通过价格水平的变动实现的。

\subsection{赞成和反对低估的情形}

即使在固定汇率制度下,经济在中期也能回到自然产出水平,但是这样的调整过程可能很漫长痛苦。在相当长的一段时间内产出很可能维持在很低的水平,而失业率维持在一个相当高的水平。

假设政府尽管要维持固定汇率制,但决定允许一次性低估,这在短期将导致实际贬值,从而导致产出增加。原则上,适当规模的低估在短期就可以实现上述分析中只有中期才能实现的效果,因此避免了很多痛苦。

反对转向浮动汇率制或者反对低估的人指出选择固定汇率制有其充分的理由,过多的低估倾向首先会违背采取固定汇率制的初衷,他们认为政府过多地考虑低估会使汇率为记得可能性增加。

\section{固定制度下的汇率危机}

假设一国已决定采用固定汇率制,并假设金融投资者开始认为不久后将会有汇率的调整——低估。或向浮动汇率制度转变并且伴随着贬值。可能出现这种状况的原因为:

1. 实际汇率可能太高了。换言之,本币可能被高估了,导致了巨额的经常账户赤字。在这种情况下就会要求实际贬值。尽管在中期并不需要低估就可以实现,但金融投资者还是会认为政府会采取最快速的办法,即低估。

这种过高的估值通常发生在那些实施钉住名义汇率的国家,且被钉住国的通货膨胀率较低。相对于被钉住国,本国的通胀率较高,因此本币持续升值。


2. 内部经济条件可能会要求降低本国利率。在固定汇率制下,国内利率无法降低,但是如果国家愿意变成浮动汇率制就可以实现国内利率的下降。如果本国让汇率浮动,同时降低国内利率,就可以实现名义贬值。

只要金融市场认为低估很快将到来,要维持汇率就要提高国内利率,而且通常是大幅度提高。

\[
i_t=i_t^*-\frac{E^e_{t+1}-E_t}{E_t}
\]


在固定汇率制度下,当前汇率$ E_t $固定在某个水平上,例如$ E_t=\overline{E} $。如果市场预期该平价会在一段时间内保持不变,那么$ E^e_{t+1}=\overline{E} $,此时利率评价关系表明国内利率必须等于国外利率。

假设金融市场参与者开始预期一个低估——中央银行放弃利率平价,并在未来降低汇率。这意味着如果中央银行想要保持当前的平价,现在就必须使利率比之前至少高(忽略风险溢价):

\[
\frac{E_t-E^e_{t+1}}{E_t}
\]

此时政府和中央银行的选择有:

1. 政府和中央银行让市场相信它们并没有低估的意图;

2. 中央银行可以提高利率,但可以比上面的幅度小一些。虽然国内利率很高,但并没有高到足以完全补偿可以预见到的低估风险的程度。这一举措往往会导致大规模的资本流出,因为金融投资者仍倾向于将本国债券转换成外国债券,这是由于后者可以提供更高的用本币表示的收益。如果中央银行不干预外汇市场,大规模的本币抛售购买外币将导致本币贬值。如果中央银行不干预外汇市场,大规模的本币抛出购买外币将导致本币贬值。如果中央银行希望维持汇率,它必须在现行汇率水平下买入本国货币并卖出外国货币。这样做往往会损失大部分外汇储备。

3. 把利率提高到足以满足利率平价要求的水平,或者实施低估,从而证实市场的预期。制定过高的短期国内利率会产生灾难性影响,这种措施只在以下两种情况下起作用:1. 预期低估的可能性较小,所以利率不必太高;2. 政府认为市场很快就会相信低估并不会发生,使国内利率可以降下来。否则唯一的选择是低估。

总结:预期将会发生低估会引发汇率危机,面对这样的预期,政府有两种选择:

1. 让步并实施低估;

2. 斗争并维持汇率平价,但要以非常高的利率和潜在的衰退为代价。

即使相信低估即将到来的想法最初是毫无根据的,但低估也可能会真的发生。

\section{浮动汇率制下的汇率波动}

在前面建立的模型中,利率和汇率之间有一个简单的关系:利率越低,汇率也越低。这就意味着一个国家想要维持一个稳定的汇率只要保持其利率接近国外利率即可。一个国家想要达到一个给定贬值,只需要将其利率将第一个合适的量。

在现实中,即使利率不变动,汇率也经常波动。一定量的利率降低对汇率的影响程度往往难以预测,这就使货币政策达到其预定目标更加困难。

考虑利率平价条件:

\[
(1+i_t)=(1+i^*_t)\frac{E_t}{E^e_{t+1}}
\]

即:

\[
E_t=\frac{1+i_t}{1+i^*}E^e_{t+1}
\]

则第$ t+1 $年的情况为:

\[
E_{t+1}=\frac{1+i_{t+1}}{1+i^*_{t+1}}E^e_{t+2}
\]

因此$ t+1 $年的预期汇率为:

\[
E_{t+1}^e=\frac{1+i_{t+1}^e}{1+i^*_{t+1}}E^e_{t+2}
\]

故:

\[
E_t=\frac{(1+i_t)(1+i^e_{t+1})}{(1+i^*_t)(1+i^{*e}_{t+1})}E^e_{t+2}
\]

推广至第$ t+n $年:

\[
E_t=\frac{(1+i_t)(1+i^e_{t+1})\cdots(1+i^e_{t+n})}{(1+i^*_t)(1+i^{*e}_{t+1})\cdots(1+i^{*e}_{t+n})}E^e_{t+n+1}
\]

这个关系告诉我们当前汇率取决于两组因素:

1. 接下来n年每一年的当前和预期国内外利率;

2. n年后的预期汇率。

当前汇率水平会随着未来预期汇率1:1变动;

当未来预期利率在两国的任意一国中有变动时,当前汇率也会变动;

因为今天的汇率会随着预期的任何变动而变动,汇率将是不稳定的,也就是说变动得很频繁,而且变动幅度也许会很大。

\subsection{汇率和经常账户}

改变预期未来汇率$ E^e_{t+n} $的因素都将改变当前汇率$ E_t $。预期未来汇率的任何变动对当前汇率的影响都是一一对应的。

考虑一个较大的n值,可以考虑将$ E_{t+n}^e $视为在中长期达到经常账户平衡时所需要的汇率。这样任何影响未来经常账户平衡预期的新闻很可能对预期未来汇率产生影响,反过来对今天的汇率产生影响。例如,发布比预期贸易赤字大的通告可能导致投资者得出重新实现贸易平衡最终需要贬值的结论。因此,$ E^e_{t+n} $将下降,这也使$ E_t $下降。

\subsection{汇率、当前和未来的利率}

在t到$ t+n $年任何改变当前或者预期未来国内外利率的因素也将改变当前汇率。这意味着任何能促使投资者改变他们对于未来预期利率的变量都将导致今天汇率的变动。

\subsection{汇率波动}

利率$ i_t $和汇率$ E_t $之间的关系并不是机械性的。当中央银行降低利率时,金融市场会评估这个行动是否时货币政策转变的信号,以及这次削减利率是否是后续进一步削减的第一步,还是只是利率的临时性变化。

很多汇率的波动可以由金融市场关于未来利率和未来汇率新闻的理性反应来解释,这有非常重要的意义:一个决定实行浮动汇率制的国家必须接受一个事实,即该国将会随时暴露在巨大的汇率波动中。

\section{汇率制度的选择}

在中期汇率制度并不重要,但在短期还是很重要的。在短期,实行固定汇率制并且资本玩完全流动的国家将放弃两个宏观工具:利率工具和汇率工具。这不仅降低了对经济冲击反应的能力,而且也可能带来汇率危机;

在一个固定汇率制的国家,低估的预期将使投资者要求一个更高的利率。这反过来又促使经济情况变得更糟糕,也会让国家背负更多的低估压力。这是反对固定汇率的另一个论点;

反对浮动汇率的观点:在浮动汇率下,汇率很可能大幅波动并且很难通过货币政策控制。

\hspace*{\fill}

一般而言,浮动汇率要更好一些,也有两个意外:

1. 当一组国家已经紧密融合时,共同货币也许是个正确的选择;

2. 当人们不相信中央银行会在浮动汇率制下遵循一个负责人的货币政策的时候,固定汇率制也许更有利。

\subsection{共同货币区}

实施固定汇率制的国家被约束在相同的利率水平。对于那些组成最优货币区(optimal currency area)的国家,它们需要满足以下两个条件之一:

1. 国家必须经历相同的经济冲击。如果它们经历相同的冲击,无论如何它们本来就是应该选大致一样的货币政策;

2. 如果国家经历了不同的冲击,它们之间必须有很高的要素流动性。例如,如果工人愿意从不景气的国家迁移到经济状况较好的国家,要素流动性而不是宏观经济政策就能使国家对冲击做出调整。

使用共同货币有很多优势。1. 降低交易成本;2. 增加竞争。

\subsection{硬钉住、货币发行局制度和美元化}

可能在某些情况下,一个国家希望限制使用货币政策的能力。

若使用固定汇率制,在金融市场预期该平价会被保持的情况下,他们就不会再担心货币当局用货币增长来弥补财政赤字。

\hspace*{\fill}

固定汇率必须作为一个更一般的宏观经济政策组合的一部分。如果使用固定汇率,但是继续保持巨大的财政赤字,那么只会使金融市场参与者相信货币增长会再度开始,低估很快会发生;

不管是象征意义上还是从技术上使平价难以改变,一种途径是被称为硬钉住(hard peg)的制度。

\hspace*{\fill}

硬钉住的一种极端形式是简单地使用外国货币取代本国货币。使用美元则称为美元化(dollarization)。但是很少有国家愿意放弃本国货币而采用另一个国家的货币。

一种稍缓和的办法是货币发行局(currency board)制度,在这种制度下,中央银行随时准备以官方汇率买入或者卖出外国货币;而且中央银行不能参与公开市场,即不能买卖政府债券,这一点正是与标准的固定汇率制的差异所在。

































































































































\end{document}
