\documentclass{article}
\title{Blanchard Ch.22}
\author{Dawei Wang}
\date{\today}
\usepackage{ctex}
\usepackage{amsmath}
\usepackage{amssymb}
\usepackage{graphicx} %插入图片的宏包
\usepackage{float} %设置图片浮动位置的宏包
\usepackage{subfigure} %插入多图时用子图显示的宏包
\begin{document}
	\maketitle

短期中政府支出和税收如何影响需求和产出:

短期财政扩张,即政府支出的增加或税收的减少导致产出增加;

\hspace*{\fill}

财政政策对产出和利率的短期效应:

财政紧缩导致可支配收入减少从而导致消费减少。需求减少通过乘数效应进一步导致产出和收入减少。因此在给定的政策利率下,财政紧缩导致产出下降。然而,中央银行降低政策利率可以部分抵消财政紧缩的不利影响。

\hspace*{\fill}

当经济处于流动性陷阱时,降息不能带来产出增加。因此财政政策可以发挥重要作用。

\hspace*{\fill}

在中期(资本存量既定),财政整顿对产出没有影响,只会简单地反映在支出构成不同上。然而从短期来看,产出下降。换句话说,如果产出处于潜在状态,尽管出于其他原因,财政整顿一开始可能会导致经济衰退。

\hspace*{\fill}

一旦考虑了资本积累,大规模的预算赤字就意味着低水平地国民储蓄率,从而导致长期产出的低下。

\hspace*{\fill}

考虑财政政策对预期的影响:

赤字降低对产出的影响取决于对将来的财政政策和货币政策的预期。

\hspace*{\fill}

财政政策如何影响产出和贸易余额,预算赤字和贸易赤字的关系。

\hspace*{\fill}

产品市场和金融市场均开放的条件下,财政政策的作用:

在存在资本流动的情况下,财政政策效果依赖于汇率制度。与浮动汇率制相比,在固定汇率制下,财政政策对产出的影响更大。

\section{政府预算约束:赤字、债务、政府支出与税收}

\subsection{赤字和债务的计算}

赤字和债务的区别:

债务是一个存量,即政府过去历年积累赤字的结果。赤字是一个流量,即政府在给定的一年内的借款额。

\hspace*{\fill}

第t年的预算赤字:

\[
deficit_t=rB_{t-1}+G_t-T_t
\]

所有的变量都是实际值

$ B_{t-1} $是第t-1年年底,或者t年年初的政府债务;r是实际利率,这里假设为常量。因此,$ rB_{t-1} $就等于政府在第t年对债务的实际利息支付;

$ G_t $是第t年政府对商品和服务的支出;

$ T_t $是第t年的税收减去转移支付后的净额。

注:

我们度量利息支付使用的是实际利息支付,即实际利率与债务存量的乘积,而非现行利息支付,即名义利率与债务存量的乘积。官方的赤字度量使用现行(名义)利息支付。度量赤字的正确方法有时称作通货膨胀调整后的赤字(inflation-adjusted deficit)。

G的定义中不包括转移支付。转移支付从T中减去,因此T就表示为税收减去转移支付的净额。官方的定义里转移支付在G中被扣除而不是T中。

政府预算约束(government budget constraint):

\[
B_t-B_{t-1}=deficit_t
\]

如果政府运行出现赤字状态,政府债务将增加;如果政府处于盈余状态,政府债务将减少。

运用赤字的定义,我们可以将政府预算约束写作:

\[
B_t-B_{t-1}=rB_{t-1}+G_t-T_t
\]

将赤字分解为以下两个部分之和:

对债务的利息支付,即$ rB_{t-1} $;

支出和税收之差,即$ G_t-T_t $。这个差用基础赤字表示(primary deficit),相应地$ T_t-G_t $称作基础盈余(primary surplus)。

利用这一分解得:

\[
change\enspace in\enspace the\enspace debt=Interest\enspace Payments+Primary\enspace Deficit
\]

或:

\[
B_t=(1+r)B_{t-1}+(G_t-T_t)
\]

\subsection{当前和未来税收}

假设第一年年底债务$ B_1=1 $。

如果政府打算在第2年偿还清债务。根据式:

\[
B_2=(1+r)B_1+(G_2-T_2)
\]

得到:

\[
T_2-G_2=(1+r)\times 1=(1+r)
\]

如果政府打算在第t年一次性还清:

\[
T_t-G_t=(1+r)^{t-1}
\]

结论:

如果政府支出不变,当前税收减少最终需要通过将来的税收增加来抵消;

政府为抵消而 增加税收等待的时间越久,或者实际利率越高,最终需要增加的税收越多。

\hspace*{\fill}

第t年债务稳定

如果政府仅仅保持债务存量不变,税收的变化,假设政府决定从第2年起保持债务不变($ B_2=B_1=1 $):

\[
B_2=(1+r)B_1+(G_2-T_2)
\]

\[
T_2-G_2=(1+r)-1=r
\]

为了避免债务在第1年进一步增加,政府必须使基础盈余等于对当前债务的实际付息额。在接下来各年的时间里政府也必须这样做:每年,基础盈余必须足够支付利息从而保持债务水平不变。

无论政府何时开始保持债务不变,政府都必须从那时起使基础盈余维持在足够偿还债务利息的水平。

\hspace*{\fill}

结论:

过去政府赤字将带来更高的政府债务;

为了保持债务不变,政府必须清除赤字;

为了清除赤字,政府的基础盈余必须等于对当前债务的利息支付。这要求永久性的税收增加。

\subsection{债务占GDP比率的演变}

\[
\frac{B_t}{Y_t}=(1+r)\frac{B_{t-1}}{Y_t}+\frac{G_t-T_t}{Y_t}
\]

即:

\[
\frac{B_t}{Y_t}=(1+r)(\frac{Y_{t-1}}{Y_t})\frac{B_{t-1}}{Y_{t-1}}+\frac{G_t-T_t}{Y_t}
\]

将第t年的产出增长率表示为g,因此$ Y_{t-1}/Y_t=1/(1+g) $。同时运用近似计算:$ (1+r)/(1+g)=1+r-g $。

因此:

\[
\frac{B_t}{Y_t}=(1+r-g)\frac{B_{t-1}}{Y_{t-1}}+\frac{G_t-T_t}{Y_t}
\]

即:

\[
\frac{B_t}{Y_t}-\frac{B_{t-1}}{Y_{t-1}}=(r-g)\frac{B_{t-1}}{Y_{t-1}}+\frac{G_t-T_t}{Y_t}
\]

上式的解释:一段时间债务比率的变化等于下列两项之和:

1. 实际利率和增长率之差与初始债务比率的乘积;

2. 基础赤字占GDP的比率。

\hspace*{\fill}

下列情况将会导致债务占GDP的比率增加:

1. 实际利率越高;

2. 产出增长率越低;

3. 初始的债务比率越大;

4. 基础赤字占GDP的比率越大。

\section{李嘉图等价、周期性赤字调整和战争筹资}

李嘉图等价(Ricardian equivalence)(李嘉图——巴罗命题,Ricardo-Barro proposition):一旦考虑了政府的预算约束,无论是赤字还是债务都不会对经济活动产生影响。

假设政府在今年将税收减少1,而且在减税的同时宣布为了偿还债务,政府将在下一年增加$ (1+r) $税收。

如果消费者意识到了今年的低税收正好等于明年的高税收的现值,则减税没有影响。

从另一个角度讲:消费者在减税后不改变消费,等价于私人储蓄的增加和赤字是一一对应的关系。因此李嘉图等价命题认为,如果政府通过赤字来提高既定支出所需的资金,私人储蓄的增加就刚好一对一地对应公共储蓄的减少,从而社会总储蓄并不改变。随着时间的推移,政府预算约束机制意味着,政府债务将增加,但是,政府债务增加并没有达到资本积累支出的增加。

在李嘉图等价命题下,长期赤字以及与之相关的政府债务的增加并未引起担忧。因为随着政府不断动用储蓄,人们会预测将来有更高的税收,于是人们会更多地储蓄。公共储蓄的减少被等额的私人储蓄增加相抵消,因此储蓄总额也就不受影响,进而投资也不会受影响。经济体即便没有出现债务增加的情况,现在的资本积累也不会不同。高债务是没有理由担心的。

但是只要将来的税收增加现得遥远,时间显得更加不确定,消费者就更可能忽略它们,出现这种情况是因为消费者预期到税收增加之前他就去世了,或者更可能的是消费者不能想得那么远。无论哪种情况李嘉图等价都可能会失效。

从短期来看,越大的赤字将可能导致越高的需求和产出。从长期来看,政府债务越高,资本积累和产出的降低越多。

\subsection{赤字、产出稳定和周期性赤字调整}

事实上预算赤字对资本积累继而对产出确实有长期的负面影响,这一事实意味着经济衰退时的赤字应该通过经济繁荣时的盈余来抵消,这样才不至于导致债务持续增加。

产出处于自然率水平的赤字:充分就业赤字(full-employment deficit)到中周期赤字(mid-cycle deficit)再到标准就业赤字(standardized employment deficit)以及结构性赤字(structural deficit),周期调整的赤字(cyclically adjusted deficit):

构造周期调整的赤字的步骤:

1. 确定如果产出高于诸如$ 1\% $时赤字会低多少;

产出对赤字的效应:自动稳定性(automatic stabilizer):经济衰退会自然导致赤字,因此,财政扩张会部分地抵消衰退。

2. 确定产出离自然率水平还有多远。

\subsection{战争和赤字}

政府依赖赤字为战争融资的正确性:

1. 分配性——赤字融资是将一部分债务负担转移给战后仍活着的那些人的一种方式,让下一代人分担战争必需的牺牲;

2. 经济性——赤字支出有助于减缓税收扭曲。

\hspace*{\fill}

假设经济是封闭的,所以$ Y=C+I+G $。假设G增加且Y保持不变。那么C+I就必须下降,如果税收没有增加,那么多数下降来自I的下降。如果税收增加,大多数的下降来自C的下降。

依赖赤字融资:政府支出G增加,在假设产出不变的情形下,为了保持均衡就必须增加足够多的利率。所以,取决于利率的投资将会急剧下降;

假设政府增加税收,消费将急剧下降(可支配收入下降)。政府支出的增加部分被消费的减少而抵消。利率比通过赤字融资情况下增加得少,因此投资也比赤字融资情况下减少得少。

如果政府对赤字依赖得越多,战争中消费的减少越少,投资的减少越多。较低的投资意味着战后较低的资本存量,因此战后得产出也更低。通过降低资本积累,赤字成了将战争负担部分地转移到下一代人身上地一种途径。

\subsection{减少税收扭曲}

如果政府通过增加税收来为支出增加而融资,那么税率将会非常高。非常高的税率将会导致非常高的扭曲。为了保持预算平衡,与其将税率调动,不如保持相对稳定的税率以及平滑税收。税收平稳化意味着当政府支出异常大时,会出现巨额的赤字,而在其余时间,只会出现很少的盈余。

\section{极高的债务危险}

政府债务很高时伴随两个成本——更低的资本积累、更大的税率和扭曲税收。还存在另一种成本:很高的债务将导致恶性循环,并使财政政策实施起来非常困难。

\subsection{极高债务、违约风险和恶性循环}

\[
\frac{B_t}{Y_t}-\frac{B_{t-1}}{Y_{t-1}}=(r-g)\frac{B_{t-1}}{Y_{t-1}}+\frac{G_t-T_t}{Y_t}
\]

假设金融投资者开始担忧政府未来可能无法全额偿还债券,他们要求更高的利率以弥补察觉到的更高债务违约风险。这将反过来使政府更难稳定债务。为了应对利率上升,政府采取措施使基础盈余增加,所需的削减支出或者增加税收可能会在政治上代价高昂,即产生更多的政治不确定性、更高的违约风险,进而进一步提高利率。因此急剧的财政紧缩很可能会导致衰退,降低增长率。实际利率上升和增长率下降都进一步增加$ (r-g) $,将需要更多的预算盈余来稳定债务。到某一时刻,政府可能无法充分增加基础盈余,债务比率开始上升,这令投资者更担心并要求更高的利率。利率的提高和债务比率的提高相互促进。(预期的自我实现)

简而言之,债务占GDP比率越高,灾难性债务动态变化的可能性越大。

\subsection{债务违约}

在某些时候,如果一国政府发现自己无力偿还债务,它可以违约。违约往往是局部的,采取债权人所谓的价值折扣(haircut)。违约的形式有债务重组(debt restructuring)或者债务延期(debt rescheduling),或者私营部门参与(private sector involvement)。

当债务很高时,违约似乎是一个吸引人的解决办法。在债务违约后,债务水平会缩小所需财政整合的规模,从而使其更可信。它降低了必要的税收,从而可能带来高增长。但违约也有高昂的成本。

\subsection{货币创造}

政府可以通过发行债券,然后迫使中央银行购买债券以换取资金。这一过程被称为货币创造(money finance)或债务货币化(debt monetization)。因为在这种情况下,货币创造的速度取决于政府赤字而非中央银行地位,所以也被称作货币政策的财政支配(fiscal dominance)地位。

记H为中央银行在经济中的资金数量金额。记$ \Delta H $为货币创造,即名义货币存量从一个月到下一个月的变化。政府通过创造一定数量的货币而产生实际收入等于$ \Delta H/P $。这种由货币创造带来的实际收入被称为铸币税(seignorage):

\[
seignorage=\frac{\Delta H}{P}
\]

铸币税等于名义货币发行除以价格水平,变形得:

\[
\frac{\Delta H}{P}=\frac{\Delta H}{H}\frac{H}{P}
\]

即:

\[
\frac{seignorage}{Y}=\frac{\Delta H}{H}\frac{H/P}{Y}
\]

随着货币增加,通货膨胀通常会随之到来。高通货膨胀导致人们减少对货币的需求,进而减少对中央银行货币的需求(实际货币余额H/P减少)。因此想要达到同样的收入水平,政府需要进一步提高货币增长率。然而较高的货币增长率将进一步导致通货膨胀,(H/P)/Y进一步下降,需要更高的货币增长。很快,高通货膨胀转变为恶行通货膨胀。








































	
\end{document}