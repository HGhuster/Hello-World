\documentclass{article}
\title{Blanchard Ch.23}
\author{Dawei Wang}
\date{\today}
\usepackage{ctex}
\usepackage{amsmath}
\usepackage{amssymb}
\usepackage{graphicx} %插入图片的宏包
\usepackage{float} %设置图片浮动位置的宏包
\usepackage{subfigure} %插入多图时用子图显示的宏包
\begin{document}
	\maketitle

中央银行通过货币供应量的变化来控制政策利率。当政策利率达到零,货币供应量的进一步增加对政策利率没有影响。

\hspace*{\fill}

货币政策对产出的短期影响:

利率下降导致支出增加,进而导致产出增加。

\hspace*{\fill}

实际利率等于名义利率减去预期通货膨胀。借款利率等于政策利率加上风险溢价。

对私人支出决策起重要作用的是实际借贷利率。

\hspace*{\fill}

货币政策的中期影响:

货币政策既不影响产出也不影响实际利率(中性利率);更高的货币增长只会导致更高的通货膨胀。

高失业率可能会导致通货紧缩,通货紧缩的下限为零,导致实际利率上升,从而进一步减少需求,增加企业。

\hspace*{\fill}

长期利率取决于对未来短期利率和长期溢价的预期,股票价格取决于预期的未来短期利率、未来股息和股票溢价。

股票价格也可能受到泡沫的影响,使价格不同于股票的基本价值。

\hspace*{\fill}

货币政策影响短期利率,但支出取决于当前和预期未来的短期实际利率;

货币政策对产出的影响主要取决于预期对货币政策的反应。

\hspace*{\fill}

在开放经济中,货币政策不仅通过利率而且通过汇率影响支出和产出,货币的增加导致利率下降和贬值,两者都增加支出和产出。在固定汇率下,中央银行放弃货币政策作为政策工具。

\hspace*{\fill}

在浮动汇率下,利率变动会导致汇率发生巨大变化;在固定汇率下,投机导致汇率危机和大幅贬值。

\hspace*{\fill}

政策效果的不确定性应导致更谨慎的政策。有理由限制决策者。

\section{从货币政策目标到通货膨胀目标}

货币政策的目标有两个:第一,保持低通胀和稳定;第二,稳定潜在产出,避免或至少限制衰退或繁荣。

20世纪80年代以前,中央银行的策略是选择目标货币增长率,并允许偏离目标比率。因为当时认为低目标货币增长率意味着较低的平均通货膨胀率。

\hspace*{\fill}

但是货币增长与通货膨胀之间并不存在紧密联系,即使在中期也是如此(必须排除零利率下限)。其次,短期货币供给量与利率之间的关系并不稳定。

中期货币增长与通货膨胀之间并非紧密关系以及短期利率与货币供给之间并非紧密联系的根源在于:货币需求的变化。

\[
\frac{M}{P}=YL(i)
\]

如果引入信用卡,实际的货币需求减半,那么$ \frac{M}{P}=\frac{1}{2}YL(i) $,在短期P不变,利率必须调整(下降)。在中期,P可以调整,给定产出水平与利率水平,$ \frac{M}{P} $必须减半。给定M这意味着P必须增加一倍。

20世纪90年代开始,由于基于货币增长目标转变成基于通货膨胀目标,使用利率准则。

\subsection{通货膨胀目标}

在中期致力于一个特定的通货膨胀目标几乎没有争议,但试图在短期内达到一个特定的通货膨胀目标似乎具有争议性。仅仅关注通货膨胀似乎会消除货币政策在减少产出波动中所发挥的作用。但是事实不是如此。

考虑菲利普斯曲线表达式:

\[
\pi_t=\pi_t^e-\alpha(u_t-u_n)
\]

设通货膨胀目标是$ \pi^* $,假设人们预期通货膨胀率与$ \pi^* $相等。

\[
\pi_t=\pi^*-\alpha(u_t-u_n)
\]

如果中央银行可以准确地实现目标通货膨胀率,即$ \pi_t=pi^* $,那么实际失业率将会始终等于自然失业率;这意味着产出也将始终处于自然产出水平。通过设定目标并实现与通货膨胀预期相一致的通货膨胀率,中央银行还将失业率保持在自然水平,从而使产出保持在潜在水平。

即使政策制定者不关注通货膨胀本身,而只关心产出,通货膨胀目标制仍然有意义。保持通货膨胀稳定是保持产出处在潜在水平的一种方法。这个结果被称为神圣的巧合(divine coincidence)。根据菲利普斯曲线,在保持通货膨胀恒定和和保持产出处于潜在水平之间没有冲突。因此,从短期和中期来看,注重维持稳定的通货膨胀率是正确的货币政策方法。

注:菲利普斯曲线远非一个精确的关系。有时通货膨胀率可能高于目标,产出可能低于潜在水平,中央银行需要在这两个目标间权衡。所有的中央银行都采用了所谓的弹性通货膨胀目标制(flexible inflation targeting)。

\subsection{利率规则}

中央银行不能直接控制通货膨胀,但可以控制政策利率。

中央银行控制政策利率的规则(泰勒规则,Taylor rule):

以$ \pi_t $表示通货膨胀率,以$ \pi^* $表示目标通货膨胀率;

以$ i_t $表示政策利率,即中央银行所控制的名义利率,$ i^* $表示目标名义利率,目标名义利率与中性利率($ r_n $)相关,目标通货膨胀率为$ \pi^* $,因此,$ i^*=r_n+\pi^* $;

以$ u_t $表示失业率,以$ u_n $表示自然失业率。

中央银行应该采用下面的规则:

\[
i_t=i^*+\alpha(\pi_t-\pi^*)-b(u_t-u_n)
\]

这里的系数a和b均是正的。

如果通货膨胀率正好等于目标通货膨胀率$ (\pi_t=\pi^*) $,失业率也等于自然失业率$ (u_t=u_n) $,那么,中央银行就应该设定名义利率等于目标值$ i_t=i^* $。这样经济可以处于同一轨道:通货膨胀率等于目标值,失业也等于自然失业率。

如果通货膨胀率高于目标值$ \pi_t>\pi^* $,中央银行应该将名义利率增加到目标值$ i^* $之上。这一更高的利率将使失业增加,失业的增加又将导致通货膨胀降低。

系数$\alpha$反映与通货膨胀相比中央银行重视失业的程度。系数$\alpha$越大,中央银行应该针对通货膨胀越多地提高利率,经济下降幅度越大,通货膨胀回落到目标的速度越快。泰勒指出,在任何一种情况下,系数$\alpha$的值都应该大于1。因为对支出水平真正重要的是实际利率而不是名义利率。当通货膨胀增加时,中央银行如果想要减少支出和产出,就必须提高实际利率。换句话说,它提高名义利率的程度必须比通货膨胀增加的程度更大。

如果失业率高于自然失业率$ u>u_n $,中央银行应该降低名义利率。较低的名义利率可以增加产出进而降低失业率。系数b代表与重视通货膨胀相比中央银行重视失业的程度:系数b越大,中央银行越愿意通货膨胀偏离其目标而使失业保持在接近自然失业率的水平。

\section{最优通货膨胀率}

\subsection{通货膨胀的成本}

通货膨胀率高达30\%甚至更高时,经济活动将被彻底扰乱。

现讨论是零通货膨胀率给经济带来的好处多还是4\%的年通货膨胀率带来的好处多一点。

\hspace*{\fill}

皮鞋成本

中期而言,通货膨胀率越高,名义利率越高,从而持币的机会成本越高。结果人们为了减少持有货币的余额不得不更频繁地去银行,即将这种成本表示为皮鞋成本(shoe-leather costs)。如果通货膨胀率维持在低水平,人们跑银行地时间就省下来了,从而做点别的事情。

在恶性通货膨胀时期,皮鞋成本可能非常大。在温和通胀水平下,皮鞋成本的重要性就有限。

\hspace*{\fill}

税收扭曲

通货膨胀的第二个成本源于税收系统和通货膨胀之间的相互作用。

尽管一项资产的实际回报率是用实际利率而不是名义利率来表示的,但是应税收入是基于名义利息计算,而不是基于实际利率计算的。结果随着时间增加的名义收入将人们从低税率等级推向更高的税率等级,这种效应叫做税级攀升(bracket creep)。(随着税收收入的持续增加,政府控制支出的压力就小很多。)

如果政府将价格指数化,就可以消除这一影响,但是税法没有这样操作,故导致了税收扭曲。

\hspace*{\fill}

货币幻觉(money illusion)

人们在评估名义和实际变化时犯了系统性的错误。价格稳定时的计算很简单,在出现通货膨胀时计算变得很复杂。人们常常不能区分名义利率和实际利率,因此通货膨胀使企业做出错误的决策。

\hspace*{\fill}

通货膨胀的可变性

较高的通货膨胀水平意通常伴随着较高通货膨胀的可变性,而且通货膨胀的可变性越大,就意味着诸如债券这样的固定收益金融资产的风险变得越大。

这些成本实际上不是由通货膨胀本身造成的,而是因为金融市场不能提供那些可以使持有人抵消通货膨胀的资产。

\subsection{通货膨胀的好处}

铸币税

货币创造,即通货膨胀的最终根源,是一种政府为其支出进行融资的渠道。从另一个角度理解,货币创造是向公众举债和征收税收之外政府获取资金的一种替代方法。

政府通常并不通过“创造”货币来支付政府的支出,而是通过发行和销售债券来取得相应收入进行支付。但是如果这些政府债券被中央银行购买了,这样政府就是通过创造货币来应付支出。结果是一样的,从货币创造中获得收入。

\hspace*{\fill}

重返货币幻觉

货币幻觉至少提供了一个支持通货膨胀率为正的观点。

通货膨胀的存在使实际工资的下行调整要比不存在通货膨胀时的下行调整容易些。

\hspace*{\fill}

负实际利率的选择

平均通货膨胀率较高的经济体在利用货币政策对应经济衰退时拥有更多的余地,而平均通货膨胀率较低的国家可能会无法使用货币政策使产出回到自然水平。

\subsection{最优通货膨胀率:目前的争论}

当前,大多数发达国家的中央银行均采取2\%的目标通货膨胀制。然而一部分经济学家想要实现价格稳定,即零通货膨胀率;另一些想要更高的通货膨胀率,例如4\%。

追求零通货膨胀率的人认为零通货膨胀率可以实现价格稳定。由于中央银行面临时间不一致问题,目标通货膨胀率的可信度和单一性就显得很重要,一些人认为价格稳定的目标在零通货膨胀水平下比在2\%的目标通货膨胀率下更容易达到。

希望提高通货膨胀率的人认为,今后不要陷入流动性陷阱至关重要。因此提高通货膨胀率目标将会有所帮助。


\section{非常规货币政策}

当利率达到零利率下限,即危机开始时,中央银行无法进一步降低利率,因此失去了使用常规货币政策(conventional monetary policy)的机会。

非常规货币政策:

购买除了短期债券以外的资产,目的是降低这些资产的溢价,从而降低相应的借贷利率,以刺激经济活动。它们通过货币创造为购买提供资金,导致货币供应量大幅增加。虽然货币供应量的增加对政策利率没有影响,但其购买其他资产降低了溢价,导致借贷利率降低,支出增加。这些购买计划被称为量化宽松(quantitative easing)或信用宽松(credit easing)政策。

\section{货币政策和财政稳定}

\subsection{流动性提供和最终贷款人}

限制挤兑的措施:

1. 存款保险,这使投资者有信心,即使银行破产,他们也可以得到自己的资金,因而没有动力挤兑;

2. 在实际挤兑发生时,中央银行向银行提供流动性,同时银行抵付一些抵押品,即银行的一些资产。中央银行的这一职能被称为最终贷款人(lender of the last resort)。

\subsection{宏观审慎工具}

面对危机:

等待是危险的。面对过度的银行杠杆,更好的办法是防止高杠杆,即使冒着银行信贷减少的风险,也不应该让它积累起来,增加金融危机的风险;

为了应对金融体系的泡沫、信贷繁荣或危险行为,利率不是恰当的政策工具。正确的工具是宏观审慎工具(macroprudential tools),是直接针对借款人、放款人、银行和其他金融机构的规则视情况需要而定。

\hspace*{\fill}

宏观审慎工具形式:

针对借款人:

房贷抵押,贷款价值比(loan-to-value ratio),贷款规模相当于房价的上限。

针对放款人:

巴塞尔协议,规定最低资本比率,限制杠杆;

资本控制来限制资本流动。

\hspace*{\fill}

宏观审慎工具的问题:

1. 许多情况下不知道这些工具的效果如何;

2. 传统货币工具与宏观货币工具之间可能存在复杂的互动关系;

宏观审慎工具应与传统货币政策工具一起至于中央银行的控制之下还是置于另一个机构的控制下。

























\end{document}