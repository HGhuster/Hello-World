\documentclass{article}
\title{Blanchard Ch.21}
\author{Dawei Wang}
\date{\today}
\usepackage{ctex}
\usepackage{amsmath}
\usepackage{amssymb}
\usepackage{graphicx} %插入图片的宏包
\usepackage{float} %设置图片浮动位置的宏包
\usepackage{subfigure} %插入多图时用子图显示的宏包
\begin{document}
	\maketitle
\section{不确定性和政策}

宏观经济政策效应确实存在不确定性,这种不确定性应使政策制定者更加谨慎并较少使用积极的政策。政策的主要目标应该是避免经济长期衰退,减缓过热经济,消除通胀压力。只有失业率和通货膨胀率很高时才采取积极的政策。

\section{预期和政策}

宏观经济政策效果不确定性的原因之一是预期与政策的相互作用。一个政策如何起作用以及起多大作用不仅取决于它如何影响目前的变量,还取决于它如何影响对未来的预期,然而,对政策预期的重要性已经超过了政策效果的不确定性。






	
	
\end{document}