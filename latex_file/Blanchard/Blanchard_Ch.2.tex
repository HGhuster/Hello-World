\documentclass{article}
\title{Blanchard Ch.2}
\author{Dawei Wang}
\date{\today}
\usepackage{ctex}
\usepackage{amsmath}
\usepackage{amssymb}
\begin{document}
	\maketitle
\section{总产出}
\subsection{GDP:产出和收入}

在国民收入账户中,对总产出(aggregate output)的度量叫作GDP

GDP(gross domestic product)的定义:

\hspace*{\fill}

从产出的角度:

1. GDP是经济中一定时期内所产生的最终产品和劳务的价值之和。(注意区分中间品和最终品。)

中间品是用于生产其他产品的产品,有些东西既可以是中间产品,也可以是最终产品。


2. GDP是一个经济体内及一定时期内增加值的加总。

一个企业在生产过程中的增加值(value added)被定义为所有产出的价值减去生产过程中所使用的中间产品的价值。

\hspace*{\fill}

从收入的角度:

3. GDP是在一个经济体内及一定时期内的收入总和。

\hspace*{\fill}

注:1. GDP是对总产出的衡量,可以从产出的一方也可以从收入一方考虑(总收入);2. 总产出和总收入是相等的。

\subsection{名义GDP和实际GDP}
名义GDP(Nominal GDP)是所生产的最终产品的产量乘以各自当期价格后的总和。

实际GDP(Real GDP)所生产的最终产品数量乘以一个恒定的(而非当期)价格加总。

由于最终产品不止一种,必须将实际GDP界定为所有最终产品产量的加权平均,各个产品的权数因此对应其相对价格。但相对价格是会随时间改变的,若选择给定年份的相对价格作为权数则会得到一个静态的权数集。这里就涉及一个取舍问题:是选择静态的权数集,还是让权数集随时间变化。

\hspace*{\fill}

同义词:

名义GDP=美元GDP(dollar GDP)=以当期美元计算的GDP(GDP in current dollars)

实际GDP=实物表示的GDP(GDP in terms of goods)、以不变价格表示的GDP(GDP in constant dollars)、通货膨胀调整的GDP(GDP adjusted for inflation)

如非特别指出,GDP均指实际GDP,用$ Y_t $表示第t年的实际GDP。

\subsection{GDP:水平与增长率}
人均实际GDP:一国实际GDP与人口数的比值。这一数值代表了一国的平均生活水平。

当评估一年中的经济发展状况时,经济学家们关注的是实际GDP的增长率,简称GDP增长率,GDP增长率为正时被称为扩张期;GDP为负时被称为衰退期。(按照惯例,经济学家通常将至少持续两个季度的经济负增长称为衰退。)

\section{失业率}
就业(employment):指拥有工作的人口的数量。

失业(unemployment):指没有工作但是正在寻找工作的人口数量。

劳动力(labor force):就业人数和失业人数的总和。
\[
L=N+U
\]
失业率:
\[
u=\frac{U}{L}
\]
根据定义,一个人要被确定为失业,必须满足两个条件:第一,没有工作;第二,正在找工作。

只有没有工作但在寻找工作的人才被计入失业者;那些没有工作但不去寻找工作的人被计入非劳动力范畴。当失业率很高时,一些没有工作的人放弃了寻找工作,因而不被计入失业者中,这些人被称为丧失信心的工人。

高的失业率一般与低的劳动力参工率相联系。劳动力参工率的定义为:劳动力人数与属于劳动年龄人口总数的比率


\subsection{为什么宏观经济学家关注失业}
1. 失业对失业者的福利有直接影响。

2. 失业率是经济是否有效使用资源的信号。许多有工作意愿的人找不到工作,这就意味着经济并没有有效地利用人力资源。从这一观点看,失业率过低说明经济正在过度使用资源,之后可能遭遇劳动力短缺。

\section{通货膨胀}

通货膨胀(inflation)价格总体水平持续上涨的现象。通货紧缩(deflation)物价水平持续下降的现象。

通货膨胀率(inflation rate)是指物价水平上升的速度。

物价水平有两种定义方法(两种物价指数):
\subsection{GDP平减指数(GDP Deflator)}
第t年的GDP平减指数$ P_t $被定义为第t年的名义GDP和实际GDP的比率。
\[
P_t=nominal\enspace GDP_t/real\enspace GDP_t=\$Y_t/Y_t
\]

GDP平减指数的变化率:$ \pi_t=(P_t-P_{t-1})/P_{t-1} $,可以用来刻画通货膨胀率。

名义GDP增长率=实际GDP增长率+通货膨胀率

\subsection{消费者物价指数(Consumer Price Index)}
经济中所生产的最终物品的集合不等于消费者所购买的物品的集合,这其中有两个原因:

1. GDP中的某些物品不卖给消费者,而是卖给企业、政府或外国人。

2. 消费者所购买的某些物品,不是本国生产的而是从外国进口的。

\hspace*{\fill}

消费者物价指数(consumer price index, CPI)度量消费的平均价格。衡量平均消费价格或生活成本。

CPI描述了一定时期特定篮子里的商品和服务以美元计价的消费价格。这一篮子是根据消费者支出的详细研究结果制定的,目的是描述一个典型的城镇居民的消费组合。

\hspace*{\fill}

GDP描述的是某国所生产的产品的价格,CPI是描述该国所消费的物品的价格。

\subsection{为什么宏观经济学家关注通货膨胀}
纯粹的通货膨胀:所有价格和工资按比例增长,此时相对价格不受通胀影响。

现实中不存在纯粹的通货膨胀:

1. 通胀时期,并非所有价格和工资都是按工资上涨的,因此通胀会影响收入分配。

2. 通胀导致的相对价格扭曲产生了很多不确定性,使企业的决策变得越来越困难。如果名义收入上升而税级没适时调整,则需缴税更多。

\hspace*{\fill}

通缩也会产生相对价格扭曲和不确定性。通缩是指价格水平的下降,衰退是指实际产出的下降。

通货紧缩和衰退虽然会同时出现,但实际含义不同。通货紧缩是指价格水平的下降,而衰退是指实际产出的减少。


\section{产出、失业与通货膨胀率:奥肯定律与菲利普斯曲线}
\subsection{奥肯定律}
当产出增长率较高时,失业会下降。 

\subsection{菲利普斯曲线}
奥肯定律表明,如果有足够强劲的增长,我们可以将失业率降低到非常低的水平。但是当失业变得非常低的时候,经济就很可能过热,这会对通胀造成上升的压力。

菲利普斯曲线定义为通胀率的变化和失业率之间的关系:

平均意义上来看,较高的失业会导致通胀率的下降,较低的失业率会导致通胀率上升。

\section{短期、中期和长期}
在短期,例如几年内,产出年复一年的变化主要受需求的变化的影响。

在中期,也就是一二十年内,经济趋向于回到由供给因素决定的产出水平。供给因素即资本存量、技术水平和劳动力规模。

在长期,也就是几十年或更长时间内,必须考虑诸如教育体系、储蓄率、政府作用之类的因素。

\hspace*{\fill}

特征价格法(Hedonic price method)

商品提供了一些集中的性能,每种性能都有一个隐含的价格,根据性能的价格定价被称为特征定价。


\section{实际GDP的构造和环比指数}



\end{document}