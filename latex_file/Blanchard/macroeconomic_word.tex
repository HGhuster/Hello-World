\documentclass{article}
\title{macroeconomic term}
\author{Dawei Wang}
\date{\today}
\usepackage{ctex}
\usepackage{amsmath}
\usepackage{amssymb}
\usepackage{graphicx} %插入图片的宏包
\usepackage{float} %设置图片浮动位置的宏包
\usepackage{subfigure} %插入多图时用子图显示的宏包
\begin{document}
	\maketitle
\section{Chapter 2}
GDP三个定义:本国要素的国内产出+外国要素的国内产出

一定时间内经济体内生产的最终产品和服务的价值的总和;

一定时间内,一个经济体中生产的增加值的总和;

一定时间内,一个经济体中收入的总和。

GNP:本国要素的国内产出+本国要素的国外产出

GNP=GDP+NI

\hspace*{\fill}

名义GDP和实际GDP:

名义GDP:以经济体中生产的最终产品和劳务的当期价格乘以各自的数量并加总得到的GDP;

实际GDP:以某一组固定价格(基期价格)乘以经济体中生产的最终产品和劳务并加总得到的GDP。

\hspace*{\fill}

失业率:失业人数/劳动力人数

\hspace*{\fill}

劳动力参与率:劳动力人数/劳动力年龄人口数

\hspace*{\fill}

CPI,GDP和CPI的区别:

CPI:一定时期内特定篮子里的商品和服务的消费价格的加权平均。

GDP和CPI区别:消费者会购买进口商品;GDP中某些物品不是卖给消费者而是卖给企业/政府/出口。

\hspace*{\fill}

通货膨胀率的计算,包括GDP平减指数、CPI两种计算方式

\[
\pi_t=\frac{P_t-P_{t-1}}{P_{t-1}}
\]

\hspace*{\fill}

为什么经济学家关注通胀?

1.纯粹的通胀(所有的工资和价格按比例上涨)并不存在。

2.通胀的成本:

皮鞋成本:人们减少持有货币余额而频繁去银行;

税收扭曲:应税收入以名义利息计,名义收入增加,税级攀升;

货币幻觉:人们不能区分名义利率和实际利率,通胀放大了这种扭曲;

通胀的可变性:通胀的可变性使固收的风险变大。

3.通胀的好处:

铸币税:

货币幻觉:名义工资向下刚性,通胀降低实际工资;

负利率空间:$ r=i-\pi $

\hspace*{\fill}

奥肯定律:

失业率和经济增长率负相关:经济增长率高的时候失业率低。

\[
u-u_t=-\beta(g_y-g_y^0)
\]

\section{Chapter 3}

GDP里的消费、投资、可支配收入、政府支出、税收的定义(涵盖了什么范围,什么纳入,什么不纳入):

消费:购买商品和服务的支出,T=税收-转移支付

\[
C=c_0+c_1Y_D=c_0+c_1(Y-T)
\]

投资:固定投资=住宅投资(住房、公寓)+非住宅投资(机器厂房)

\[
I=I(Y,i)
\]

政府支出:政府对物品或劳务的购买,不包括转移支付、政府债务利息

\hspace*{\fill}

内生变量:可以由模型解释的变量;

外生变量:不可以由模型解释的变量,给定的变量;

乘数:某一数值的增加使得产出成倍数地增加;

\hspace*{\fill}

自主支出:

\[
c_0-c_1T+I+G
\]

\hspace*{\fill}

储蓄悖论:

\[
I=S+T-G
\]

I、T、G不变,即使人们想要在收入一定时增加储蓄,但也会因为收入减少而储蓄没有变化。(仅在短期适用)。

\hspace*{\fill}

比例税是自动稳定器

\section{Chapter 4}

公开市场操作:政府在公开市场购买或者出售债券,购买债券则为扩张性公开市场操作;出售债券则为紧缩型公开市场操作。

\hspace*{\fill}

流动性陷阱:零名义利率下限,当名义利率降为0时,再进行货币增发只会让个人愿意持有所有的货币(货币和债券是无差异的),银行愿意以保证金地形式持有所有增发货币,货币政策不能进一步降低名义利率。

\hspace*{\fill}

货币收入和财富;

货币:能用于交易,=可开支票存款+通货

收入:流量,=工资+利息+红利

(金融)财富:存量,=金融资产-金融负债

储蓄:流量,收入中没花掉的那一部分。

\hspace*{\fill}

高能货币/基础货币:=银行准备金+通货

\hspace*{\fill}

货币乘数:$ \frac{1}{c+(1-c)\theta} $

\section{Chapter 6}

名义利率:以国家货币形式表达的利率

实际利率:以一揽子商品形式表达的利率

\hspace*{\fill}

风险溢价:由于债券存在违约的可能性,投资者要求风险溢价作为对风险的补偿。风险溢价由违约的可能性和投资者的风险厌恶程度决定,二者与风险溢价正相关。

\hspace*{\fill}

杠杆率:银行资产/银行资本

资本比率:银行资本/银行资产

\hspace*{\fill}

风险厌恶:风险厌恶程度越高,风险溢价越高。

\hspace*{\fill}

政策利率和借款利率的区别:

借款利率=政策利率+风险溢价

政策利率进入LM关系,受央行控制;借款利率进入IS关系。

\hspace*{\fill}

影子银行:常规银行体系外的各种金融中介业务,通常由非银金融机构为载体,对金融资产信用、流动性、期限等因素进行转换,扮演着“类银行”的角色。

\section{Chapter 7}

丧失信心的劳动者:退出劳动力市场的一部分人,虽然没有积极找工作,但找到工作也会接受。

\hspace*{\fill}

保留工资:使人们感觉失业和工作无差异的工资水平。

\hspace*{\fill}

议价能力:与雇主进行工资谈判的能力,议价能力与劳动力市场状况和工作性质相关。失业率越低,工作不可替代性越强,议价能力越强。

\hspace*{\fill}

效率工资理论:将工人生产率或效率联系与工资联系起来的理论称为效率工资理论。

效率工资理论说明工资取决于工作的性质和劳动力市场状况。
高科技企业为了提升员工士气愿意支付高于保留工资水平的工资;失业率低时企业为了减少员工离职必须支付高工资。

\hspace*{\fill}

自然失业率/结构性失业率:劳动力市场均衡且预期价格水平等于实际价格水平时的失业率为自然失业率(使通货膨胀保持不变的失业率)。

\section{Chapter 8}

无附加通胀的失业率:自然失业率是使通货膨胀保持不变时的失业率,因此也被叫做无附加通胀的失业率。

\hspace*{\fill}

修正的菲利普斯曲线/附加预期的菲利普斯曲线/加速的菲利普斯曲线

\[
\pi_t-\pi_{t-1}=-\alpha(u_t-u_n)
\]

附加预期的通货膨胀曲线说明$ \pi_{t-1} $表示预期通货膨胀;

加速的菲利普斯曲线说明低失业率导致通货膨胀率的增加,由此加速了价格水平的上升。

\section{Chapter 9}

劳动力囤积:

由于公司培训新员工也是有成本的,因此再再经济不景气(失业率高)的时候,企业也宁愿留住现有员工,当经济好转的时候,它们利用这些员工,企业的这种行为叫作劳动力囤积。

\hspace*{\fill}

自然利率/中性利率/维克塞尔利率:

使经济产出位于潜在产出水平的利率被称为中性利率。中性利率同时可以保证通货膨胀不变。

\hspace*{\fill}

通货紧缩螺旋/陷阱:

\hspace*{\fill}

滞涨:较低的经济产出水平和较高的通货膨胀率同时发生时的经济状况。可能由于经济冲击(例如石油价格上涨)使得自然产出水平大幅下降引起。

\hspace*{\fill}

产出波动/商业周期:

\section{Chapter 10}

绝对购买力平价:

一国货币的价值及对它的需求是由单位货币在国内所能买到的商品和劳务的量决定的,即由它的购买力决定的,因此两国货币之间的汇率可以表示为两国货币的购买力之比。而购买力的大小是通过物价水平体现出来的。根据这一关系式,本国物价上涨将意味着本国货币相对外国货币的贬值。

相对购买力平价:

相对购买力平价弥补了绝对购买力平价一些不足的方面。它的主要观点可以简单地表述为:两国货币的汇率水平将根据两国通胀率的差异而进行相应地调整。它表明两国间的相对通货膨胀决定两种货币间的均衡汇率。

\hspace*{\fill}

绝对购买力平价:货币的价值是由购买力决定的,世界上的商品遵循一价定律$ E=\frac{P^*}{P} $。货物在各地的价格一样,汇率由两个国家的物价水平决定。物价水平表示货币的购买力。此时实际汇率为1,$ \epsilon=EP/P^*=1 $

相对购买力平价:汇率的变动由两个国家的通胀率决定的,没有一价定律的前提。承认有交易费用。满足决定购买力平价必定满足相对购买力评价,反之不然

\hspace*{\fill}

伊斯特林悖论:
通常在一个国家内,富人报告的平均幸福和快乐水平高于穷人,但如果进行跨国比较,富国的平均收入的提高带来的国家幸福感的提高并没有穷国明显,当整个国家所有人都能满足基本物质需求,再去通过收入很难衡量幸福感变化。$ Y=k^\alpha(AN)^{1-\alpha} $

\hspace*{\fill}

马尔萨斯陷阱:

人口增长是按照几何级数增长的,而生存资源仅仅是按照算术级数增长的,多增加的人口总是要以某种方式被消灭掉,人口不能超出相应的农业发展水平。这个理论就被人称为“马尔萨斯陷阱”。

\hspace*{\fill}

规模报酬不变、资本/劳动力的规模报酬递减:

按相同比例增加所有投入,总产出按同一比例增加,具有这种特性的总产量函数被称为规模报酬不变;

按一定比例增加资本(劳动),总产出的增加比例低于增加的资本(劳动)的比例,具有这种特性的总产量函数被称为资本(劳动)的规模报酬递减。

\section{Chapter 11}

产出和资本的相互作用

资本数量决定产出数量;产出数量决定储蓄数量,由此决定累积的资本数量。

\hspace*{\fill}

经济达到稳态

当人均资本(单位有效工人资本)和人均产出(单位有效工人产出)不再变化时,我们称之为经济达到了稳态。

\hspace*{\fill}

黄金法则下的资本存量/黄金律水平

在稳态经济中,消费水平达到最高时的由储蓄率决定的资本存量称之为黄金法则下的资本存量。

\hspace*{\fill}

内生增长模型

没有技术进步而保持经济稳态增长的模型,我们称之为内生增长模型。

\hspace*{\fill}

人力资本靠什么积累?人力资本实物资本有何不同?

人力资本通过教育和培训积累。
实物资本通过实物投资积累,实物资本折旧率高于人力资本折旧率,人力资本用得越多退化得越慢。

\section{Chapter 12}

平衡增长/均衡增长的特点/特征

平衡增长下,产出、资本、有效劳动的增长速度都为$ g_A+g_N $

\hspace*{\fill}

技术进步的决定因素(研究的多产性/专属性/制度)

多产性:研发支出如何转化为新产品和新概念;

专属性:企业能在多大程度上从新产品中赚取利润;

产权的保护很重要,专利权赋予了研制出新产品的企业在特定时间内不允许其他企业使用该新产品的权利。

\hspace*{\fill}

全要素增长率/索洛剩余

索洛剩余=产出增长率-劳动力份额$ \times $劳动增长率+资本份额$ \times $资本增长率=劳动力份额$ \times $技术进步率

\section{Chapter 13}

技术性失业

指因技术进步,达成相同产能的劳动力需求减少,所引发的失业现象。

\hspace*{\fill}

工资不平等加剧的原因

国际贸易:

企业为了保持竞争实力,将生产转移到低成本国家,如此一来,对技能工人需求减少;

技能型进步:

新机器和新生产方法要求更多高技能工人。新生产方法要求工人更加灵活,更好地适应新任务,灵活性则要求更多技能和教育。

\section{Chapter 14}

现值,折现因子,折现率

现值:一系列未来收益在今天的期望值;

折现因子、折现率:名义利率为$ i_t $,则折现因子为$ 1/(1+i_t) $,$ i_t $被称为折现率。

\hspace*{\fill}

收益率曲线(搞清楚这个曲线到底是干啥的,横纵坐标是啥,不要只是看个词)

某一个时间点看到的不同期限债券在某一个时间点的收益率的组成的线,横坐标是期限,纵坐标是收益率

\hspace*{\fill}

内部融资、外部融资、债务融资、股权融资

内部融资:用一部分利润来融资;

外部融资:银行贷款;

债务融资:债券和贷款;

股权融资:发行股票

\hspace*{\fill}

除息价格:刚刚支付过红利的股票的价格。

\hspace*{\fill}

随机游走:

一些事物,如果向上和向下的每一步的变动的概率相等,即服从随机游走。服从随机游走的变动是不可预测的。

\hspace*{\fill}

消费支出提高对股票市场影响

消费支出的提高带来更高的产出水平,股票市场的反应要看投资者预期央行的行动;

\hspace*{\fill}

扩张性货币政策对股票市场影响

货币扩张降低了利率并增加了产出,这对股票市场的影响取决于金融市场是否能预期到货币扩张。

\hspace*{\fill}

理性投机泡沫

只要投资者预期股价会上涨,股价就会上涨。这种股票价格呈现的变动称为理性投机泡沫。

\hspace*{\fill}

预期假说

\section{Chapter 15}

消费的持久收入理论,消费的生命周期理论

“持久收入”强调消费者不仅仅考虑当前收入。

生命周期强调消费者自然的消费计划计划范围是其一生。

\hspace*{\fill}

总财富,人力财富,非人力财富

总财富=人力财富+非人力财富

人力财富:估计工作年限内获得的税后收入,并计算其预期贴现值。

非人力财富:金融财富+房产财富

\hspace*{\fill}

机器的租赁成本/使用者成本

实际利率和折旧率之和

\hspace*{\fill}

消费和投资的波动性对比

投资波动比消费波动更大,二者通常同步变动,投资波动范围比消费更大,投资和消费对产出波动的贡献大致相同。

\hspace*{\fill}

静态预期

预期为一常量

\hspace*{\fill}

适应性预期

随已发生情况变动

\hspace*{\fill}

理性预期

经济主体的预期是合乎理性的(最大化利用有效信息)

\section{Chapter 17}

开放的三个方面:产品、金融、要素

产品市场:

消费者和公司在本国和外国市场之间进行选择的能力

金融市场开放:

金融投资者在本国和外国资产之间进行选择的能力

要素市场的开放:

公司选择生产地点和工人选择工作地点的能力

\hspace*{\fill}

更好的开放程度指标

比进出口的比率更好的开放程度指标是:可贸易产品占总产出的比率。

可贸易产品是指在本国市场或在外国市场与外国产品进行竞争的产品。

\hspace*{\fill}

实际汇率和名义汇率

名义汇率,用外国货币表示的本币价格

实际汇率,用外国产品表示的本国产品的价格

\hspace*{\fill}

经常账户/资本账户,具体包括的内容

经常账户=NX+NI+TR

NX=贸易余额=出口-进口

NI=净投资收益(投资的收入,与本金分开)=本国居民从外国资产中获得的收入-外国居民从本国资产中获得的收入。

TR=一个国家获得的净转移支付收入

资本账户=净外国债务=外国持有的本国资产增量-本国持有的外国资产增量=净资本流动


\hspace*{\fill}

利率平价条件/无抛补利率平价关系

\[
1+i_t=(1+i_t^*)\frac{E_t}{E^e_{t+1}}
\]

\section{Chapter 18}

马歇尔勒纳条件:

实际贬值带来净出口提高的条件

\hspace*{\fill}

J曲线

在贬值开始的一段时间,贬值的影响更多地体现在价格上而非数量上。由于进出口数量调整缓慢。因此,在贬值的初始阶段将导致贸易余额出现恶化。实际汇率下降,但X和IM不变,贸易余额恶化。

\section{Chapter 19}

不可能三角/三元悖论

指一个国家不可能同时完成下列三者:

完全资本流动/独立的货币政策/固定汇率

在资本自由进出及固定汇率之下,政府无法固定货币供给,也不能固定利率,于是资本的流动将使利率趋向于国际利率水准,失去货币政策自主性。

在资本自由进出且自主的货币政策之下,利率变化造成的资本流动会被汇率的反向变动所抵销,使资本的进出不影响货币供给,因此国家可以拥有货币政策自主性,却不能固定汇率。

在上面两个情况皆为资本流动造成双率只能择一控管,于是在管制资本流动后,便可控制双率。

\section{Chapter 20}

汇率制度的选择:固定和浮动汇率利弊

固定汇率制:放弃利率工具和汇率工具。不仅降低经济冲击的反应能力,而且可能造成汇率危机;

浮动汇率制度:汇率可能大幅波动并且很难通过货币政策控制。

一般使用浮动汇率制度更好,除了:

当一组国家经济紧密结合时,共同货币或许更好;

人们不相信中央银行会在浮动汇率制度下遵循一个负责人的货币政策的时候,固定汇率制度可能更好。

\section{Chapter 22}

1. 赤字的官方度量/通货膨胀调整后的赤字

通胀调整后的赤字t=$ rB_t+G_t-T_t $

赤字的官方度量t=$ i B_t+G_t-T_t $

\hspace*{\fill}

2. 李嘉图等价

一旦考虑了政府预算约束的影响,无论是赤字还是债务都不会对经济活动产生影响。

假设政府预算初始是平衡的。政府实行减税以图增加私人部门的公众的支出,扩大总需求,但减税引起财政赤字。如果政府通过发债来弥补财政赤字,则会在未来增加税收,以偿还债务和利息。具有前瞻性的消费者知道,政府今天的减税意味着未来更高的税收,政府通过债务融资实现的减税并没有减少税收负担,仅仅是重新安排税收的时间。因此,这种政策不会影响消费者的支出。该理论认为消费者根据持久收入决定当前消费,政府当前的减税意味着未来的增税,减税不影响持久收入,因此减税不影响人们的消费、储蓄、投资。

\hspace*{\fill}

3. 周期调整赤字

产出处于自然产出的赤字水平

\section{Chapter 23}

1. 泰勒规则/利率规则

\[
i_t=i^*+a(\pi_t-\pi^*)-b(u_t-u_n)
\]
a>1,利率提高程度应大于通胀程度,实际利率才能提高。

\hspace*{\fill}

2. 通货膨胀的成本

皮鞋成本:中期内通胀越高,名义利率越高,持有货币的机会成本越高,为了减少货币持有量,人们更频繁地到银行存取款而造成的时间和其他损失。

菜单成本:企业频繁更换菜目表而产生的成本

税收扭曲:对于资本而言,通胀使资本名义价格上升,资本收入税增加,但资本的真实价值不变。对于工资而言,存在税级攀升效应,通胀使名义工资更高,——不是实际工资,提高了纳税人的税率等级。

货币幻觉:人们在判断名义变化和实际变化会犯系统性错误,很难准确判断是名义量的变化还是实际量的变化。如,难以区分实际利率和名义利率变化,导致企业作出错误决策。

通胀的波动性:通胀越高,通胀的波动性越大。对于支付固定金额的金融资产而言,这意味着风险越高。

3. 通货膨胀的好处

\hspace*{\fill}

铸币税:由创造货币带来的收入。获得铸币税收入通常是通胀率较高国家进行融资的重要渠道。

负实际利率的选择:实际利率=名义利率-预期通胀率
在中期,实际通胀=预期通胀。
负的实际利率可以刺激需求,提高产出。负实际利率的实现要求名义利率小于通胀,名义利率最低为0,因此高通胀的经济体拥有更广阔的空间运用货币政策来应对衰退。

货币幻觉使实际工资的调整更容易。
名义工资较难下调(会受到工会等势力的阻挠),通胀能够在不改变名义工资的条件下降低实际工资,由于货币幻觉的存在,由通胀引起的实际工资下调更易于被接受。


费雪效应:中期名义利率和通胀等量提高。

托宾Q值=股票市值/资产重置成本

租赁成本:实际利率+折旧率

\end{document}