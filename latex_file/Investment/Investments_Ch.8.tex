\documentclass{article}
\title{指数模型}
\author{Dawei Wang}
\date{\today}
\usepackage{ctex}
\usepackage{amsmath}
\usepackage{amssymb}
\begin{document}
	\maketitle
\section{单因素证券市场}
\subsection{马科维茨模型的输入数据}
需要数据:

n个期望收益估计值,n个收益的方差,$ \frac{n(n-1)}{2} $个协方差。

总共$ \frac{n^2+3n}{2} $个数据。(注:相关系数矩阵一定是正定的。)

引用一个简化描述证券风险来源方式的模型可以用更少且具有一致性的风险参数和风险溢价估值。因为受共同经济因素影响,证券间协方差为正。通过将不确定性分解为系统性和公司层面的来源,可以大大简化协方差和相关系数的估计。
  
\subsection{收益的正态分布和系统性风险}

将证券i的收益分解为期望收益率和非期望部分之和。
\[
r_i=E(r_i)+e_i
\]
$ e_i $的均值为0,标准差为$ \sigma_i $。

单因素模型:
\[
r_i=E(r_i)+\beta_im+e_i
\]

其中m为影响所有公司的宏观经济因素,其均值为0,方差为$ \sigma_m $,m和$ e_i $不相关,$ \beta_i $为公司i对因素m的敏感程度,因此:
\[
\sigma_i^2=\beta_i^2\sigma_m^2+\sigma^2(e_i)
\]
任意两证券之间的协方差:
\[
Cov(r_i,r_j)=Cov(\beta_im+e_i,\beta_jm+e_j)=\beta_i\beta_j\sigma_m^2
\]

\section{单指数模型}
使单因素模型具备可操作性(寻找代理变量)的一个方法是将一些大型的股票指数视为共同宏观经济因素的有效代理指标。这一方法推导出和单因素模型相似的等式,称为单指数模型(single index model)。

\subsection{单指数模型的回归方程}
令证券超额收益率$ R_i=r_i-r_f $对$ R_M $回归,数据采用历史样本$ R_i(t) $和$ R_M(t) $配对,回归方程为:
\[
R_i(t)=\alpha_i+\beta_iR_M(t)+e_i(t)
\]
其中$ \alpha_i $为截距,$ e_i $为残差,其均值为0。

\subsection{期望收益与$ \beta $的关系}
\[
E(R_i)=\alpha_i+\beta_iE(R_M)
\]
从上式第二项可知证券风险溢价部分源自于指数的风险溢价,系统性风险溢价为指数的风险溢价乘以相对敏感系数$ \beta $。

风险溢价的剩余部分是$ \alpha $,为非市场溢价。当证券价格处于均衡时$ \alpha $会趋于0。

\subsection{单指数模型的风险和协方差}
总风险=系统性风险+公司特定风险
\[
\sigma_i^2=\beta_i^2\sigma_m^2+\sigma^2(e_i)
\]
协方差=$ \beta $的乘积×市场指数风险
\[
Cov(r_i,r_j)=\beta_i\beta_j\sigma_m^2
\]
相关系数=与市场之间的相关系数之积
\begin{equation*}
	\begin{split}
	Corr(r_i,r_j)&=\frac{\beta_i\beta_j\sigma_m^2}{\sigma_i\sigma_j}\\
	&=\frac{\beta_i\sigma_m^2\beta_j\sigma_m^2}{\sigma_i\sigma_m\sigma_j\sigma_m}\\
	&=Corr(r_i,r_M)\times Corr(r_j,r_M)
	\end{split}
\end{equation*}

\subsection{单指数模型的估计值}
对单因素模型进行估计需要:

n个超市场收益估计值,$ \alpha_i $

n个敏感系数估计值,$ \beta_i $

n个公司特有方差估计值,$ \sigma^2(e_i) $

1个市场溢价估计值,$ E(R_M) $

1个宏观经济因素方差的估计值,$ \sigma_m^2 $

\hspace*{\fill}

共3n+2个估计值。

\subsection{指数模型和分散化}
指数模型同样为投资组合的分散化提供了新的视角。假设我们选择等权重n个证券构成的组合,每个证券的超额收益率为:
\[
R_i=\alpha_i+\beta_iR_M+e_i
\]
则这一等权重组合的超额收益率为:
\begin{equation*}
	\begin{split}
	R_P&=\sum_{i=1}^{n}w_iR_i=\frac{1}{n}\sum_{i=1}^{n}R_i\\
	&=\frac{1}{n}\sum_{i=1}^{n}(\alpha_i+\beta_iR_M+e_i)\\
	&=\frac{1}{n}\sum_{i=1}^{n}\alpha_i+(\frac{1}{n}\sum_{i=1}^{n}\beta_i)R_M+\frac{1}{n}\sum_{i=1}^{n}e_i
	\end{split}
\end{equation*}
设
\[
\beta_P=\frac{1}{n}\sum_{i=1}^{n}\beta_i
\]
\[
\alpha_P=\frac{1}{n}\sum_{i=1}^{n}\alpha_i
\]
\[
e_P=\frac{1}{n}\sum_{i=1}^{n}e_i
\]
因此组合方差为:
\[
\sigma_P^2=\beta_P^2\sigma_M^2+\sigma^2(e_P)
\]
由于
\[
\sigma^2(e_P)=\sum_{i=1}^{n}(\frac{1}{n})^2\sigma^2(e_i)=\frac{1}{n}\overline{\sigma^2}(e)
\]
其中$ \overline{\sigma^2}(e) $为公司的平均方差。故$ \sigma^2(e_P) $趋于0。

总之,随着分散化程度增加,投资组合的总方差就会接近系统性风险$ \beta_P^2\sigma_M^2 $。

\section{估计单指数模型}
\subsection{惠普的证券特征线}

\subsection{惠普证券特征线的解释力}

\subsection{方差分析}

\subsection{$ \alpha $估计}

\subsection{$ \beta $估计}

\subsection{公司特有风险}

\subsection{相关性和协方差矩阵}


\section{组合构造与单指数模型}
\subsection{$ \alpha $和证券分析}
单指数模型框架输入数据的准备步骤。

1. 宏观经济分析,用于估计市场指数的风险和溢价。

2. 统计分析,用于估计$ \beta $系数和残差的方差$ \sigma^2(e_i) $

3. 用市场风险指数风险溢价和证券$ \beta $系数的估计值来构建证券的期望收益。

4. 准确的证券特有收益的预测($ \alpha $)从各种证券估值模型中得到,因此$ \alpha $值反映了证券分析中发现的私人信息带来的增量风险溢价。

\hspace*{\fill}

真正决定一个证券是否有吸引力的是它的$ \alpha $值。若$ \alpha>0 $则证券价格被低估,应买进;若$ \alpha<0 $,则证券价格被高估,应卖空。

\subsection{指数组合作为投资资产}
如果担心投资组合的分散化不足,可以直接投资于指数资产。

\subsection{单指数模型输入数据}

\subsection{单指数模型的最优风险组合}
运用估计的$ \alpha $和$ \beta $系数,加上指数组合的风险溢价,能得到n+1个期望收益值。运用$ \beta $系数的估计值和残差方差以及指数组合的方差,可以建立协方差矩阵。给定风险溢价和协方差矩阵,可以如第七章一般实行最优化程序。

\[
\alpha_P=\sum_{i=1}^{n+1}w_i\alpha_i
\]
对于指数而言,$ \alpha_{n+1}=\alpha_M=0 $
\[
\beta_P=\sum_{i=1}^{n+1}
\]
对指数而言,$ \beta_{n+1}=\beta_M=1 $
\[
\sigma^2(e_P)=\sum_{i=1}^{n+1}w_i^2\sigma^2(e_i)
\]
对指数而言,$ \sigma^2(e_{n+1})=\sigma^2(e_M)=0 $

\hspace*{\fill}

目标是通过组合权重的选择来最大化组合的夏普比率。得到组合的夏普比率为:
\begin{equation*}
	\begin{split}
	E(R_P)&=\alpha_P+E(R_M)\beta_P=\sum_{i=1}^{n+1}w_i\alpha_i+E(R_M)\sum_{i=1}^{n+1}w_i\beta_i\\
	\sigma_P&=[\beta_P^2\sigma_M^2+\sigma^2(e_P)]^{1/2}=[\sigma_M^2(\sum_{i=1}^{n+1}w_i\beta_i)^2+\sum_{i=1}^{n+1}w_i^2\sigma^2(e_i)]^{1/2}\\
	S_P&=\frac{E(R_P)}{\sigma_P}
	\end{split}
\end{equation*}

如果只对分散化感兴趣,将只持有市场指数。证券分析给我们去寻找非零$ \alpha $值证券的机会并选择不同的持有头寸。这种不同头寸的成本是对分散化的背离,换言之,承担了不必要的公司特有风险。这个模型显示最优化风险投资组合实在寻找$ \alpha $和偏离有效分散化之间的权衡。

最优风险组合被证明是由两个组合构成:1. 积极组合,称之为A,由n个分析过的证券组成;2. 市场指数组合,是第n+1中资产,目的是为了分散化,称之为消极组合并标记为组合M。

假定积极组合的$ \beta $值为1,此时积极组合的最优权重与$ \frac{\alpha_A}{\sigma^2(e_A)} $成比例,类似地指数组合的权重与$ \frac{E(R_M)}{\sigma_M^2} $成比例。因此积极组合的初始头寸为:
\[
w^0_A=\frac{\alpha_A/\sigma^2_A}{E(R_M)/\sigma_M^2}
\]
当$ \beta_A\ne 1 $,积极组合头寸应调整为:
\[
w^*_A=\frac{w^0_A}{1+(1-\beta_A)w^0_A}
\]

\subsection{信息比率}
投资于积极组合的权重为$ w_A^* $,投资于市场指数组合的权重为$ 1-w_A^* $。最优化组合的夏普比率会超过指数组合,其精确关系为:
\[
S_P^2=S_M^2+[\frac{\alpha_A}{\sigma(e_A)}]^2
\]
上式表明积极组合对整个风险投资组合夏普比率的贡献取决于它的$ \alpha $值和残差标准差的比率。称$ \frac{\alpha_A}{\sigma(e_A)} $为信息比率(information ratio)。因此最大化夏普比率相当于最大化积极组合信息比率。

当投资于每个证券的相对比例为$ \alpha_i/\sigma^2(e_i) $,此时积极组合的信息比率将实现最大化。故最大化夏普比时每个证券的权重为:
\[
w_i^*=w_A^*\frac{\alpha_i/\sigma^2(e_i)}{\sum_{i=1}^{n}\frac{\alpha_i}{\sigma^2(e_i)}}
\]
运用这组权重,可以得到每个证券对积极组合信息比率的贡献依赖于它们各自的信息比率,即:
\[
[\frac{\alpha_A}{\sigma(e_A)}]^2=\sum_{i=1}^{n}[\frac{\alpha_i}{\sigma(e_i)}]^2
\]
某一证券的加入对组合的正面贡献是增加了非市场风险溢价,对组合的负面影响则是公司特有风险带来组和方差的增加。

如果一个证券的$ \alpha $为负,则该证券在最有风险投资组合中应该为空头头寸。如果禁止卖空,则应将其从最优风险组合剔除。

当且仅当所有$ \alpha $为零时,指数组合是一个有效的投资组合。

\subsection{最优化过程总结}
风险最优组合的构造程序如下:

\hspace*{\fill}

1. 计算积极组合中每个证券的原始头寸: 
\[
w^0_i=\frac{\alpha_i}{\sigma^2(e_i)}
\]

2. 调整这些原始权重,使组合权重和为1,即:
\[
w_i=\frac{w_i^0}{\sum_{i=1}^{n}w_i^0} 
\]

3. 计算积极组合的$ \alpha $值:
\[
\alpha_A=\sum_{i=1}^{n}w_i\alpha_i
\]

4. 计算积极组合的残差:
\[
\sigma^2(e_A)=\sum_{i=1}^{n}w_i^2\sigma^2(e_i)
\]

5. 计算积极组合原始头寸:
\[
w_A^0=\frac{\alpha_A/\sigma^2_A}{E(R_M)/\sigma_M^2}
\]

6. 计算积极组合的$ \beta $值:
\[
\beta_A=\sum_{i=1}^{n}w_i\beta_i
\]

7. 调整积极组合的原始头寸:
\[
w_A^*=\frac{w_A^0}{1-(1-\beta_A)w_A^0}
\]

8. 此时最优风险组合的权重:
\[
w_M^*=1-w_A^*\quad w_i^*=w_Aw_i
\]

9. 计算最优风险组合的风险溢价:
\[
E(R_P)=(w_M^*+w_A^*\beta_A)E(R_M)+w_A^*\alpha_A
\]

10. 运用指数组合的方差和积极组合的残差计算最优风险组合的方差:
\[
\sigma_P^2=(w_M^*+w_A^*\beta_A)^2\sigma_M^2+[w_A^*\sigma(e_A)]^2
\]

\subsection{实例}


\section{指数模型在组合管理中的实际应用}
\subsection{指数模型比全协方差模型差吗}

\subsection{行业指数模型}

\subsection{预测$ \beta $}

\subsection{指数模型和跟踪证券组合}


\end{document}