\documentclass{article}
\title{最优风险资产组合}
\author{Dawei Wang}
\date{\today}
\usepackage{ctex}
\usepackage{amsmath}
\usepackage{amssymb}
\begin{document}
	\maketitle
投资决策可以看作是自上而下的过程:1. 风险资产和无风险资产之间的资本配置;2. 各类资产之间的配置;3. 每类资产内部的证券选择。

在本章和第八章讨论的组合基于一个短期的视野:即使整个投资期限很长,组合也可以通过调整各部分资产来重新平衡整个资产组合。在短期中,描述长期复利收益的偏度并不存在,因此正态假设可以足够精确地描述持有期收益。
\section{分散化与组合风险}
当所有风险都是公司层面上的,分散化可以将风险降至低水平。这是因为风险来源是相互独立的,那么组合对任何一种风险的敞口都可降至可以忽视的水平。保险原理(insurance principle),保险公司通过对很多独立的风险源做保险业务从而分散降低风险。

当普遍性的风险影响所有公司时,即使分散化也无法消除风险。这个无法消除的风险叫作市场风险(market risk)、系统性风险(systematic risk)或不可分散风险(nondiversifiable risk)。相反,可以消除的风险叫作独特风险(unique risk)、公司特有风险(firm-specific risk)或可分散风险(diversifiable risk)。

\section{两个风险资产的组合}
投资于债券基金的比例定义为$ w_D $,剩余的为$ 1-w_D $,定义为$ w_E $,投资于股票基金。则该投资组合收益率$ r_P $是:
\[
r_P=w_Dr_D+w_Er_E
\]
其期望是:
\[
E(r_P)=w_DE(r_D)+w_EE(r_E)
\]
方差是:
\[
\sigma^2_P=w_E^2\sigma^2_E+w_D^2\sigma^2_D+2w_Ew_DCov(r_E,r_D)
\]
由于
\[
Cov(r_E,r_D)=\sigma_E\sigma_D\rho_{ED}
\]
当$ \rho_{ED}=1 $
\[
\sigma^2_P=(w_E\sigma_E+w_D\sigma_D)^2
\]
当$ \rho_{ED}=-1 $
\[
\sigma^2_P=(w_E\sigma_E-w_D\sigma_D)^2
\]
此时若
\[
w_E\sigma_E-w_D\sigma_D=0
\]
则可实现完全对冲($ \sigma_P=0 $),此时:
\[
w_E=\frac{\sigma_D}{\sigma_E+\sigma_D}\quad w_D=\frac{\sigma_E}{\sigma_E+\sigma_D}
\]

当$ w_D>1,w_E<0 $卖空股票并购入债券;

当$ w_E<0,w_D>1 $卖空债券并购入股票;

只要$ \rho<\frac{\sigma_D}{\sigma_E} $,当我们从全部投资于债券开始逐渐增加股权投资时波动率将先下降。

\hspace*{\fill}

求解最小化方差组合(minimum-variance portfolio)
\[
min:\sigma^2_P=w_E^2\sigma^2_E+(1-w_E)^2\sigma^2_D+2(1-w_D)w_DCov(r_E,r_D)
\]
解得:
\[
w_D=\frac{\sigma_E^2-Cov(r_E,r_D)}{\sigma_E^2+\sigma_D^2-2Cov(r_E,r_D)}\quad w_E=\frac{\sigma_D^2-Cov(r_E,r_D)}{\sigma_E^2+\sigma_D^2-2Cov(r_E,r_D)}
\]

两个资产构造的所有期望收益和标准差的组合为投资组合可行集,完全正相关($ \rho=1 $)的资产分散化并没有意义。完全负相关的资产($ \rho=-1 $)的资产可以构造完全对冲($ \sigma_P=0 $)的资产组合。

\section{股票、长期债券、短期债券的资产配置}
优化资产配置实际上是想找出斜率最大或夏普比最大的资本配置线(CAL),斜率越大的CAL,任何给定波动性时相应的预期收益最大。现在考虑问题:构造包含主要资产类的风险资产组合以实现尽可能高的夏普比。

sharpe ratio:
\[
S_P=\frac{E(r_P)-r_f}{\sigma_P}
\]
约束条件:
\[
E(r_P)=w_EE(r_E)+w_DE(r_D)
\]
\[
\sigma_P=(w_E^2\sigma_E^2+w_D^2\sigma_D^2+2w_Ew_DCov(r_E,r_D))^{1/2}
\]
\[
w_E+w_D=1
\]
解得最优风险组合(optimal risky portfolio):
\[
w_D=\frac{E(R_D)\sigma_E^2-E(R_E)Cov(R_D,R_E)}{E(R_D)\sigma_E^2+E(R_E)\sigma_D^2-[E(R_D)+E(R_E)]Cov(R_E,R_D)}
\]
\[
w_E=1-w_D=\frac{E(R_E)\sigma_D^2-E(R_D)Cov(R_D,R_E)}{E(R_D)\sigma_E^2+E(R_E)\sigma_D^2-[E(R_D)+E(R_E)]Cov(R_E,R_D)}
\]

\hspace*{\fill}

构造整个组合的步骤:

(1) 确定所有证券的特征(期望收益率、方差、协方差);

\hspace*{\fill}

(2) 建立风险资产组合;

a. 计算最优风险资产组合P

b. 由a.计算组合P的期望收益和标准差

\hspace*{\fill}

(3) 在风险资产和无风险资产之间配置资金。

a. 计算风险资产组合P的比例

b. 计算整个组合中各资产的比例

\section{马科维茨资产组合选择模型}
\subsection{证券选择}
组合构造问题可以归纳为多个风险资产和一个无风险资产的情况。

第一步是决定投资者面临的风险收益机会。由风险资产的最小方差边界(minimum-variance frontier)给出。所有单个资产都在该边界的右方(至少存在卖空机制时是这样),说明单个资产构成的风险组合不是最有效的,分散化投资可以提升期望收益降低风险。

所有最小方差边界上最小方差组合上方的点提供最优的风险和收益,因此可以作为最优组合,这一部分称为风险资产有效边界(effcient frontier of risky assets)。

第二步是包含无风险资产的最优化,寻找夏普比最高的资本配置线。

最后一步是投资者在最优风险资产和无风险资产之间选择合适的比例。

风险资产组合边界的核心原理是,对于任意风险水平,只关注期望收益率最高的组合(边界是给定期望收益风险最小的组合集。)

\subsection{资本配置和分离特性}
设最小化方差组合为G,最小方差边界上存在非有效组合于最优风险组合的协方差系数为零,称之为组合Z。

组合边界的一个重要特性是,最小方差边界上的任何两个组合构造出的组合依然在边界上,它处在边界上的位置取决于组合的权重。因此组合P加组合G或Z可以得到整个有效边界。这个性质称之为分离性质。

给定投资经理所有证券的数据,最优风险组合对所有客户就是一样的。整个投资组合在无风险资产和最优风险组合之间的配置取决于客户的偏好。

事实上,不同客户的最优风险组合也因其各自的约束条件而不同,比如股利收益约束、税收因素和其他客户偏好。

\subsection{分散化的威力}
组合的方差为(令$ w_i=1/n $):
\begin{equation*}
	\begin{split}
	\sigma_P^2&=\sum_{i=1}^{n}\sum_{j=1}^{n}w_iw_jCov(r_i,r_j)\\
	&=\frac{1}{n}\sum_{i=1}^{n}\frac{1}{n}\sigma_i^2+\sum_{j=1,j\ne i}^{n}\sum_{i=1}^{n}\frac{1}{n^2}Cov(r_i,r_j)
	\end{split}
\end{equation*}

定义平均方差和平均协方差为:
\begin{equation*}
	\begin{split}
	\overline{\sigma^2}&=\frac{1}{n}\sum_{i=1}^{n}\sigma_i^2\\
	\overline{Cov}&=\frac{1}{n(n-1)}\sum_{j=1,j\ne i}^{n}\sum_{i=1}^{n}Cov(r_i,r_j)
	\end{split}
\end{equation*}
因此投资组合方差为:
\[
\sigma_P^2=\frac{1}{n}\overline{\sigma^2}+\frac{n-1}{n}\overline{Cov}
\]
设所有证券的标准差都为$ \sigma $,证券间的相关系数都为$ \rho $,协方差为$ \rho\sigma^2 $,此时:
\[
\sigma_P^2=\frac{1}{n}\sigma^2+\frac{n-1}{n}\rho\sigma^2
\]

当$ \rho=0 $,我们得到保险原理,组和方差在n变大时趋于零。当$ \rho=1 $,组和方差恒等于$ \sigma^2 $,分散化没有意义。更一般的情况下($ 0<\rho<1 $),当n增大,系统性风险趋近于$ \rho\sigma^2 $。

相关系数为正时,组合风险随着证券数量上升而下降的速度相对慢了很多,因为证券间的相关性限制了分散化空间。

当我们持有分散化组合,某一证券对整个组合风险的贡献主要取决于该证券和其他证券之间的协方差,而并非该证券的方差。这意味着风险溢价也主要取决于协方差而非方差变动。

\subsection{资产配置和证券选择}
证券选择和资产配置都需要构造有效边界,并在有效边界上选择一个最优组合。但区分证券选择和资产配置仍然是有意义的:

1. 社会对专业投资管理的需求上升;

2. 专业投资者的表现通常优于业余投资者;

3. 投资分析具有巨大的规模效应。

实践中总是独立地最优化每类资产中地证券选择,同时,高一级地管理会更新各资产类的最优情况并调整完善资产组合的投资权重。

\subsection{最优组合和非正态收益}

\section{风险集合、风险共享与长期投资风险}
\subsection{风险集合和保险原理}
风险集合(risk pooling)是指将互不相关的风险项目聚合在一起来降低风险。应用到保险行业,风险集合主要为销售风险不相关的保单,即保险原理。

风险集合增加了风险投资的规模,风险集合并不降低总体风险。

\subsection{风险共享}
风险共享(risk sharing),卖掉一部分风险资产来闲置风险的同时保持夏普比。

\subsection{长期投资}
横向增加风险资产到资产池中类似于把投资期限拓展到下一期(增加该时期收益的不确定性)。

长期投资累积了投资风险。将一项投资期限拓展提升了夏普比率也提升了风险,因此“时间分散化”并不是真正的分散化。

(1) 第一阶段全部投风险资产,第二阶段全部投无风险资产,夏普比为$ S=R/\sigma $。总风险溢价R,标准差$ \sigma $。

(2) 第一、二阶段全部投风险资产,相当于风险集合,夏普比为$ \sqrt{2}R/\sigma $。总风险溢价2R,标准差$ \sqrt{2}\sigma $。

(3) 第一、二阶段一半资产投资风险资产、一半投资无风险资产,相当于风险共享,夏普比为$ \sqrt{2}R/\sigma $。总风险溢价R,标准差$ \sigma/\sqrt{2} $。

\hspace*{\fill}

风险不会在长时段中消失,将全部预算投资于一个风险组合的投资者会发现尽可能在更多时段进行风险资产投资但降低每一期投资预算的策略更好。简单的风险集合不能降低风险。

\end{document}