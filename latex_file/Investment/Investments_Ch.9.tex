\documentclass{article}
\title{资本资产定价模型}
\author{Dawei Wang}
\date{\today}
\usepackage{ctex}
\usepackage{amsmath}
\usepackage{amssymb}
\begin{document}
	\maketitle
\section{资本资产定价模型概述}
资本资产定价模型(CAPM)的关键是假设所有投资者根据马科维茨原则最大化组合效用。

CAPM提出的问题是:如果所有投资者共享同样的投资集并用同样的投入组合来绘制有效边界,投资组合选择将会怎样?显然,他们的有效边界相同。面对同样的无风险利率,他们会绘制出同样的切线CAL,并自然而然地得到同样的最优风险资产组合P。所有投资者因此对每个风险资产有同样的持有比例。

CAPM的一个关键观点是:因为市场组合是所有风险组合的加总,市场组合内的资产比例也是投资者的持有比例。所以,如果所有投资者选择相同的风险资产组合,这个组合一定是市场组合,即可投资所有资产以市值加权平均得到的组合。

\subsection{为什么所有投资者都持有市场组合}
什么是市场组合?当把单个投资者的资产组合加总起来时,借与贷会相互抵消,其加总起来的风险资产组合的价值等于整个经济中的全部财富,这就是市场投资组合,用M表示。每只股票在这个资产组合中所占的比例等于股票的市值占股票总市值的比例。

所有资产都必须包括在市场投资组合中,唯一的区别在于在怎样的价位上投资者才愿意将一只股票纳入其最优资产组合。

\subsection{消极策略是有效的}
资本市场线是资本配置线的一个特例:在CAPM的简单形式中,市场投资组合M是有效边界与资本市场线的切点。

在这里,所有投资者持有的市场投资组合都建立在相同的输入列表上,因此它们能体现出证券市场中所有的相关信息。

因此投资于市场指数组合这种消极策略是有效的,有时把这一结论称之为共同基金原理(mutual fund theorem)。可以将不同投资管理者创立的很多不同于市场指数的风险资产组合理解为是由于在最有资产组合中不同的输入列表造成的。共同基金原理的重要性在于它为投资者提供了一个消极投资的渠道,投资者可以将市场指数视作合理的、最有效的资产组合。

\subsection{市场组合的风险溢价}
如果所有投资者选择投资于市场组合M和无风险资产,则应怎样确定市场投资组合M的风险溢价?

假设每位投资者投资于最有资产组合M的比例为y,则:
\[
y=\frac{E(r_M)-r_f}{A\sigma^2_M}
\]
在简化形式的CAPM中,无风险资产包括所有投资者之间的借入和贷出,投资者整体之间借贷抵消之后净借入和净贷出的总和为0,称此时的风险厌恶系数为$ \overline{A} $,对$ \overline{A} $而言,风险资产组合的平均比例为100\%,即$ \overline{y}=1 $。故:
\[
E(R_M)=\overline{A}\sigma^2_M
\]
即市场投资组合的风险溢价预期方差和平均风险厌恶水平有关。
\subsection{单个证券的预期收益}
CAPM认为,单个证券的合理风险溢价取决于单个资产对投资者的所有资产组合风险的贡献程度。资产组合风险对于投资者而言,其重要性在于投资者根据资产组合风险来确定他们要求的风险溢价。

每只股票对资产组合方差的贡献率可以表述为该股票于其他所有股票的协方差及其自身方差之和(权重系数相乘)。以GE公司股票为例:

GE公司股票对市场组合方差的贡献为:
\[
w_{GE}[\sum_{i=1}^{n}w_iCov(R_i,R_{GE})=\sum_{i=1}^{n}Cov(w_iR_i,R_{GE})=Cov(\sum_{i=1}^{n}w_iR_i,R_{GE})]
\]
由于:$ \sum_{i=1}^{n}w_iR_i=R_M $,故:
\[
\sum_{i=1}^{n}w_iCov(R_i,R_{GE})=Cov(R_M,R_{GE})
\]
因此GE公司股票对市场投资组合的方差的贡献程度为:$ w_{GE}Cov(R_{GE},R_M) $

\hspace*{\fill}

同时,GE公司股票对市场投资组合的风险溢价的贡献为$ w_{GE}E(R_{GE}) $。

\hspace*{\fill}

因此,GE公司股票的回报-风险比率可以表达为:
\[
\frac{w_{GE}E(R_{GE})}{w_{GE}Cov(R_{GE},R_M)}=\frac{E(R_{GE})}{Cov(R_{GE},R_M)}
\]

\hspace*{\fill}

定义风险的市场价格为
\[
\frac{E(R_M)}{\sigma^2_M}
\]

\hspace*{\fill}

均衡的一个基本原则是所有投资应具有相同的回报-风险比率。因此令GE公司的风险-回报比率与市场组合的相等:
\[
\frac{E(R_{GE})}{Cov(R_{GE},R_M)}=\frac{E(R_M)}{\sigma^2_M}
\]
得到:
\[
E(R_{GE})=\frac{Cov(R_{GE},R_M)}{\sigma^2_M}[E(R_M)]
\]

这里$ Cov(R_{GE},R_{M})/\sigma^2_M $这一比率衡量了GE股票对市场投资组合方差的贡献程度,也被称作$ \beta $。GE股票的期望收益可以表达为:
\[
E(r_{GE})=r_f+\beta[E(r_M)-r_f]
\]

对某一资产组合P,$ E(r_P)=\sum_{i=1}^{n}w_iE(r_i) $,$ \beta_P=\sum_{i=1}^{n}w_i\beta_i $。

市场组合$ \beta_M=1 $。

证券市场价格已经反映了有关公司市场前景的一切公开信息,因此只有公司的风险($ \beta $)会影响到期望收益率。

\subsection{证券市场线}
由于$ \beta $值与证券对最优风险组合风险的贡献程度成正比,因此证券的$ \beta $值是证券风险的适当指标。

期望收益-$ \beta $关系就是证券市场线(security market line, SML)。

证券市场线和资本市场线的比较。资本市场线描述了有效资产组合的风险溢价(有效资产组合是指由风险资产和无风险资产构成的资产组合)是资产组合标准差的函数。标准差可以用来衡量有效分散化的资产组合,即投资者总的资产组合的风险。相较而言,SML刻画的是单个风险资产的风险溢价,它是该资产风险的函数。作为高度分散化资产组合一部分的单项资产的风险测度并不是资产的标准差或方差,而是该资产对资产组合方差的贡献程度,用$ \beta $值来测度这一贡献程度。

由于证券市场线是期望收益-$ \beta $关系的几何表述,所以“公平定价”资产一定在资本市场线上。股票的实际期望收益率与正常期望收益之间的差称之为股票的$ \alpha $。

\subsection{资本资产定价模型和单指数市场}
CAPM的关键应用可以归纳为两点:

1. 市场组合是有效的。

2. 一个风险资产的风险溢价与它的$ \beta $值成正比。

以上的讨论为从市场组合是有效的到期望收益-$ \beta $关系,现利用指数模型市场结构来考察从期望收益-$ \beta $关系到市场组合有效性。

\hspace*{\fill}

现假设投资者都面临这样一个市场,其中股票超额收益$ R_i $,符合正态分布,且这个超额收益由单一系统性因素引起,假设可以用一个涉及众多股票的、按市值加权平均的股指M来描述宏观因子。
\[
R_i=\alpha_i+\beta_iR_M+e_i
\]
$ e_i $是公司特有的、均值为0的残差项,并于市场因素$ R_M $无关。残差代表分散化的、非系统性的或特异的风险。股票的风险溢价和方差为:
\[
E(R_i)=\alpha_i+\beta_iE(R_M)
\]
\[
\sigma^2_i=\beta^2_i\sigma_M^2+\sigma^2(e_i)
\]
一个用n只股票对应权重集$ w_i $构成的组合Q,其收益为:
\[
R_Q=\sum_{i=1}^{n}w_i\alpha_i+\sum_{i=1}^{n}w_i\beta_iR_M+\sum_{i=1}^{n}w_ie_i=\alpha_Q+\beta_QR_M+e_Q
\]
由于:

1. 投资者会分散化风险,残差风险$ \sigma^2(e_Q)=\sum_{i=1}^{n}w_i^2\sigma^2(e_i) $会变小;

2. Q的风险溢价会因为选择正$ \alpha $做空负$ \alpha $而增加。

因此正$ \alpha $的股票的价格会增加,负$ \alpha $的股票的价格会减少。投资者会追求完全消除特异性风险来最小化风险,即持有尽可能广泛的市场组合。当所有股票$ \alpha $都为0时,市场组合时最优风险组合。

\section{资本资产定价模型的假设和延伸}
\subsection{资本资产定价模型的假设}
个体行为:

1. 投资者都是理性的,追求均值-方差最优化;

2. 只考虑单期投资期限;

3. 投资者具有一致预期(同质期望)。

\hspace*{\fill}

1. 所有资产在交易所公开交易、允许做空、投资者可以按无风险利率借入或贷出资金;

2. 所有信息公开可获取;

3. 无税;

4. 无交易成本。

\subsection{对资本资产定价模型的质疑}
1. 做空比做多难:

(1)一项资产空头头寸的投资者的负收益可能是无限的,因此做空需要很大的抵押,这部分抵押无法进行投资获得收益;

(2)任何提供给做空者的股份都是有限的;

(3)很多投资公司不允许做空,有的国家法律禁止做空。

没有做空经济泡沫很难戳破。

\hspace*{\fill}

2. 3个不切实际的假设:所有资产可交易、不存在交易费用、单期投资。

\subsection{零$ \beta $模型}
有效边界资产组合的两个特点:

(1)两个有效边界上的投资结合而成的任何资产组合都在有效边界上。

(2)有效边界上的任一资产组合,除去最小方差组合,都在有效边界下半部分存在一个与其不相关的“伴随”资产组合,这些伴随资产组合叫作有效组合的零$ \beta $投资组合。

\subsection{工资收入于非交易性资产}

\subsection{多期模型于对冲组合}

\subsection{基于消费的资本资产定价模型}

\subsection{流动性与资本资产定价模型}


\section{资本资产定价模型和学术领域}

\section{资本资产定价模型和投资行业}

	
\end{document}