\documentclass{article}
\title{套利定价理论与风险收益多因素模型}
\author{Dawei Wang}
\date{\today}
\usepackage{ctex}
\usepackage{amsmath}
\usepackage{amssymb}
\begin{document}
	\maketitle
利用证券之间的错误定价来赚取无风险利润的行为称为套利(arbitrage)。在均衡市场价格的情况下没有套利机会也许是资本市场理论最基本的原理。

将因子模型与无套利条件相结合得到期望收益和风险之间的关系。这种风险收益之间平衡的方法叫作套利定价理论(APT)。

\section{多因素模型概述}
单指数模型中,市场投资组合反映了 宏观因素的重要影响。资产风险溢价可能与市场外的风险因子,如通胀、利率、波动性、市场风险溢价和$ \beta $有关。

所有宏观因素的期望值都为0:这代表这些变量的变化没有被预期到。每个因素的系数($\beta$)度量了股票收益对该因素的敏感程度。

多因素模型仅仅是对影响证券收益的因素进行描述,在模型中并不存在什么理论描述期望收益的来源。

\section{套利定价理论}
套利定价理论(arbitrage pricing theory, APT)预测了与期望收益相关的证券市场线。套利定价理论基于三个基本假设:1. 因素能描述证券收益;2. 市场上有足够的证券来分散风险;3. 完善的证券市场不允许任何套利机会存在。

\subsection{套利、风险套利和均衡}
一价定理:如果两项资产在所有的经济性方面均相同,那么它们应该具有相同的市场价格。

一个无风险套利投资组合最重要的性质是:不管其风险厌恶程度和财富水平如何,投资者都愿意持有一个无限的头寸。大量的头寸价格使价格上涨或下跌至套利机会完全消除。

套利与风险-收益占优的均衡价格形成机制不同,风险-收益占优观点认为资产价格背离均衡价格时,投资者会改变其投资组合,这取决于他们的风险厌恶程度;套利的观点认为当套利机会存在时,每个投资者都愿意尽可能地多持有头寸,因此不需要很多投资者就会给价格带来压力使价格恢复平衡。

对于衍生品,其市场价值完全由其他证券的价值决定,因此严格套利是完全可能的,而对于股票以及价值不是严格地由其他一种或多种资产所决定的原始证券,无套利条件一定要从分散化投资的角度得出。

\subsection{充分分散的投资组合}
如果构建一个n只股票的投资组合,其权重为$ w_i $,$ \sum w_i=1 $,则投资组合的收益率可表述为:
\[
R_P=E(R_P)+\beta_PF+e_P
\]
其中,$ \beta_P=\sum w_i\beta_i $;$ E(R_P)=\sum w_iE(R_i) $。$ e_P=\sum w_ie_i $是n种股票的加权平均值。

可以将投资组合的方差分为系统的和非系统的两个方面:
\[
\sigma^2_P=\beta_P^2\sigma_F^2+\sigma^2(e_P)
\]
注意非系统性风险:$ \sigma^2(e_P) $

假设投资组合是等权重的,即$ w_i=1/n $,非系统方差为:
\[
\sigma^2(e_P)=\sum(\frac{1}{n})^2\sigma^2(e_i)=\frac{1}{n}\sum\frac{\sigma^2(e_i)}{n}=\frac{1}{n}\overline{\sigma^2}(e_i)
\]
当投资组合n变大时,非系统方差趋近于0。

由于任何充分分散的投资组合的$ e_P $的期望值为0,同时方差也趋近于0,因此在充分分散的投资组合中:
\[
R_P=E(R_P)+\beta_PF
\]
对于充分分散的投资组合,其收益完全由系统因素所决定。

在单因素的情况下,充分分散的组合的所有配对都是完全相关的:它们的风险完全由相同的系统性因素决定。考虑另一个充分分散组合Q,其收益率为$ R_Q=E(R_Q)+\beta_QF $则:
\[
\sigma_P=\beta_P\sigma_F;\quad\sigma_Q=\beta_Q\sigma_F
\]
\[
Cov(R_P,R_Q)=\beta_P\beta_Q\sigma^2_F
\]
\[
\rho_{PQ}=\frac{Cov(R_P,R_Q)}{\sigma_P\sigma_Q}=1
\]

\subsection{实践中的分散化和残差风险}
在理论上,如果投资集足够大的话,即使对于很不平衡的组合,分散化也可以消除风险。

\subsection{实施套利}
假设一个单因子市场,充分分散化的组合M代表市场因子M。任意证券的超额收益由式$ R_i=\alpha_i+\beta_iR_M+e_i $决定,则一个充分分散化的组合P的超额收益满足:
\[
R_P=\alpha_P+\beta_PR_M
\]
\[
E(R_P)=\alpha_P+\beta_PE(R_M
)
\]
无论式组合M还是组合P都没有残差风险,两个组合唯一风险来源就是系统性风险。构造零$\beta$组合Z:
\[
\beta_Z=w_P\beta_P+w_M\beta_M=0
\]
解得:
\[
w_P=\frac{1}{1-\beta_P};\quad w_M=1-w_P=\frac{-\beta_P}{1-\beta_P}
\]
此时:
\[
E(R_Z)=w_P\alpha_P=\frac{1}{1-\beta_P}\alpha_P
\]

\subsection{套利定价理论和无套利等式}
套利活动会迅速将任意零$\beta$充分分散组合的风险溢价调至0。因此任意充分分散化的组合在无套利空间时$\alpha$也必须为0。因此无套利条件为:
\[
E(R_P)=\beta_PE(R_M)
\]
意即通过套利定价理论的“无套利要求”得到充分应用于充分分散组合的CAPM证券市场线。为了排除套利机会,所有充分分散的投资组合的期望收益必须在无风险资产线上。

\section{套利定价理论、资本资产定价模型和指数模型}
\subsection{套利定价理论与资本资产定价模型}
当积极型组合的规模或者投资集的规模有限的时候,我们无法分散资产的巨大残差风险。如果残差风险狠搞且实现分散化的途径困难,APT和套利活动将不再适用。

CAPM要求几乎所有的投资者都是均值-方差最优化者而套利定价理论则不需要这一假设。少部分精明的套利者会将市场的套利机会消除这一假设是充分的。APT产生的一条证券市场线对于所有资产(残差风险较大的资产除外)都是一个较好的无偏估计。

更为重要的是APT是由可观测的诸如市场指数这样的组合锚定的。而由于CAPM依赖于一个无法观测到的无所不包的组合,它实际是不可检验的。但在单个资产和高残差风险这一层次,APT并不是完全优于CAPM的。

在CAPM中运用可观测的较大指数组合来替换不可观测到的市场组合时,这种做法不是有效的。我们无法确定对于所有资产,CAPM是否能无偏地估计它们的风险溢价。APT和CAPM都是有缺陷的。

\subsection{单指数市场中的套利定价理论和组合的最优化}
Treynor-Black模型中,当积极组合的残差风险为0时,这个组合的投资头寸趋向于无穷大,这和APT预示的一样。当组合充分分散化时,将无限地扩大套利头寸。同理,当积极型T-B组合中的某个资产地残差风险为0时,它将代替组合中所有其他的资产,并最终使得整个积极型组合得残差风险为0,从而导出同样极端的组合响应。

当残差组合非0时,T-B组合会产生一个最有风险组合,它是介于追求正的$\alpha$和排除潜在分散化风险之间的一种妥协产物。APT直接忽视了残差风险,因此此时并不适用。当残差风险可以通过分散化变小时,T-B模型解决了错误定价证券的激进头寸问题,这些证券会对均衡风险溢价造成巨大压力并最终消除非零的$\alpha$值。T-B模型完成了APT的目标,但是可以更加灵活以致更好地适应分散化受限的现实操作。

\section{多因素套利定价理论}
将单因素模型拓展为两因素模型:
\[
R_i=E(R_i)+\beta_{i1}F_1+\beta_{i2}F_2+e_i
\]

纯因子组合(factor portfolio):构建一个充分分散的投资组合,该组合只对模型中的一个因素的$\beta$为1,对其他所有因素的$\beta$为0。可以将一个纯因子组合看作跟踪某些特殊宏观经济风险来源的演变而与其他风险来源无关的跟踪投资组合。

任何投资组合P所面临的风险因素都由$ \beta_{P1} $和$ \beta_{P2} $来表示。

如果不存在套利机会,$\beta$值为$ \beta_{P1} $和$ \beta_{P2} $的充分分散投资组合一定有:
\[
E(R_P)=\beta_{P1}E(R_1)+\beta_{P2}E(R_2)
\]
除非每一个证券都可以单独地满足条件,否则上式不可能使每一个充分分散风险的投资组合都满足条件。

\section{Fama-French(FF)三因素模型}
现在的主流方法用公司特征来表示系统性风险相关来源的代理变量。其中具有代表性的为Fama-French三因素模型:
\[
R_{it}=\alpha_i+\beta_{iM}R_{Mt}+\beta_{iSMB}SMB_t+\beta_{iHML}HML_t+e_{it}
\]
其中SMB代表小减大,即市值规模小的股票投资组合于市值规模大的投资组合的收益差;HML为高减低,即高账面-市值比的股票与低账面-市值比的股票的收益差。其中市场指数用于测量源于宏观经济因素的系统性风险。

Fama和French指出,高账面-市值比的公司更容易陷入财务危机,而小公司对商业条件变化更加敏感。因此这些变量可以反映宏观经济风险因素的敏感度。

\end{document}