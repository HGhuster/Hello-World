\documentclass{article}
\title{金融资产类别与金融工具}
\author{Dawei Wang}
\date{\today}
\usepackage{ctex}
\usepackage{amsmath}
\usepackage{amssymb}
\begin{document}
	\maketitle
金融市场通常被分为货币市场和资本市场。货币市场工具主要包括期限短、变现能力强、流动性好、风险低的债务证券。货币市场工具有时被称为现金等价物,或简称现金。相反,资本市场主要由期限较长、风险较大的证券组成。资本市场上的证券种类远多于货币市场,因此又可将资本市场细分为4个部分:长期债券市场、权益市场以及期权与期货衍生工具市场。
\section{货币市场}
货币市场是固定收益市场的一部分,它由变现能力极强的超短期债务证券组成。大多数这类证券交易的面值很大,个人投资者无实力购买,但他们可以通过货币市场基金参与这个市场。货币市场基金汇集投资者的资金,并以他们的名义购买各种货币市场证券。
\subsection{短期国库券(T-bills)}
政府通过向公众出售国库券筹集资金,投资者以面值的一定折扣购入国库券,当国库券到期时,政府按面值从持有者手里赎回,购买价格与面值之差构成投资者的投资收益。

个人可以直接在一级市场上拍卖取得,也可以在二级市场上从政府证券交易商那里购入。

短期国库券的收益可以免除所有的州和地方税,这是短期国库券区别于其他货币市场工具的又一特征。

\subsection{大额存单 (certificate of deposit,CD)}
大额存单是一种银行定期存款,不能随时提取,银行旨在大额存单到期时才向储户利息和本金。但面值超过100 000美元的大额存单通常是可以转让的。

短期大额存单变现能力很强,但3月期以上的大额存单流动性大打折扣。

\subsection{商业票据}
大型公司发行的短期无担保债务票据。商业票据由一定的银行信贷额度支持,这样可以保证借款者在票据到期时有足够的现金来清偿。

期限通常在1、2个月以内,面值一般是10 000美元的倍数,因此小型投资者不能直接投资商业票据,只能通过货币市场上的共同基金投资。

通常由非金融公司发行,但近年来一些金融类公司开始发行资产支持商业票据。

\subsection{银行承兑汇票}
银行承兑汇票(banker's acceptance)是指由银行客户向银行发出在未来某一日期支付一笔款项的指令。期限通常在6个月以内。

类似于远期支票,当银行背书承兑后,银行开始负有向汇票持有者最终付款的责任。(可以在二级市场上交易,在面值的基础上折价销售。)

\subsection{欧洲美元}
欧洲美元(Eurodollars)是指国外银行或美国银行的国外分支机构中以美元计价的存款。由于这些银行或分支机构位于美国国外,因此可以不受美联储的监管。

大多数欧洲美元存款是数额巨大期限短于6个月的定期存款。

\subsection{回购和逆回购}
政府证券的交易商使用回购协议(repurchase agreements, repos 或RPs)作为一种短期(通常是隔夜)借款手段。交易商把政府债券卖给投资者,并签订协议在第二天以稍高的价格购回。协议约定的价格增幅为隔夜利息。

定期回购本质与普通回购一样,只是定期回购的期限可以超过30天。

逆回购是交易商找到持有政府证券的投资者买入证券,并协定在未来某一日期以稍高的价格售回给投资者。

\subsection{联邦基金}
联储会员银行的准备金账户的资金叫做联邦基金(federal funds)。联邦基金利率就是联邦基金市场上银行间拆借的利率(通常是隔夜交易)。

联邦基金利率倾向于成为一个货币政策松紧的度量指标。

\subsection{经纪人拆借}
通过支付保证金形式购买股票的个人投资者可以向经纪人借款来支付股票。经纪人可能向银行借款,并协定只要银行需要将即时归还。这种借款利率通常比短期国库券利率高出1\%。

\subsection{伦敦同业拆借市场}
伦敦银行同业拆借利率(London interbank offered rate,LIBOR)是位于伦敦的大型银行之间相互借款的利率。这种依据美元计价的贷款而确定的利率已经成为欧洲货币市场上短期借款的主要利率报价,也成为很多金融交易中的参考利率。

除美元外,LIBOR还可能与其他多种货币挂钩。例如,LIBOR广泛地用于以英镑、日元、欧元等计价的交易。

\section{债券市场}
债券市场由长期借款或债务工具组成。主要包括中长期国债、公司债券、市政债券、抵押证券和联邦机构债券。

有时候人们认为这些工具构成了固定收益资本市场,因为它们中的大多数都承诺支付固定的收入流或是按特定公式来计算的收入流。

\subsection{中长期国债}
中期国债(treasury note)和长期国债(treasury bond):

中期国债的期限为1-10年,长期国债的期限为10-30年面值为1000美元,半年付息。其YTM为APR。

\subsection{通胀保值债券}
TIPS(treasury inflation-protected securities)。这种债券的本金需要根据消费者物价指数(CPI)调整,因此它们可以提供不变的实际货币收益流。

\subsection{联邦机构债券}
联邦抓房贷款银行(FHLB)、联邦国民贷款协会(FNMA,房利美)、国民政府抵押贷款协会(GNMA)、联邦住房贷款抵押公司(FHLMC,房地美)。

\subsection{国际债券}
有许多公司从国外借款,也有许多投资者购买国外发行的证券。

欧元债券是一种以发行国以外货币计价的债券。

与以外币计价的债券相对应,许多外国公司在发行国以发行国货币计价的债券。例:扬基债券:非美发行者在美国发行的以美元计价的债券,武士债券是指非日发行者在日本发行的以日元计价的债券。

\subsection{市政债券}
市政债券(municipal bond)是由州和地方政府发行的债券。市政债券无需缴纳联邦所得税,在发行州也无需缴纳州和地方税。

市政债券通常分为两类:

一般责任债券和收入债券。

一般责任债券完全由发行者的信用支撑(即征税能力);而收入债券则是为特定项目筹资发行的,并由该项目获得的收入或运作该项目的特定市政机构担保(风险较一般责任债券高)。

产业发展债券是一种为商业企业筹措资金的债券。事实上这种以鼓励私营企业发展为目的的债券使企业可以获得像市当局那样的免税借贷,但是联邦政府限制这类证券的发行量。

\hspace*{\fill}

应税等值收益率(equivalent taxable yield):

假设投资者的边际税率(联邦与州复合税率)为t,r表示应税债券的税前收益率,那么r(1-t)则表示这些债券的税后收益率

设$ r_m $为市政债券的收益率,则其应税等值收益率为:

\[
r=r_m/(1-t)
\]

\subsection{公司债券}
私营企业通过发行公司债券直接向公众借款。半年付息。可分为抵押债券、无抵押债券(信用债券)以及次级债券。

抵押债券:公司破产时有担保物支持;

无抵押物债券:没有担保物支持;

次级债券:公司破产时,对资产的求偿权位于其他债券之后的债券。

可转债:

可赎回债券:

\subsection{抵押贷款和抵押担保证券}
抵押担保证券既代表了对抵押贷款资产池的求偿权,也代表了由该资产池做担保的一项负债。这种求偿权代表了抵押贷款的证券化,抵押贷款的贷款者发放贷款,然后将这些贷款打包并在二级市场销售。具体来讲,他们销售的是抵押贷款被偿还时的求偿权。贷款发起者继续为这些贷款服务,负责收取本金和利息并转交给抵押贷款的购买者。因此抵押担保证券也叫转递证券。

\section{权益证券}
\subsection{代表所有权股份的普通股}
普通股(common stock)又称为权益证券或权益,代表对公司的所有权份额。

多数大型公司的普通股可以在一个或多个股票交易所自由买卖。股票不能公开交易的公司叫做封闭式持股公司。

\subsection{普通股的特点}
剩余索取权(residual claim)和有限责任(limited liability)。

\subsection{优先股}
优先股(preferred stock)具有权益和债务的双重特征。
它承诺每年支付固定收益;

没有投票权;

股利累积(支付清优先股股利之前不能支付普通股股利);

公司计算应税收益时可以扣除从国内公司收到的70\%的股利。(个人不能享受此政策)

求偿权介于普通股和债券之间。

\hspace*{\fill}

此外有可赎回优先股、可转换优先股、浮动利率优先股。

\subsection{存托凭证}
美国存托凭证(American depository receipt,ADR)是一种在美国市场上交易的代表对国外公司所有权份额的凭证。每张凭证都与某一国外公司的部分股份相对应。推出存托凭证的目的是使国外公司更容易满足美国注册证券的要求。存托凭证是美国投资者投资海外公司股票最常用的方式。

\section{股票市场指数与债券市场指数}
道琼斯工业平均指数(DJIA)是其成分股价格的简单平均数,即把该指数包含的30只成分股的价格加起来除以30。因此,道琼斯工业平均指数变化的百分比即为30只股票平均价格变化的百分比。DJIA被称为价格加权平均(price-weighted average)指数。

股票拆分和替换成分股时需要重新计算除数以确保指数不变。

\subsection{标准普尔500指数}
市值加权平均(market-value-weighted index)。(现在大多数指数使用的是市值加权修正方法,该方法赋予权重时不是根据每只股票的总市值,而是根据公众持股的市值,即投资者可自由买卖的股票的市值。)

\subsection{等权重指数}
市场表现有时会用指数中每只股票收益率的等权重平均值来衡量。这种平均方法赋予每种收益率相同的权重,即对指数中的每只股票的投资金额相等。

其他指数:
日本日经指数:Nikkei、英国富时指数:FTSE、德国综合指数:DAX、中国香港恒生指数:Hang Seng 和加拿大多伦多股市指数:TSX

\subsection{债券市场指标}
三大债券市场指数:美林指数、巴克莱指数、所罗门美邦指数。


\section{衍生品市场}

\end{document}