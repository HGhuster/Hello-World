\documentclass{article}
\title{风险与收益入门及历史回顾}
\author{Dawei Wang}
\date{\today}
\usepackage{ctex}
\usepackage{amsmath}
\usepackage{amssymb}
\begin{document}
	\maketitle
\section{利率水平的决定因素}
利率水平由一些基本要素决定:1.来自于存款人(家庭)的资金供给;2.来自于企业投资工厂车间、设备以及存货的融资需求;3.通过央行运作调整后的政府净资金供给或需求。

\subsection{名义利率和实际利率}
严格来讲,名义利率和实际利率有下式成立:
\[
1+rr=\frac{1+rn}{1+i}
\]
其中rr为实际利率,rn为名义利率,i为通胀率
近似地讲:
\[
rr=rn-i
\]

\subsection{均衡实际利率}
三个基本因素:供给、需求和政府行为决定了实际利率,影响名义利率的第四个因素是通胀率。

尽管实际利率最为基本的决定因素是居民的储蓄倾向和投资项目的预期生产率,其同时也受到政府财政政策或货币政策的影响。

\subsection{均衡名义利率}
欧文·费雪认为名义利率应该伴随着预期通胀率E(i)的增加而增加。费雪等式即为:
\[
rn=rr+E(i)
\]

实证研究很难证实费雪的假设,这是因为实际利率往往也在发生着无法预测的变化,名义利率可以被视为名义上无风险资产的必要收益率加上通胀“噪声”的预测值。

\subsection{税收与实际利率}
税赋是基于名义收入的支出,税率则由投资者的税收累进等级决定。

同价格指数联系的税收累进制度并没有将个人收入的纳税完全同通胀率分离开来:
\[
rn(1-t)-i=(rr+i)(1-t)-i=rr(1-t)-it
\]
因此税后实际利率随着通胀率上升而下降,投资者相当于承受了税率乘以通胀率的通胀损失。

\section{比较不同持有期的收益率}
假设国库券(零息债券)价格为P(T),面值为100美元,持有期为T年,把期限为T年的无风险收益率表示成投资价值增长的百分比:
\[
r_f=\frac{100}{p(T)}-1
\]

国库券的有效年利率:
\[
1+EAR=[1+r_f(T)]^{1/T}
\]

\subsection{年化百分比利率,APR}
对一个期限为T的投资者来说:
\[
1+APR\times T=(1+EAR)^T
\]
\subsection{连续复利}
设连续复利时的年化百分比利率为$ r_{cc} $:
\[
1+EAR=e^{r_{cc}}
\]
\[
r_{cc}=ln(1+EAR)
\]
\section{国库券与通货膨胀}

\section{风险与风险溢价}
\subsection{持有期收益率}
实现的收益率,也叫持有期收益率(holding-period return,HPR):

HPR=(期末每份价格-期初价格+现金股利)/期初价格

持有期收益率的定义假设股利在持有期期末支付。如果股利支付提前,那么持有期收益率便忽略了股利支付点到期末这段时间的股利在投资收益。来自股利的收益率被称为股利收益率(dividend yield),所以股利收益率加上资本利得收益率等于持有期收益率。

\subsection{期望收益率和标准差}
期望收益率:
\[
E(r)=\sum_sp(s)r(s)
\]
方差:
\[
\sigma^2=\sum_sp(s)[r(s)-E(r)]^2
\]
标准差:
\[
\sigma=\sqrt{\sigma^2}
\]
\subsection{超额收益和风险溢价}
风险溢价(risk premium):预期持有期收益率(HPR)和无风险利率(risk-free rate)的差值;

超额收益率(excess return):实际收益率与实际无风险收益率的差值。

风险溢价的超额收益的期望,超额收益的标准差是其风险的测度。

投资者投资股票的意愿取决于其风险厌恶(risk aversion)水平
\section{历史收益率的时间序列分析}
\subsection{时间序列与情境分析}

\subsection{期望收益和算术平均值}
使用历史数据时,认为每一个观测值等概率发生。如果有n个观测值,此时期望收益为:
\[
E(r)=\sum_sp(s)r(s)=\frac{1}{n}\sum_{s=1}^{n}r(s)
\]
\subsection{几何(时间加权)平均收益}
\[
FV=(1+r_1)\times(1+r_2)\times\cdots(1+r_n)
\]
\[
g=FV^{1/n}-1
\]

收益率波动越大,两种平均方法的差异越大。如果收益服从正态分布,预期差异为分布方差的1/2,即:
E[几何平均值]=E[算数平均值]-1/2$ \sigma^2 $

\subsection{方差和标准差}
样本方差:
\[
\hat{\sigma}^2=\frac{1}{n-1}\sum_{s=1}^{n}[r(s)-\overline{r}]^2
\]

样本标准差:
\[
\hat{\sigma}=\sqrt{\hat{\sigma}^2}
\]

\subsection{高频数据种的均值与方差估计}
观测值得频率不会影响均值估计的准确性。样本时段的长度而非样本观测值的数量能改进估计的准确性。

均值和方差随时间段成比例增长,而标准差随时间段长度的平方根的增长而增长。

\subsection{收益的波动性(夏普)比率}
夏普比率(sharpe ratio):风险溢价/超额收益率的标准差

夏普比率将风险溢价(与时段长度等比例变化)除以标准差(与时段长度为平方根关系)。因此用高频收益计算年化时夏普比增大。总体来说,一项长期投资为T年的夏普比率以$ \sqrt{T} $的比率增加。

\section{正态分布}

\section{偏离正态分布和风险度量}
正态分布保证标准差是风险的完美度量,因此夏普比率是证券表现的完美度量。然而,很多情况下资产的收益率显著偏离正态分布。

关于不对称性的度量:

偏度(skewness):

\[ E[\frac{(R-\overline{R})^3}{\hat{\sigma}^3}] \]

偏度为正时,标准差高估风险;偏度为负时,标准差低估风险。

\hspace*{\fill}

峰度(kurtosis):

\[ E[\frac{(R-\overline{R})^4}{\hat{\sigma}^4}]-3 \]

峰度为正时,存在肥尾分布。

\hspace*{\fill}

极端负值可能由负偏度以及正峰度产生。极端负收益的频繁发生会导致出现负偏和肥尾。因此需要揭示极端负收益发生的风险测度。

业界使用最频繁的极端负收益测度为:在险价值、预期损失、下偏标准差和极端收益频率。

\subsection{在险价值}
在险价值(risk at value, VaR)是度量一定概率下发生极端负收益所造成的损失。在险价值本质是分位数,从业者通常估计5\%的VaR,因此VaR实际上是5\%的最坏的情况下最好的收益率。

\subsection{预期尾部损失}
VaR是5\%最坏的情况发生时最好的收益率。一个对损失敞口头寸更加现实的观点是:关注最坏情况发生条件下的预期损失。预期损失(expected shortfall, ES)或叫条件尾部期望(conditional tial expectation, CTE)强调了其与左尾分布之间的密切关系。计算最底部5\%的观测的平均值。

\subsection{下偏标准差与索提诺比率}
考虑到:1. 分布的非对称性要求我们独立地考察收益率为负的结果;2. 由于无风险投资工具是风险投资组合的替代投资,因此应从无风险投资收益角度考察收益的负偏离而不是从平均投资收益的角度考察。

下偏标准差(lower partial standard deviation, LPSD)的计算方式与普通标准差计算相似,但其只使用(相对无风险收益率)造成损失的样本。因此下偏标准差实际代表的是给定损失发生的情况下的均方偏离。

用LSPD代替超额收益的标准差计算的夏普比率叫作索提诺比率(Sortino ratio)。

\subsection{-3$ \sigma $收益的相对频率}
考察大幅度负收益与相同均值和标准差正态分布相比的相对发生频率。当股票价格发生大幅度变动时,称这种极端收益为jump。比较低于均值3$ \sigma $收益发生与正态分布下-33$ \sigma $收益发生的相对频率。

\section{风险组合的历史收益}

\section{长期投资}
\subsection{正态分布与对数正态分布}
一个随机变量X,如果其对数形式ln(X)服从正态分布,则X服从对数正态分布。如果瞬间股价服从正态分布,那么一个较长时间段的复利收益以及未来的股票价格服从对数正态分布。反过来如果一个股票价格服从对数正态分布,则其连续复利收益服从正态分布。

假设对数股票价格服从预期年化增长率为g、标准差为$ \sigma $的正态分布。预期年化连续复利收益率为:
\[
E(r_{cc})=m=g+\frac{1}{2}\sigma^2
\]
预期实际收益率为:
\[
E(r)=e^{g+\frac{1}{2}\sigma^2}-1
\]
若为T年投资,则预期期末财富为:
\[
E(W_T)=W_0e^{mT}=W_0e^{(g+\frac{1}{2}\sigma^2)T}
\]
累计收益率的方差与时段长度成比例:
\[
Var(r_{cc}T)=TVar(r_{cc})
\]

\subsection{再论无风险收益率}
总体原则是无风险收益率的期限应与投资期限相匹配。利率总体上和期限相关,越长的期限通胀越难预测,因此通胀风险与期限长度相关。

实际风险资产收益率应该是实际无风险收益率加风险溢价。即便是长期国债的无风险名义利率由于未来通胀和利率的不确定性也包含有风险溢价。

通胀保值债券(TIP)承诺向投资者支付一定期限的保护通胀的真实利率。这样我们可以把一定期限风险投资的预期收益率看作相同期限TIP国债的利率加风险溢价。

相同期限的国债和TIP之间的收益率之差称为远期通胀率,既包含预期值也包含风险溢价。

严格意义上,在讨论长期投资时,必须要用相应的真实无风险利率。

\subsection{长期预测}
短期的算术平均收益来预测长期累积收益将是有偏的,因为对期望收益进行估计的样本误差会在长期复利计算中产生非对称性影响,且正的误差比负的误差影响更大。

长期总收益的无偏估计要求计算所用的复利采用算术和几何平均收益率的加权值。几何平均的权重系数等于预测期的长度和样本长度的比值。

\end{document}