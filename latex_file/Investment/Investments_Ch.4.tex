\documentclass{article}
\title{共同基金与其他投资公司}
\author{Dawei Wang}
\date{\today}
\usepackage{ctex}
\usepackage{amsmath}
\usepackage{amssymb}
\begin{document}
	\maketitle
\section{投资公司}
投资公司(investment company)是一种金融中介,它从个人投资者手中汇集资金再将其广泛投资于各种有潜力的证券或资产。对于投资公司设立的资产组合,每位投资者都有与其投资数额成比例的索偿权。

投资公司的功能:

1.记账与管理。

2.分散化与分割性。

3.专业化管理。

4.较低的交易成本。

\hspace*{\fill}

把个人投资者的资产都汇集在一起的同时,投资公司也需要分配投资者对资产的所有权。投资者购买投资公司的股份,其所有权与购买股份的数量成正比。每一股份的价值被称为资产净值(net asset value,NPV)。资产净值等于资产减负债再除以发行在外的股份数量。

\section{投资公司的类型}
\subsection{单位投资信托}
基金存续期间,单位投资信托(unit investment trust)的资金都投资在一个固定的投资组合中。信托的发起人购买一个证券资产组合并将其存入信托中。之后,单位信托销售信托中的基金份额,这些份额被称为可赎回的信托凭证。

单位投资信托不需要很多的积极管理活动,一旦成立,它的资产组合的构成是固定不变的。所以这些信托被称为无管理的基金。单位信托往往投资于相对单一的资产类型。单位信托为投资者提供了一个购买资产组合中某一系列特定类型资产的工具。

单位投资信托的发起人以标的资产的成本加溢价的价格出售股份获得收益。

\subsection{投资管理公司}
投资管理公司分为两种:开放式和封闭式。这两种公司的董事会都由股东选举产生,并聘用一家管理公司对资产组合进行管理。

开放式基金(open-end fund)可以随时以资产净值赎回或者发行基金股份。开放式基金的投资者想要变现基金份额时,他们就以资产净值把股份再卖回给基金。封闭式基金(close-end fund)不能赎回或发行股份,封闭式基金的投资者想要变现必须将股份出售给其他投资者。封闭式基金的股份可以在交易所中交易,可以像其他普通股票一样通过经纪人进行买卖,因此它的价格和资产净值不一样。

尽管许多封闭式基金按资产净值折价出售,但是首次发行的基金的价格通常高于资产净值。

与封闭式基金相比开放式基金的价格不能降至资产净值以下,因为这些基金的股份随时准备以资产净值的价格被赎回。当然,如果有手续费,其报价就会高于资产净值。开放式基金不在交易所中交易,投资者仅仅通过投资公司以资产净值购买与变现其股份。

\subsection{其他投资机构}
\subsubsection{综合基金}
综合基金时汇集投资者资金的合伙制企业。由管理公司如银行或保险公司来组织管理这个合伙制企业,并收取管理费用。

综合基金形式上与开放式共同基金相似。但是综合基金发行基金单位而不是基金股份,这些基金单位以资产净值进行交易。

\subsubsection{房地产投资信托}
房地产投资信托与封闭式基金相似。房地产投资信托投资于不动产或有不动产担保的贷款。除发行股份外,它们通过银行借款、发行债券或抵押来筹集资金。产权信托直接投资于不动产,抵押信托主要投资于抵押与工程贷款。

\subsubsection{对冲基金}
和共同基金一样,汇集私人投资者的资产并由基金管理公司负责投资。但是对冲基金通常以私人合伙形式存在,也几乎不受证监会监管。一般只对富有的投资者和机构投资者开放。许多对冲基金要求投资者同意一开始就锁定,也就是说,在长达几年的投资期中,投资者不能收回投资。锁定允许对冲基金投资于流动性不强的资产而无需考虑基金赎回的要求。由于监管很松,对冲基金的管理者可以使用一些共同基金管理人不能使用的投资策略,比如大量使用衍生工具、卖空交易和财务杠杆。

\section{共同基金}
开放式投资公司通常被称为共同基金。

\subsection{投资策略}
根据投资策略的不同,基金通常可以分成以下几种类型。

\subsubsection{货币市场基金}
货币市场基金投资于货币市场证券

\subsubsection{股权基金}
主要投资与股票,资产组合管理人出于谨慎也可能持有固定收益证券或其他类型证券。基金通常会将4\%~5\%的总资产投资于货币市场证券以满足潜在的份额赎回时的流动性需求。

\subsubsection{行业基金}
一些股权基金也被称为行业基金,它们专门投资于某个特定行业。

\subsubsection{债券基金}
专门投资于固收。

\subsubsection{国际基金}
致力于国际市场,在全球范围内投资的基金。

\subsubsection{平衡型基金}
一些基金被设计成备选对象,供个人投资者投资整个资产组合时选择使用。这些平衡型基金以相对稳定的比例持有权益和规定收益两类证券。

\subsubsection{资产灵活配置型基金}
与平衡基金相似,都包含股票和债券。这些基金强调市场时机的选择,不是低风险投资工具类。

\subsubsection{指数基金}
指数基金试图跟踪某个主板市场指数的业绩。这种基金购买某个指数中的证券,所购的份额与该证券在指数中所占比例相一致。

\subsection{如何出售基金}
共同基金通常有两种公开发售方式:一是通过基金承销商直接交易;二是通过代表销售商的经纪人间接交易。

\section{共同基金的投资成本}
\subsection{费用结构}
\subsubsection{运营费用}
运营费用是共同基金在管理资产组合时所发生的成本,包括支付给投资管理人的管理费用和咨询费用。这些费用定期从基金资产中扣除。
\subsubsection{前端费用}
前端费用是当购买股份时需要支付的佣金或销售费用,这些费用主要支付给出售基金的经纪人
\subsubsection{撤离费用}
撤离费用是出售基金份额时的赎回或撤出费用。
\subsubsection{12b-1费用}
12b-1费用可以用来替代前端费用来支付经纪人。与基金的运营费用一样,投资者不必直接缴纳12b-1费用,费用是从基金资产中扣除的。
\subsection{费用与共同基金的收益}
收益率=($ NAV_1-NAV_0 $+收入和资本利得的分配)/$ NAV_0 $

收益率的这种测度方法忽略了所有的佣金。

基金的收益率等于标的资产组合的总收益率减去总费用率。
\section{共同基金所得税}
换手率是投资组合的交易量与其资产的比率。高换手率意味着资本利得和损失的出现。因此,投资者无法确定变现时间,也无法统筹管理税赋。
\section{交易所交易基金}
交易所交易基金(exchange-traded fund,ETF)是共同基金的一个分支,它使投资者可以像交易股票一样运作指数投资组合。
\section{共同基金投资业绩:初步探讨}
共同基金的一个好处是能够将投资组合委托给专业投资人管理。投资者通过资产配置决策保持对整个资产组合主要特征的控制权,每个投资者选择投资到债券基金、股权基金和货币市场基金的比例,但是可以把每个投资组合中特定的证券选择权留给每个基金的管理人。
\section{共同基金信息}

\end{document}