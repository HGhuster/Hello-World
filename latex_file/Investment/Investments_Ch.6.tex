\documentclass{article}
\title{风险资产配置}
\author{Dawei Wang}
\date{\today}
\usepackage{ctex}
\usepackage{amsmath}
\usepackage{amssymb}
\begin{document}
	\maketitle
构造一个投资组合分为两步:1. 投资者确定组合中风险资产的构成;2. 决定这个风险资产组合和无风险资产的配置比率。 
\section{风险与风险厌恶}
\subsection{风险、投机和赌博}
投机是指承担一定的风险并获取相应的报酬。投机中的风险溢价必须为正。

赌博是为了一个不确定的结果下注。赌博与投机的差异主要在于赌博并没有“相应的报酬”。从经济学上讲,赌博是为了享受冒险的乐趣而承担风险,而投机是为了风险溢价而承担风险。把赌博变成投机需要有足够的风险溢价来补偿风险厌恶投资者。风险溢价为零的风险投资叫作公平博弈(fair game)。

由于主观预期不同,公平博弈双方可能都认为彼此在投机,这种现象叫作异质预期。解决异质预期的方案是博弈双方充分交换信息,但信息的交换是有成本的,因此一定程度上异质预期的存在并非不理性。

\subsection{风险厌恶和效用价值}
风险厌恶(risk averse)的投资者会放弃公平赌局或更差的投资。他们只考虑无风险资产和具有正风险溢价的投资选项。

关于风险厌恶者,金融学应用最多的效用函数是:
\[
U=E(r)-\frac{1}{2}A\sigma^2
\]
其中U为效用值,A为投资者的风险厌恶系数。收益率必须采用小数形式。

\hspace*{\fill}

无风险资产的效用值就是其自身的收益率,因为其风险补偿为0。

可以把风险资产的效用值看作确定等价收益率,即无风险资产为达到与风险资产相同的效用值所需要的收益率。

只有当一个资产组合的确定等价收益率超过无风险资产收益率时,该风险资产组合才是值得投资的。

\hspace*{\fill}

风险厌恶(risk averse):A>0;

风险中性(risk neutral):A=0;

风险偏好(risk lover):A<0;

\hspace*{\fill}

均值-方差准则(mean-variance criterion, M-V)可以表述为:若投资组合A优于投资组合B,则$ E(r_A)\ge E(r_B) $与$ \sigma_A\le \sigma_B $至少有一个严格成立。

\subsection{估计风险厌恶系数}

\section{风险资产与无风险资产组合的资本配置}
把投资者的风险资产组合用P表示,无风险资产组合用F表示。

风险投资组合P在完整资产组合中的比重记为y,E为股权资产占整个投资组合的比例,B为债券资产占整个资产组合的比例。
\[
E+B=y
\]

\section{无风险资产}
现实中唯一的无违约风险资产是一种理想的指数化债券(TIP)。且唯有在其期限等于投资者愿意持有的期限时,才能对投资者的实际收益率进行担保。

实际中把短期国债看作无风险资产。

实际操作中大多数投资者应用货币市场工具作为无风险资产。可以把货币市场基金看作大多数投资者最易接触到的无风险资产。

\section{单一风险资产与单一无风险资产的投资组合}
风险投资组合的投资比例为y,无风险投资组合的比例为1-y,整个组合C的收益率$ r_C $为:
\[
r_C=yr_P+(1-y)r_f
\]
取期望值,得:
\[
E(r_C)=yE(r_P)+(1-y)r_f=r_f+y[E(r_P)-r_f]
\]

整个组合的标准差:
\[
\sigma_C=y\sigma_P
\]

资本配置线(capital allocation line, CAL),表示对投资者而言所有可能的风险收益组合。纵轴截距为$ r_f $,斜率为夏普比率:
\[
S=\frac{E(r_P)-r_f}{\sigma_P}
\]
资本配置线表示对投资者而言所有可能的风险收益组合。

\section{风险容忍度与资产配置}
效用最大化:
\[
\max_y U=E(r_C)-\frac{1}{2}A\sigma^2_C=r_f+y[E(r_P)-r_f]-\frac{1}{2}Ay^2\sigma^2_P
\]
解得风险厌恶者风险资产的最优头寸:
\[
y^*=\frac{E(r_P)-r_f}{A\sigma^2_P}
\]

\section{被动策略:资本市场线}
资本配置线CAL由无风险资产和风险投资组合P导出,风险投资组合P源于被动策略或积极策略。

积极策略不是免费的,被动策略的成本只有短期国债所需少量的佣金和支付给共同基金等市场指数基金和证券交易所的管理费用。

一个被动型投资策略牵涉两个投资组合:接近于无风险的短期国债和一个跟踪大盘指数的股票基金。


\end{document}