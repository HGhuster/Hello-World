\documentclass{article}
\title{证券是如何交易的}
\author{Dawei Wang}
\date{\today}
\usepackage{ctex}
\usepackage{amsmath}
\usepackage{amssymb}
\begin{document}
	\maketitle
\section{公司如何发行证券}
公司一般可以通过发行债券或股票为投资项目筹措资金。新发行的证券由投资银行推销给公众,该市场叫做一级市场(primary market)。投资者交易已发行证券的市场叫做二级市场(secondary market)。

\subsection{私人控股公司}
私人控股公司的股东相对少,可以免于定期公开披露财务报告信息和其他信息。

私人控股公司最多只能有499个股东。私人控股公司可以通过直接向少数机构投资者或高净值投资人群销售证券来募集必要的资金,这种方式叫私募(private placement)。通过私募方式发行的证券不能在证券交易所等二级市场公开交易,证券流动性大幅降低,投资者为此支付的价格也相对较低。

\subsection{上市公司}
上市公司可以把证券销售给公众,而投资者可以在证券市场中自由地交易其股票。首次公开发行股票给公众被称为公司的首次公开发行 (initial public offering,IPO)。之后,公司可以回到公开市场继续发行额外的股票。增发(seasoned equity offering) 是指上市公司再次发行股票的行为。

通常情况下,投资银行在股票和债券的公开发行中扮演了承销商(underwriter) 的角色。负责推销政权的投资银行通常不止一家,而是以其中一家为主承销商、其他多家投资银行辅助承销的辛迪加组织来分担股票发行任务。

投资银行会为拟上市公司提供证券发行辅导。公司必须首先向SEC提交初步注册说明(初步募股说明书), 以说明发行事宜以及公司前景。得到SEC批准后,说明书终稿被称为募股说明书(prospectus),进行到这一步后,证券的发行价格就可以公布。

包销(firm commitment)是投资银行承销证券时普遍采用的方式。投资银行先从发行公司那里购买证券,然后再把这些证券销售给公众。证券买卖差价为承销商的佣金,除了买卖差价外,投行还可能获得发行公司的普通股或其他证券。

\subsection{暂搁注册}
SEC允许公司在证券首次注册后的两年内逐期内向公众销售债券。由于债券发行手续已事先办理,一旦公司产生实际的融资需求,只需简单的资料更新即可发行证券,几乎无须额外的文书工作。此外,证券可零散发行,无须大量的发行成本。

\subsection{首次公开发行}
投资银行负责将新证券发售给公众投资者,一旦SEC对申请做出回复且初步募股说明书分发给有兴趣的投资者,投资银行便会组织路演(road shows),在全美巡回宣传即将发行的证券。承销商可借路演机会与投资者沟通发行证券的相关信息,激发潜在投资者的认购兴趣。机构投资者也会向承销商表达其购买IPO证券的兴趣,这种兴趣的暗示过程叫做“预约”,赢得潜在投资者的过程叫作“建立投资者购股意愿档案”。

\section{证券如何交易}
\subsection{市场的类型}
可以把市场分为4种类型:直接搜寻市场、经纪人市场、交易商市场和拍卖市场。

直接搜寻市场(direct search market) 组织性最差,买卖双方需要耗费精力搜寻交易对手。特点是交易频次低、商品非标准化且低值。

经纪人市场(brokered market) 组织性优于直接搜寻市场的是经纪人市场。在交易活跃的市场中,经纪人发现为买方和卖方提供信息并撮合交易有利可图。经纪人可为市场参与者提供信息搜寻服务赚取佣金。特定市场上的经纪人逐渐积累了对该市场中交易资产进行估值的专业知识。(IPO中投行就扮演了经纪人的角色。)

交易商市场(dealer market) 当某类特定资产的交易活动日益频繁时,交易商市场便诞生了。专于某类资产的交易商用自己的账户买入这类资产,再择机将其售出,在低买高卖中赚取利润。交易商市场为交易者节省了信息搜寻成本。足够的市场交易量才能保证交易商有利可图。大多数债券都在场外交易商市场交易。

拍卖市场(auction market) 拍卖市场时组织性最强的市场,所有交易者聚集在同一场所进行交易。拍卖市场的一个显著优势就是无须在交易商中寻找最优的交易报价。如果所有参与者聚集在一起,他们便可以在价格上达成一致,从而降低买卖差价。

\hspace*{\fill}

场外交易商市场和股票交易所同属二级市场,投资者在这些市场中可彼此买卖已发行的证券。

\subsection{交易指令的类型}
交易指令分为两种类型:市价委托指令和限价指令。

市价委托指令(market order) 市价委托指令是按当前市场价格立即执行买入或卖出指令。

注:

1. 报价实际上只代表了对一定数量股票交易的承诺,如果市价委托指令要求的交易数量超过了这个数量,该指令可能需要按照多种不同的价格才能成交。市场深度(在最好的买价和卖价上总的可供交易的股票数量) 被认为是流动性的另一个组成部分。

2. 其他交易者的报价指令被优先执行,这意味着投资人可能要按更不利的价格才能成交。

3. 在指令到达之前最佳报价已发生变化,同样可能导致成交价格与发出指令时的市价不同。

\hspace*{\fill}

限价指令(price-contingent order)

投资者可以严格设定他们希望买卖证券的价格。限价买入指令嘱托经纪人在股价降到或者低于约定价格时买入某一数量的股票。相反,限价卖出指令要求经纪人在股价升到或高于约定价格时卖出一定数量的股票。等待成交的限价指令的集合叫作限价指令簿。

最高买方报价和最低卖方报价叫作内侧报价(inside quote),内测报价指令的订单数量通常很少,因此,大单交易者通常需要面对大的价差,他们无法按照内侧报价完成全部交易。

止损指令(stop orders),类似于限价指令,只有当股票价格触及价格界限时才会被执行。对卖出止损指令而言,只有股价跌到指定价格时才会卖出股票。买入止损指令则要求股价上涨到指定价格时购入股票。买入止损通常伴随着做空。

\subsection{交易机制}
全美证券产品的交易主要依赖于场外交易商市场、电子交易所和专家做市市场。

交易商市场(dealer market) 约有35 000中证券在场外交易市场(over-the-counter, OTC)进行交易。成千上万的经纪人在证卷交易委员会注册成为券商。券商根据自身意愿报出其所期望的证券买卖价格,经纪人联系报价最具有吸引力的券商执行交易。

\hspace*{\fill}

电子交易所(electronic communication network,ECN)电子交易所允许参与者通过计算机网络发出市场委托指令和限价指令,所有参与者都可以查看限价订单簿。

电子交易所的吸引力在于:首先,直接撮合交易而不用经纪人-券商机制,消除了买卖价差;由于电子交易所可以自动撮合交易,因此交易成本可以达到最低;电子交易执行速度快,而且投资者在交易过程中可以完全是匿名的。

\hspace*{\fill}

专家做市市场(specialist market)每种证券都有一个专家做市商(specialist)来负责,任何以客户的名义买卖证券的经纪人必须把交易提交到专家做市商所在的交易所的平台。每种证券都被分配给一个专家做市商,但是每家专家做市商公司可以负责多种证券。

\section{电子交易的繁荣}
NASDAQ开始建立的时候主要时场外交易市场,NYSE时专家做市市场。但是今天两者都是主要的电子交易所。科技帮助交易者比较所有市场的价格,交易者可以选择拥有最好价格的市场进行交易。

尽管专家做市商依然存在,但对于股票而言,还是更多地采用电子化交易。债券依然还在传统的交易商市场中交易。

\section{美国证券市场}
\subsection{纳斯达克}
NASDAQ的会员分为三个级别。

证券做市商为三级会员(最高级别)。它们通常为金融机构,持有某种证券,并根据自身意愿提供买卖报价。三级会员可以实时将最新的买卖报价输入网络加以平台,系统保证其能够最快速地被执行。买卖差价成为其主要的利润来源。

二级会员为经济公司,它们执行客户的指令,但不会主动交易自己账户中的股票。其可以查询所有报价,但不能输入自己的报价。买卖证券的经纪人与最佳报价的做市商(三级会员)进行交易。

一级会员只能收到内侧报价,但看不到报价的股票数量。

\subsection{纽约证券交易所}
NYSE一直保持它的专家做市商交易系统,在交易执行中这一系统很依赖与人的参与。

\subsection{电子交易市场}
随着电子交易市场的发展,一些交易者可以快速浏览许多电子交易市场的限时指令,并可以马上把这些指令转移到更好价格的市场中去。这种跨市场链接已经成为所谓高频交易者战略的一个很重要的推动因素,高频交易者从不同市场中特别小的转移价差中赚取利润。显然速度在这里至关重要,电子交易场就它们可以提供的速度进行竞争。等待时间(latency)特质一个交易接受、处理、传送所需花费的时间。

\section{新的交易策略}
高频交易是一类特殊的算法交易,在证券市场上很多曾经由做市经纪人提供的市场流动性已经被高频交易者所取代,但是若高频交易者放弃市场,流动性同样可以在一瞬间蒸发。暗池交易保持匿名,也影响了市场流动性。

\subsection{算法交易}
算法交易(algorithmic trading)是利用电脑程序来做出交易决定。

一些算法交易涉及的活动类似于传统的做市活动。交易者试图通过买卖价差盈利,在价格变化之前以买价买入然后再快速以卖价卖出。虽然这模仿了在股市中提供流动性给其他交易者的做市商,但这些算法交易者不是注册做市商,所以他们没有义务维护买入和卖出报价。如果他们在一段动荡期间放弃市场,市场流动性可能会遭到极大破坏。

\subsection{高频交易}
高频交易(high-frequency trading)是算法交易的一个子集,它依靠计算机程序做出极其快速的决策。高频交易员具有非常小的利润的交易,但交易数量多,因此获利多。

高频交易者交易执行时间以微秒计算,这导致交易公司把它们的交易中心建在电子交易所的电脑系统旁边。

\subsection{资金暗池}
许多大额交易商寻求匿名。他们担心被别人知道他们在执行大型的买入或出售计划,价格会向不利于他们的方向变动。大宗交易(blocks,通常被定义为超过10 000股的交易)通常会被带到特定地点,经纪人专门匹配大宗交易的买家和卖家。

今日的大宗交易很大程度上已经被暗池取代,交易系统中的参与者可以掩盖他们的身份进行大额证券买卖。不仅是在暗池的买家和卖家从公众中隐藏,交易也可能不进行报告,或者交易信息被模糊。

处理大量交易的另一种方法是将它们分割成许多小的交易,每个都可以在电子交易市场被执行。

\subsection{债券交易}
为了使尽可能多的债券上市,NYSE将其电子债券交易平台进行了延申,先被称为纽约证券交易所债券系统,它是美国所有交易所中最大的集中化债券市场。

债券交易商之间的债券大多发生在场外交易市场。该市场通过一个计算机报价系统把债券交易商联系起来的网络。

\section{股票市场的全球化}

\section{交易成本}
证券交易的部分成本是显性的,如投资者向经纪人支付的佣金。

个人投资者可以选择两类经纪人,综合服务经纪人(财务顾问)和折扣经纪人。

许多客户完全信任经纪人,他们通过设定一个授权账户将买卖决策全权委托给综合服务经纪人。在授权账户中,经纪人可以在其认为合理的情况下买入或卖出指定的证券,但他们不能提取任何资金。

相反,折扣经纪人不提供没有必要的服务,他们只负责买卖证券、保管证券、提供保证金贷款和促成卖空。他们为其管理的证券提供的唯一信息是报价信息。

除了经纪人佣金这项显性交易成本,还包括隐形交易成本,即买卖差价。经纪人有时也是一名交易商,他们不直接索取佣金,而是从买卖差价中赚取服务费用。还有一种隐性成本,即当投资者希望的交易量超过与报价对应的交易量时,其必须做出价格让步。

\section{保证金交易}
以保证金交易购买意味着投资者需要从经纪人那里接入部分购买股票的资金。账户中的保证金(margin)是由投资者提供的购买款项,剩下的购买款项是向经纪人借入的。经纪人以活期贷款利率从银行借入资金以支持这些购买,再按这个利率向投资者收取贷款利息和服务费用。所有以保证金购买的证券必须由经济公司保管,因为这些证券都被视为贷款的抵押物。

\hspace*{\fill}

保证金=权益价值/股票市值

\hspace*{\fill}

维持保证金:经纪人设定的一个保证金比例,若保证金比例下降到维持保证金比例之下,经纪人就会发出保证金催缴通知,要求投资者向保证金账户增加现金或证券,若投资者不履行要求,经纪人就会从账户中出售证券并偿还贷款,以使保证金恢复到一个可接受的水平。

\section{卖空}
卖空(short sale)允许投资者从证券价格下跌中获利。投资者首先从经纪人处借入股票并将其卖出,然后再买入等量相同的股票偿还借入的股票。

实际操作中,通常由卖空者的经纪公司向卖空者借出股票,经纪公司持有其他投资者的各种股票,股票所有者无须知道股票被借给卖空者。如果股票所有者希望出售股票,经纪公司从其他投资者处借入股票即可。但若经纪公司无法找到新的股票填补已售出的股票,卖空者须马上平掉空仓。

交易所规定卖空的收益必须保留在经纪人账户中,卖空者不能利用这笔资金再投资。卖空期间,卖空者必须向经纪人追加保证金以填补股价上涨带来的损失。


保证金比例=权益价值/所欠股票价值。

\section{证券监管}
SEC对募股说明书或财务报告的批准并不能说明该公司的证券是一项好投资,SEC仅关心相关信息是否如实披露。
\subsection{自律}
最主要的自律机构是金融行业监管局(Financial Industry Regulatory Authority,FINRA),它是美国最大的监管证券公司的非政府监管机构。

\end{document}