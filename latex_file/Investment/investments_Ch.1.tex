\documentclass{article}
\title{投资环境}
\author{Dawei Wang}
\date{\today}
\usepackage{ctex}
\usepackage{amsmath}
\usepackage{amssymb}
\begin{document}
	\maketitle
\section{实物资产与金融资产}
一个社会的物质财富最终取决于该社会经济的生产能力,即社会成员创造产品和服务的能力。这种生产能力是经济体中实物资产(real assets)的函数,如土地、建筑物、机器以及可用于生产产品和提供服务的知识。(在资产负债表的资产侧)

与实务资产对应的是金融资产(financial assets),如股票和债券。这些证券(金融资产),代表了对实物资产所产生收入的索取权。(在资产负债表的负债与所有者权益侧)

实物资产创造利润,金融资产确定收益在投资者之间的分配。金融产品的收益(亏损)还是取决于实物资产的表现。

\section{金融资产}
金融资产通常可以分为三类:固定收益型、权益型、衍生品。

固定收益型金融资产(fixed-income),即债券(debt securities),承诺支付一系列固定的,或按某一特定公式计算的现金流。

货币市场中交易的债券:期限短、流动性强、风险小。

资本市场交易债券:期限长、不同品种债券风险和流动性差异很大。

\hspace*{\fill}

权益型金融资产代表了证券持有者对公司的所有权。权益性证券持有者未被承诺任何的特定收益,但他们可以获得公司分配的股利,并按相应的比例拥有对公司实物资产的所有权。

\hspace*{\fill}

衍生证券的收益取决于其他资产的价格。

\section{金融市场与经济}
\subsection{金融市场信息作用}
股价是投资者对公司当前业绩和未来前景综合评价的反映。股票市场不过是在适当的时间鼓励资本流向那些正处于风口的企业。那些训练有素、聪明、勤奋、专业人士会对这一风口做出独立的判断。股价是各方判断的集中体现。

\subsection{消费时机}
金融市场可以使人们的现实消费与现实收入相分离。

\subsection{风险分配}
金融市场和金融市场上交易的各种金融工具可以使偏好风险的投资者承担风险,使厌恶风险的投资者规避风险。当投资者可以选择满足自身特定风险-收益偏好的证券时,每种证券都可以以最合适的价格出售,这加速了实物资产证券化的进程。

\subsection{所有权和经营权分离}
代理问题解决机制:

1.将管理层的薪酬与公司经营业绩挂钩。

2.由董事会解雇那些表现不好的管理者。

3.由外部证券分析者和大型机构投资者密切监督公司,使那些业绩差的管理者日子不好过。

4.糟糕的业绩将使先由董事会丧失对公司的控制权。股东发起代理权之争,或公司被另一家公司收购。

\section{投资过程}
投资者在构建投资组合时,需要做出两类决策:资产配置决策和证券选择决策。资产配置决策是指投资者对这些资产大类的选择,证券选择决策是指在每一类资产大类中选择特定的证券。

资产配置决策包括对安全资产和风险资产投资比例的决策。"自上而下"的投资组合构建方法时从资产配置开始的。

证券分析包括对可能包含在投资组合中的特定证券进行估值。

"自下而上"的投资组合构建方法是通过选择那些具有价格吸引力的证券完成的,不需要过多地考虑资产配置。"自下而上"的方法使投资组合集中在那些最具投资吸引力的资产上。

\section{市场是竞争的}
\subsection{风险收益的权衡}
证券市场中的风险-收益权衡,高风险资产的期望收益率高于低风险资产的期望收益率。

\subsection{有效市场}

\section{市场参与者}
1.公司。公司是净借款人,它们筹集资金并将其投资于厂房和设备,这些实物资产所产生的收益用于向投资者支付回报。

2.家庭。家庭是净储蓄者,它们购买那些需要筹集资金的公司所发行的证券。

3.政府。政府可能是借款人也可能是投资者,取决于税收和政府支出之间的关系。

公司和政府不会将其全部或大部分证券直接出售给个人。金融中介在个人和公司/政府之间起媒介作用。

\subsection{金融中介}
大多数家庭的财务资产规模过小,直接投资很困难。

第一,个人的投资渠道有限;

第二,个人投资者不可能通过多样化借款人降低风险;

最后,个人投资者没有能力评估并监督借款人的信用风险。

金融中介由是发展起来,成为联系借款人和投资者的桥梁。它们通过发行证券筹集资金以购买其他公司发行的证券。

金融中介区别于其他商业机构的主要特点在于其资产和负债大多数是金融性的。

金融机构有以下优点:

第一,通过聚集小投资者的资金可以为大客户提供贷款;

第二,通过向众多客户贷款可以分散风险,因此可以提供单笔风险很高的贷款;

第三,通过大量业务来储备专业知识,并可以利用规模经济和范围经济来评估、监控风险。

\subsection{投资银行}
公司大部分资金是通过向公众发行证券来筹集的,但这么做的频率不高,专门从事此类业务的投资银行可以以低成本向公司提供这项业务。在这个过程中,投资银行被称为承销商。

投资银行在证券发行价格、利率等方面为公司提供建议。最后再由投资银行负责在一级市场上销售,随后投资者可以在二级市场买卖一级市场发行的证券。

\subsection{风险投资与私募股权}
初创公司唯有依赖银行贷款或吸引那些愿意获取该公司所有权份额的股权投资者。投资与早期阶段的股权投资称为风险投资。风险投资来源于专门的风险投资基金、天使投资人或养老基金等机构。

大多数风险投资基金采取有限合伙的组织形式。基金管理人以自有资金和从其他有限合伙人处募集的资金投资。

总体来说,聚焦于非上市公司的股权投资通常被成为私募股权投资。




\end{document}