\documentclass{article}
\title{baby rudin Ch.1}
\author{Dawei Wang}
\date{\today}
%\usepackage{ctex}
\usepackage{amsmath}
\usepackage{amssymb}
\usepackage{theorem}
\newtheorem{defi}{Definition}
\newtheorem{theo}{Theorem}
\begin{document}
	\maketitle	
\section{INTRODUCTION}
The rational number system has certain gaps, in spite of the fact that between any two rationals there is another: If $r<s$ the $r<(r+s)/2<s$. The real number system fills these gaps.

\hspace*{\fill}

\begin{defi}
	If A is any set, we write $x\in A$ to indicate that x is a member(or an element) of A.
\end{defi}

If x is not a member of A, we write: $x\notin A$.

If A and B are sets, and if every element of A is an element of B, we say that A is a subset of B, and write $A\subset B$, or $B\supset A$. If in addition there is an element of B which is not in A, then A is said to be proper subset of B.

If $A\subset B$	and $B\subset A$, we write A=B. Otherwise $A\neq B$.

\section{ORDERED SETS}
\begin{defi}
	Let S be a set. An order on S is a relation, denoted by $<$, with the  following two properties:
\end{defi}

(i) If $ x\in S $ then one and only one of the statements
\[
x<y,\qquad x=y,\qquad x>y
\]
is true.

(ii) If $ x,y,z\in S $, if $x<y$ and $y<z$, then $x<z$.

$ x\le y $ is the negation of $x>y$.

\hspace*{\fill}

\begin{defi}
	 Suppose S is an ordered set, and $ E\subset S  $. If there exists a $ \beta\in S $ such that $ x\le \beta $ for every $ x\in E $, we say that E is bounded above, and call $ \beta $ an upper bound of E.
\end{defi}

\begin{defi}
	Suppose S is an ordered set, $ E\subset S $, and E is bounded above. Suppose there exists an $ \alpha\in S $ with the following properties:
\end{defi}

(i) $ \alpha $ is an upper bound of E.

(ii) If$ \gamma<\alpha $ then $ \gamma $ is not an upper bound of E.

Then $ \alpha $ is called the least upper bound of E or the supremum of E, and we write
\[
\alpha= sup\quad E
\]
The greatest lower bound, or infimum, of a set E which is bounded below is defined in the same manner: The statement
\[
\alpha= inf\quad E
\]
means that $ \alpha $ is a lower bound of E and that no $ \beta $ with $ \beta>\alpha $ is a lower bound of E.

\hspace*{\fill}

\begin{defi}
	 An ordered set S is said to have the least-upper-bound property if the following is true:
\end{defi}

If $ E\subset S $, E is not empty, and E is bounded above, then sup E exists in S.

\hspace*{\fill}

Every ordered set with the least-upper-bound property also has the greatest-lower-bound property.

Q(the set of all rational numbers) does not have the least-upper-bound property.

\begin{theo}
	Suppose S is an ordered set with the least-upper-bound property, $ B\subset S $, B is not empty, and B is bounded below. Let L be the set of all lower bounds of B. Then
\end{theo}
\[
\alpha=sup\quad L
\]
exists in S, and $ \alpha=inf\quad B $

In particular, inf B exists in S.

\section{FIELDS}
\begin{defi}
	A field is a set F with two operations, called addition and multiplication, which satisfy the following so-called "field axioms":
\end{defi}
(A) Axioms for addition

(A1) If $ x\in F $ and $ y\in F $, then their sum x+y is in F.

(A2) Addition is commutative: x+y=y+xfor all $ x,y\in F $.

(A3) Addition is associative: (x+y)+z=x+(y+z) for all $ x,y,z\in F $.

(A4) F contains an element 0 such that 0+x=x for every $ x\in F $.

(A5) To every $ x\in F $ corresponds an element $ -x\in F $
such that
\[
x+(-x)=0
\]

\hspace*{\fill}

(M) Axioms for multiplication

(M1) If$ x\in F $ and$ y\in F $, then their product xy is in F.

(M2) Multiplication is commutative: xy=yx for all $ x,y\in F $.

(M3) Multiplication is associative: (xy)z=x(yz) for all $ x,y,z\in F $.

(M4) F contains an element $ 1\ne 0 $ such that 1x=x for every $ x\in F $.

(M5) If $ x\in F $ and $ x\ne 0 $ then there exists an element $ 1/x\in F $ such that
\[
x\cdot(1/x)=1
\]

\hspace*{\fill}

The distributive law
\[x(y+z)=xy+xz\]
holds for all $ x,y,z\in F $.

\hspace*{\fill}

The filed axioms clearly hold in Q, the set of all rational numbers, if addition and multiplication have their customary meaning. Thus Q is a field.

\begin{defi}
	An ordered field is a field is a field F which is also an ordered set, such that:

\end{defi}

(i)$ x+y<x+z$ if $ x,y,z\in F $ and $y<z$,

(ii) $xy>0$ if $ x\in F, y\in F $, $x>0$, and $y>0$.

If $x>0$, we call x positive; if $x<0$, x is negative.

\section{THE REAL FIELD}
\begin{theo}
	There exists an ordered field R which has the least-upper-bound property.

	Moreover, R contains Q as a subfield.
\end{theo}

\begin{theo}
	(a) If $ x\in R, y\in R $, and $x>0$ , then there is a positive integer n such that(Archimedean property)
	\[
	nx>y
	\]

	(b) If $ x\in R,y\in R $, and $x<y$, then there exists a $ p\in Q $ such that $ x<p<y $.
\end{theo}

This means that Q is dense in R: Between ant two real numbers there is a rational number.

\begin{theo}
	For every real $x>0$ and every integer $n>0$ there is one and only one positive real y such that $ y^n=x $.

\end{theo}

This number y is written $ \sqrt[n]{x} $ or $ x^{1/n} $.

\section{The extended real number system}
\begin{defi}
	The extended real number system consists of the real field R and two symbols, $+\infty$ and $-\infty$. We preserve the original order in R, and define
	\[
	-\infty<x<+\infty
	\]
	for every $x\in R$.
\end{defi}

It is then clear that $+\infty$ is an upper bound of every subset of the extended real number system, and that every nonempty subset has a least upper bound.

The extended real number system does not form a field.

\section{The complex field}
\begin{defi}
	A complex number is an ordered pair (a,b) of real numbers. "Ordered" means that (a,b) and (b,a) are regarded as distinct if $a\neq b$.
\end{defi}

Let x=(a,b),y=(c,d) be two complex numbers. We write x=y if and only if a=c and b=d. We define
\[
x+y=(a+c,b+d)
\]
\[
xy=(ac-bd,ad+bc)
\]

\begin{theo}
	Theses definitons of addition and multiplication turns the set of all complex numbers into a field, with (0,0) and (1,0) in the role of 0 and 1.
\end{theo}

\begin{theo}
	For any real numbers a and b we have
\[
	(a,0)+(b,0)=(a+b,0)
\]
\[
	(a,0)(b,0)=(ab,0)
\]
\end{theo}

the real field is a subfield of the complex field. The notation (a,b) is equivalent to the more customary a+bi;

\begin{defi}
	i=(0,1)
\end{defi}

\begin{theo}
	$i^2=-1$
\end{theo}

\begin{theo}
	If a and b are real, then (a,b)=a+bi
\end{theo}

\begin{defi}
	If a, b are real and z=a+bi, then the complex number $\overline{z}=a-bi$ is called the conjugate of z. The numbers a and b are the real part and the imaginary part of z, respectively.
\end{defi}

We shall occasionally write a=Re(z), b=Im(z).

\begin{theo}
	If z and w are complex, then
	
$ \overline{z+w}=\overline{z}+\overline{w} $

$ \overline{zw}=\overline{z}\cdot\overline{w} $

$ z+\overline{z}=2Re(z), z-\overline{z}=2iIm(z) $

$ z\overline{z} $ is real and positive (except when z=0).
\end{theo}

\begin{defi}
	if z is a complex number, its absolute value $ |z| $ is the non-negative square root of $ z\overline{z} $; that is, $ |z|=(z\overline{z})^{1/2} $.
\end{defi}

\begin{theo}
	Let z and w be complex numbers. Then

$ |z|>0 unless z=0, |0|=0, $

$ |\overline{z}|=|z|, $

$ |zw|=|z||w|, $

$ |Rez|\le |z|, $

$ |z+w|\le|z|+|w|. $

\end{theo}

\begin{theo}
	If $ a_1,\cdots,a_n  $ and $ b_1,\cdots,b_n $ are complex numbers, then
	\[
	|\sum_{j=1}^{n}a_j\overline{b}_j|^2\le \sum_{j=1}^{n} |a_j|^2\sum_{j=1}^{n}|b_j|^2
	\]
\end{theo}

\section{Euclidean spaces}

\begin{defi}
	For each positive integer k, let $ R^k $ be the set of all ordered k-tuples
	\[
	\textbf{x}=(x_1,x_2,\cdots x_k),
	\]
	where $ x_1,\cdots ,x_k $ are real numbers, called the coorinates of \textbf{x}. The elements of $ R^k $ are called points. or vectors, especially when $ k>1 $. If $ \textbf{y}=(y_1,y_2,\cdots y_k) $ and if $ \alpha $ is a real number, put
	\[
	\textbf{x}+\textbf{y}=(x_1+y_1,\cdots,x_k+y_k),
	\]
	\[
	\alpha\textbf{x}=(\alpha x_1,\cdots,\alpha x_k)
	\]
\end{defi}

The zero element of $ R^k $ is the point \textbf{0}, all of whose coordinates are 0.

We also define the so-called "inner product"(or scalar product) of \textbf{x} and \textbf{y} by

\[
\textbf{x}\cdot \textbf{y}=\sum_{i=1}^{k}x_iy_i
\]

and the norm of \textbf{x} by

\[
|\textbf{x}|=(\textbf{x}\cdot \textbf{x})^{1/2}=(\sum_{1}^{k}x_i^2)^{1/2}
\]

The structure now defined is called euclidean k-space.

\begin{theo}
	Suppose $ x,y,z\in R^k $, and a is real. Then
	
(a)	$ |\textbf{x}|\ge 0; $
	
(b)	$ |\textbf{x}|=0\enspace if\enspace and\enspace only\enspace if\enspace \textbf{x}=\textbf{0}; $
	
(c)	$ |\alpha\textbf{x}|=|\alpha||\textbf{x}|; $
	
(d)	$ |\textbf{x}\cdot\textbf{y}| \le |\textbf{x}||\textbf{y}|;  $
	
(e)	$ |\textbf{x}+\textbf{y}|\le |\textbf{x}|+|\textbf{y}|; $
	
(f)	$ |\textbf{x}-\textbf{z}|\le |\textbf{x}-\textbf{y}|+|\textbf{y}-\textbf{z}|. $
	
\end{theo}

Theorem (a), (b), (c) and (f) will allow us to regard $ R^k $ as a metric space. vc

\end{document}
