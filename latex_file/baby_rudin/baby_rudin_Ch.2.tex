\documentclass{article}
\title{baby rudin Ch.2}
\author{Dawei Wang}
\date{\today}
\usepackage{ctex}
\usepackage{amsmath}
\usepackage{amssymb}
\usepackage{theorem}
\newtheorem{defi}{Definition}
\newtheorem{theo}{Theorem}
\newtheorem{rem}{Remark}
\newtheorem{coro}{Corollary}
\begin{document}
	\maketitle
\section{FINITE, COUNTABLE, AND UNCOUNTABLE SETS}
\begin{defi}
	Consider two sets A and B, whose elements may be any objects whatever, and suppose that with each element x of A there is associated, in some manner, an element of B, which we denote by f(x). Then f is said to be a function from A to B(or a mapping of A into B). The set A is called the domain of f(we also say f is defined on A), and the elements f(x) are called the values of f. The set of all values of f is called the range of f.
\end{defi}

\begin{defi}
	Let A and B be two sets and let f be a mapping of A into B. If $ E\subset A $ is defined to be the set of all elements f(x), for $ x\in E $. We call f(E) the image of E under f. In this notation, f(a) is the range of f. It is clear that $ f(a)\subset B $. If f(A)=B, we say that f maps A onto B.
	
	If $ E\subset B $, $ f^{-1}(E) $ the set of all $ x\in A $ such that $ f(x)\in E $. We call $ f^{-1}(E) $ the inverse image of E under f. If $ y\in B $, $ f^{-1}(y) $ is the set of all $ x\in A $ such that f(x)=y. If, for each $ y\in B $, $ f^{-1}(y) $ consists of at most one element of A, then f is said to be a 1-1 (one-to-one) mapping of A into B.This may also be expressed as follows: f is a 1-1 mapping of A into B provided that $ f(x_1)\neq f(x_2) $ whenever $ x_1\neq x_2, x_1\in A, x_2\in A $.
\end{defi}

\begin{defi}
	If there exists a 1-1 mapping of A onto B, we say that A and B can be put in 1-1 correspondence, or that A and B have the same cardinal number, or, briefly, that A and B are equivalent, and we write $ A\sim B $. This relation clearly has the following proprities:
	
	It is reflexive: $ A\sim A $.
	
	It is symmetric: If $ A\sim B $, then $ B\sim A $.
	
	It is transitive: If $ A\sim B $ and $ B\sim C $, then $ A\sim C $.
	
	Any relation with these three proprities is called an equivalence relation.
\end{defi}

\begin{defi}
	For any positive integer n, let $ J_n $ be the set whose elements are the integers 1, 2, $\cdots$, n; let J be the set consisting of all positive integers. For any set A, we say:
	
	(a) A is finite if $ A\sim J_n $ for some n (the empty set is also considered to be finite).
	
	(b) A is infinite if A is not finite.
	
	(c) A is countable if $ A\sim J $.
	
	(d) A is uncountable if A is neither finite nor countable.
	
	(e) A is at most countable if A is finite or countable.
\end{defi}

For two finite sets A and B, we evidently have A~B if and only if A and B contain the same number of elements. For infinite sets, however, the idea of "having the same number of elements" becomes quite vague, whereas the notion of 1-1 correspondence retains its clarity.

\begin{rem}
	A finite set cannot be equivalent to one of its proper subsets. That this is, however, possible for infinite sets.
\end{rem}

\begin{defi}
	By a sequence, we mean a function of f defined on the set of J of all positive integers. If $ f(n)=x_n $, for $ n\in J $, it is customary to denote the sequence f by the symbol {$ x_n $}, or sometimes by $x_1,x_2,x_3,\cdots$. The values of f, that is, the elements $ x_n $, are called the terms of the sequence. If A is a set and if $ x_n\in A $ for all $ n\in J $, then {$ x_n $} is said to be a sequence in A, or a sequence of elements of A.
\end{defi}

Since every countable set is the range of a 1-1 function defined on J, we may regard every countable set as the range of a sequence of distinct terms. Speaking more loosely, we may say that the elements of any countable set can be "arranged in a sequence."

\begin{theo}
	Every infinite subset of a countable set A is countable
\end{theo}

The theorem shows that, roughly speaking, countable sets represent the "smallest" infinity: No uncountable set can be a subset of a countable set.

\begin{defi}
	Let A and $\Omega$ be sets, and suppose that with each element $\alpha$ of A there is associated a subset of $\Omega$ which we denote by $ E_\alpha $
\end{defi}

The set whose elements are the sets $ E_\alpha $ will be denoted by {$ E_\alpha $}. Instead of speaking of sets of sets, we shall sometimes speak of a collection of sets, or a family of sets.

The union of the sets $ E_\alpha $ is defined to be the set S such that $ x\in S $ if and only if $ x\in E_\alpha $ for at least one $ \alpha\in A $. We use the notation

\[
S=\bigcup_{\alpha\in A}E_\alpha
\]

If A consists of the integers 1,2,...,n, one usually writes

\[
S=\bigcup^n_{m=1}E_m
\]

If A is the set of all positive integers, the usual notation is 

\[
S=\bigcup^{\infty}_{m=1}E_m
\]

as for unions. If $ A\cap B $ is not empty, we say that A and B intersect; otherwise they are disjoint.

Many properties of unions and intersections are quite similar to those of sums and products; in fact, the words sum and product were sometimes used in this connection, and the symbols $ \sum $ and $ \prod $ were written in place of $ \bigcup $ and $ \bigcap $.

\begin{theo}
	Let {$ E_n $}, n=1,2,3,..., be a sequence of countable sets, and put 
	\[
	S=\bigcup^{\infty}_{n=1}E_n
	\]
	Then S is countable.
\end{theo}

\begin{coro}
	Suppose A is at most countable, and, for every $ \alpha\in A $, $ B_\alpha $ is at most countable. Put, 
	\[
	T=\bigcup_{\alpha\in A}B_\alpha
	\]
	
	Then T is at most countable.
\end{coro}

For T is equivalent to a subset of S.

\begin{theo}
	 Let A be a countable set, and let $ B_n $ be the set of all n-tuples ($ a_1,...,a_n $), where $ a_k\in A $(k=1,...,n), and the elements $ a_1,...,a_n $ need not be distinct. The $ B_n $ is countable.
\end{theo}

\begin{coro}
	The set of all rational numbers is countable.
\end{coro}

\begin{theo}
	Let A be the set of all sequences whose elements are the digits 0 and 1. This set A is uncountable.
\end{theo}
The element of A are sequences like 1,0,0,1,0,1,1,1

Readers who are familiar with the binary representation of the real numbers will notice that theorem 2.14 implies that the set of all real numbers is uncountable.

\section{METRIC SPACES}
\begin{defi}
	A set X, whose elements we shall call points, is said to be a metric space if with any two points p and q of X there is associated a real number d(p,q), called the distance from p to q, such that
	
	(a) d(p,q)>0 if $ p\neq q $; d(p,p)=0;
	
	(b) d(p,q)=d(q,p);
	
	(c) d(p,q)$ \le $ d(p,r)+d(r,q), for any $ r\in X $.
	
	Any function with these three properties is called a distance function, or a metric.		

\end{defi}

The distance in $ R^k $ is defined by:
\[
d(\mathbf{x},\mathbf{y})=|\mathbf{x-y}|\qquad (\mathbf{x,y}\in R^k)
\]

It is important to observe that every subset Y of a metric space X is a metric space in its own right, with the same distance function.

Thus every subset of a euclidean space is a metric space.

\begin{defi}
	By the segment (a,b) we mean the set of all real numbers x such that $ a<x<b $.
	
	By the segment [a,b] we mean the set of all real numbers x such that $ a<\le x\le b $.	
	
	Occasionally we shall also encounter "half-open intervals" [a,b) and (a,b];...
	
	If $ a_i<b_i $ for i=1,2,...,k, the set of all points $\mathbf{x}= (x_1,\cdots,x_k) $ in $ R^k $ whose coordinates satisfy the inequalities $ a_i\le x_i\le b_i (1\le i\le k)$ is called a k-cell.
	
	If $ \mathbf{x}\in R^k $ and r>0, the open (or closed) ball B with center at $ \mathbf{x} $ and radius r is defined to be the set of all $ \mathbf{y}\in R^k $ such that $ |\mathbf{y-x}|<r $(or $ |\mathbf{y-x}\le r| $)
	
	We call a set $ E\subset R^k $ convex if
	\[
	\lambda	\mathbf{x}+(1-\lambda)\mathbf{y}\in E
	\]
	whenever $ \mathbf{x}\in E,\mathbf{y}\in E $, and $ 0<\lambda<1 $.
	
\end{defi}

\begin{defi}
	Let X be a metric space. All points and sets mentioned below are understood to be elements and subsets of X.
	
	(a) A neighborhood of P is a set $ N_r(p) $ consisting of all q such that $ d(p,q)<r $, for some $ r>0 $. The number r is called the radius of $ N_r(p) $.
	
	(b) A point p is a limit point of the set E if every neighborhood of p contains a point $ q\neq p $ such that $ q\in E $.
	
	(c) If $ p\in E $ and p is not a limit point of E, then p is called an isolated point of E.
	
	(d) E is closed if every limit point of E is a point of E.
	
	(e) A point p is an interior point of E if there is a neighborhood N of p such that $ N\subset E $.
	
	(f) E is open if every point of E is an interior point of E.
	
	(g) The complement of E (denoted by $ E^c $) is the set of all points $ p\in X $ such that $ p\notin E $.
	
	(h) E is perfect if E is closed and every point of E is a limit point of E.
	
	(i) E is bounded if there is a real number M and a point $ q\in X $ such that d(p,q)<M for all $ p\in E $.
	
	(j) E is dense in X if every point of X is a limiti point of E, or a point of E(or both).
	
\end{defi}

\begin{theo}
	Every neighborhood is an open set.
\end{theo}

\begin{theo}
	If p is a limit point of a set E, then every neighborhood of p contains infinitely many points of E.
\end{theo}

\begin{coro}
	A finite point set has no limit points.
\end{coro}

\begin{theo}
	Let {$ E_\alpha $} be a (finite or infinite) collection of sets $ E_\alpha $. Then
	\[
	(\bigcup_{\alpha} E_\alpha)^c=\bigcap_{\alpha}(E_\alpha^c)
	\]
\end{theo}

\begin{theo}
	A set E is open if and only if its complement is closed.
\end{theo}

\begin{coro}
	A set E is cloed if and only if its complement is open.
\end{coro}

\begin{theo}
	
\end{theo}

\end{document}