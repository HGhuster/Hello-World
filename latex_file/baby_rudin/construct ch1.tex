\documentclass{article}
\date{\today}
\title{construct baby rudin Ch.1}
\author{Dawei Wang}
%\usepackage{ctex}
\usepackage{amsmath}
\usepackage{amssymb}
\usepackage{amsthm}
%\usepackage{theorem}
\theoremstyle{definition}
\newtheorem{defi}{Definition}
\newtheorem{theo}{Theorem}
\newtheorem{coro}{Corollary}
%\newtheorem*{cantor}{The Cantor set}
\theoremstyle{remark}
\newtheorem{Rem}{Remark}
\begin{document}
	\maketitle
INTRUODCTION 
\section*{Example}
the equation 
\[
p^2=2
\]
is not satisfied by any rational p.

\hspace{\fill}

Examine this situation more closely. Let A be the set of all positive rationals p such that $p^2<2$ and let B be the set of all positive rationals p such that $p^2>2$. We shall now show that A contains no largest number and B contains no smallest.

\section*{Definition}\noindent

A is a set, 

$x\in A$, $x\notin A$, \textit{empty set}, \textit{nonempty}, 		$A\in B$, $B\in A$, $A=B$, $A\ne B$

\hspace{\fill}

Q

\hspace{\fill}

ORDER SETS

\section*{Definition}\noindent

order, 

ordered set, 

upper bound(lower bound), 

least upper bound(supremum), 

greatest lower bound(infimum)

least upper bound property

\section*{Theorem}

Suppose S is an ordered set with the least-upper-bound property, $B\in S$, B is not empty, and B is bounded below. Let L be the set of all lower bound of B. Then 

\[\alpha=sup L\]

exists in S and $\alpha= infB$.

In particular, inf B exists in S.

\hspace{\fill}

FILEDS

\section*{Definition}

field, ordered field

\hspace{\fill}

THE REAL FIELD

\section*{Theorem}

There exists an ordered field R which has the least-upper-bound property.

Moreover, R contains Q as a subfield.

\section*{Theorem}\noindent

(a) If $x\in R$, $y\in R$, and $x>0$, then there is a positive integer n such that

\[
nx>y
\]

(b) If $x\in R$, $y\in R$, and $x<y$, then there exists a $p\in Q$ such that x<p<y.





	
\end{document}