\documentclass{article}
\title{The Rules of the Game}
\author{Dawei Wang}
\date{\today}
\usepackage{ctex}
\usepackage{amsmath}
\usepackage{amssymb}
%\usepackage{graphicx} %插入图片的宏包
%\usepackage{float} %设置图片浮动位置的宏包
%\usepackage{subfigure} %插入多图时用子图显示的宏包
\begin{document}
	\maketitle

The paradigm of game theory: the modeller assigns payoff functions and strategy sets to his players and sees what happens when they pick strategies to maximize their payoffs.

\section{Definitions}
 Game theory is concerned with the actions of decision makers who are conscious that their actions affect each other.
 
\subsection{Describing a Game}

The essential elements of a game are \textbf{\textit{players}}, \textbf{\textit{actions}}, \textbf{\textit{payoffs}}, and \textbf{\textit{information}}— PAPI,
for short. These are collectively known as the \textbf{\textit{rules of the game}}, and the modeller’s
objective is to describe a situation in terms of the rules of a game so as to explain what will
happen in that situation. Trying to maximize their payoffs, the players will devise plans
known as \textbf{\textit{strategies}} that pick actions depending on the information that has arrived
at each moment. The combination of strategies chosen by each player is known as the
\textbf{\textit{equilibrium}}. Given an equilibrium, the modeller can see what actions come out of the
conjunction of all the players’ plans, and this tells him the \textbf{\textit{outcome}} of the game.

\hspace*{\fill}

\textbf{\textit{Players}} are the individuals who make decisions. Each player’s goal is to maximize his
utility by choice of actions.

Sometimes it is useful to explicitly include individuals in the model called \textbf{\textit{pseudo-players}}
whose actions are taken in a purely mechanical way.

\textbf{\textit{Nature}} is a pseudo-player who takes random actions at specified points in the game with
specified probabilities.We say
that there are different \textbf{\textit{realizations}} of a game depending on the results of random moves.

\hspace*{\fill}

An action or move by player i, denoted $ a_i $, is a choice he can make.

Player i’s action set, $ A_i=\{a_i\} $, is the entire set of actions available to him.


An action combination is an ordered set $ a={a_i},(i=1,\cdots,n) $ of one action for each
of the n players in the game.

\hspace*{\fill}

By player i’s payoff $ \pi_i(s_1,\cdots,s_n) $, we mean either:

(1) The utility player i receives after all players and Nature have picked their strategies and
the game has been played out; or

(2) The expected utility he receives as a function of the strategies chosen by himself and the
other players.

\hspace*{\fill}

The outcome of the game is a set of interesting elements that the modeller picks from the
values of actions, payoffs, and other variables after the game is played out.

The definition of the outcome for any particular model depends on what variables
the modeller finds interesting.

\hspace*{\fill}

Player i’s strategy $ s_i $ is a rule that tells him which action to choose at each instant of the
game, given his information set.

Player i’s strategy set or strategy space $ S_i={s_i} $
is the set of strategies available to him.


A strategy profile $ S=s(s_1,s_2,\cdots,s_n) $ is an ordered set consisting of one strategy for each of
the n players in the game.



Since the information set includes whatever the player knows about the previous actions
of other players, the strategy tells him how to react to their actions.

Only rarely can we predict
a player’s actions unconditionally, but often we can predict how he will respond to the
outside world.

strategy is a complete set of instructions for him, which
tells him what actions to pick in every conceivable situation, even if he does not expect to
reach that situation.The
completeness of the description also means that strategies, unlike actions, are unobservable.
An action is physical, but a strategy is only mental.

\subsection{Equilibrium}

The distinction between strategy profiles, which are sets of strategies, and outcomes, which
are sets of values of whichever variables are important.

An equilibrium $ s^*=(s^*_1,\cdots,s^*_n) $ is a strategy profile consisting of a best strategy for each
of the n players in the game.


The equilibrium strategies are the strategies players pick in trying to maximize
their individual payoffs, as distinct from the many possible strategy profiles obtainable
by arbitrarily choosing one strategy per player.

An equilibrium concept or solution concept $ F:\{S_1,\cdots,S_n,\pi_1,\cdots\pi_n\}\rightarrow s^* $ is a rule
that defines an equilibrium based on the possible strategy profiles and the payoff functions.
Only a few
equilibrium concepts are generally accepted


\subsection{Uniqueness}

Accepted solution concepts do not guarantee uniqueness, and lack of a unique equilibrium
is a major problem in game theory.

Often the solution concept employed leads us to believe
that the players will pick one of the two strategy profiles A or B, not C or D, but we cannot
say whether A or B is more likely. Sometimes we have the opposite problem and the game
has no equilibrium at all.

A model with no equilibrium or multiple equilibria is underspecified.

\section{Dominated and Dominant Strategies: The Prisoner’s Dilemma}

For any vector$ y=(y_1,\cdots,y_n) $, denote by $ y_{-i}  $,  the vector 

\[
  (y_1,\cdots,y_{i-1},y_{i+1},\cdots,y_n) 
\]

which is the portion of y not associated with player i.

\hspace*{\fill}

Player i’s best response or best reply to the strategies $ s_{−i} $ chosen by the other players
is the strategy $ s_i^* $ that yields him the greatest payoff; that is,

\[
\pi_i(s_i^*,s_{-i}) \ge \pi_i(s_i' ,s_{-i}),\forall s_i'\ne s_i^*
\]

The best response is strongly best if no other strategies are equally good, and weakly best
otherwise.

\hspace*{\fill}

$ s^d_i $ is dominated if there exists a single $ s'_i $ such that
\[
\pi_i(s^d_i,s_{-i})<\pi_i(s'_i,s_{-i}),\enspace \forall s_{-i}
\]

Note that $ s^d_i $
is not a dominated strategy if there is no $ s_{−i} $ to which it is the best response,
but sometimes the better strategy is $ s'_i $ and sometimes it is $ s''_i $. A dominated strategy is unambiguously inferior to some single other strategy.

\hspace*{\fill}

The strategy $ s^∗_i $ is a dominant strategy if it is a player’s strictly best response to any
strategies the other players might pick, in the sense that whatever strategies they pick, his
payoff is highest with $ s^∗_i $ . Mathematically,

\[
\pi_i(s_i^*,s_{-i}) > \pi_i(s_i' ,s_{-i}),\enspace \forall s_{-i},\forall s_i'\ne s_i^*
\]

A dominant strategy equilibrium is a strategy profile consisting of each player’s dominant
strategy. Most games do not have dominant strategies.

\subsection{Cooperative and Noncooperative Games}

A cooperative game is a game in which the players can make binding commitments, as
opposed to a noncooperative game, in which they cannot.

\section{Iterated Dominance}

Strategy $ s'_i $ is weakly dominated if there exists some other strategy $ s''_i $ for player i which is
possibly better and never worse, yielding a higher payoff in some strategy profile and never
yielding a lower payoff. Mathematically, $ s'_i $ is weakly dominated if there exists $ s''_i $ such that

\[
\pi_i(s''_i,s_{-i})\ge\pi_i(s'_i,s_{-i})\forall s_{-i}
\]

and

\[
\pi_i(s''_i,s_{-i})>\pi_i(s'_i,s_{-i})\enspace for\enspace some\enspace s_{-i}
\]

One might define a weak dominance equilibrium as the strategy profile found by
deleting all the weakly dominated strategies of each player.

An iterated dominance equilibrium is a strategy profile found by deleting a weakly
dominated strategy from the strategy set of one of the players, recalculating to find which
remaining strategies are weakly dominated, deleting one of them, and continuing the process
until only one strategy remains for each player.

\subsection{zero-sum game}

A zero-sum game is a game in which the sum of the payoffs of all the players is zero
whatever strategies they choose. A game which is not zero-sum is nonzero-sum game or
variable- sum.

Often modellers will refer
to a game as zero-sum even when the payoffs do not add up to zero, so long as the payoffs
add up to some constant amount. The difference is a trivial normalization.

\section{Nash Equilibrium}

The strategy profile s∗ is a Nash equilibrium if no player has incentive to deviate from
his strategy given that the other players do not deviate. Formally,

\[
\forall i\enspace \pi_i(s^*_i,s^*_{-i})\ge\pi_i(s'_i,s^*_{-i}),\enspace \forall s'_i
\]

A Nash strategy need only be a best response to the other Nash strategies,
not to all possible strategies.

Like a dominant strategy equilibrium, a Nash equilibrium can be either weak or strong.
The definition above is for a weak Nash equilibrium. To define strong Nash equilibrium,
make the inequality strict; that is, require that no player be indifferent between his equilibrium
strategy and some other strategy.

Every dominant strategy equilibrium is a Nash equilibrium, but not every Nash equilibrium
is a dominant strategy equilibrium.If a strategy is dominant it is a best response to
any strategies the other players pick, including their equilibrium strategies. If a strategy is
part of a Nash equilibrium, it need only be a best response to the other players’ equilibrium
strategies.

\subsection{Coordination Games}

Ranked Coordination is one of a large class of games called coordination games,
which share the common feature that the players need to coordinate on one of multiple
Nash equilibria.

\subsection{Focal points}

Certain of the strategy profiles are
focal points: 

Nash equilibria which for psychological reasons are particularly compelling.

Formalizing what makes a strategy profile a focal point is hard and depends on the
context.

\hspace*{\fill}

The boundary is a particular kind of focal point.

Once the boundary is established it takes on additional significance because behavior
with respect to the boundary conveys information.

Boundaries must be sharp and well known if they are not to be violated, and
a large part of both law and diplomacy is devoted to clarifying them.

\hspace*{\fill}

Mediation and communication are both important in the absence of a clear focal
point.

\end{document}