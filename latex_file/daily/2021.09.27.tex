\documentclass{article}
\usepackage{amsmath}
\usepackage{amssymb}
\usepackage{diagbox}
\usepackage{graphicx}
\usepackage{float}
\begin{document}
	\section*{1}
	\subsection*{A}\noindent

	Because $C\sim N(5.7\%,(8\%)^2)$
	
	$\frac{C-5.7\%}{8\%}\sim N(0,1)$
	
	$P(C<0)=\Phi(\frac{0-5.7\%}{8\%})=0.2380776$
	
	\subsection*{B}\noindent
	
	Assume the value of my asset is $\alpha$, then I will invest $0.5\alpha$ in stock A, $0.25\alpha$ in stock B and $0.25\alpha$ in stock C.
	
	My expected return is :
	
	\begin{equation*}
		\begin{split}
	E(0.5\alpha\cdot A+0.25\alpha\cdot B+0.25\alpha\cdot C)&=0.5\alpha E(A)+0.25\alpha\cdot E(B)+0.25\alpha\cdot E(C)\\
			&=0.5\alpha\times 1.2\%	+ 0.25\alpha\times 3.4\%+0.25\alpha\times 5.7\%\\
			&=2.875\%\alpha	
		\end{split}
	\end{equation*}
	
	My expected return rate:$\frac{E(0.5\alpha\cdot A+0.25\alpha\cdot B+0.25\alpha\cdot C)}{\alpha}=2.85\%$
	
	\subsection*{C}\noindent
	
	As the answer for A shows, the probability that the invest in stock C gets a negative return is $P(C<0)=0.2380776$.
	
	Because  $A\sim N(1.2\%,(0.8\%)^2)$:
	
	$P(A<0)=\Phi(\frac{0-1.2\%}{0.8\%})=0.0668072$
	
	Similarly, because $B\sim N(3.4\%,(2.7\%)^2)$

	$P(B<0)=\Phi(\frac{0-3.4\%}{2.7\%})=0.1039684$
	
	\hspace*	{\fill}
	
	So the probability that all three stocks will earn a negative return is:
	
	$P(A<0\&B<0\&C<0)=P(A<0)\times P(B<0)\times P(C<0)=0.001653648 $

\newpage

	\section*{2}
	
	\subsection*{A}\noindent
	
	\begin{tabular}{l|cccc}
   \diagbox{Y}{X} & 1 & 0 \\
  	\hline
  	1 & $\frac{4\times 8}{40\times 39}$ & $\frac{120}{40\times 39}$\\
  	0 & $\frac{4\times 31}{40\times 39}$& $\frac{36\times 39}{40\times 39}-\frac{120}{40\times 39}$
	\end{tabular}
	
	\subsection*{B}\noindent
	
	\begin{tabular}{l|cc}
  Y&1 &0 \\
  \hline
  & $\frac{15}{100}$& $\frac{85}{100}$ 
\end{tabular}
	
	
	\subsection*{C}\noindent
	
	\begin{tabular}{l|cc}
  	X& 1 & 0\\
  	\hline
  	&$\frac{10}{100}$ &$\frac{90}{100}$ 
	\end{tabular}

	
	\subsection*{D}\noindent
	
	$P(X=1)P(Y=1)=\frac{15}{10000}$
	
	$P(X=1\&Y=1)=\frac{3}{100}$
	
	$P(X=1)P(Y=1)\ne P(X=1\&Y=1)$
	
	So X and Y are not independent
	
	\subsection*{E}\noindent
	
	$E(Y)=P(Y=1)\times 1+P(Y=0)\times 0=0.15$
	
	$$
	
	$D(Y)=(1-E(Y))^2\times P(X=1)+(0-E(Y))^2\times P(X=0)=0.1275$
	\newpage
	\section*{3}
	
	\subsection*{A}\noindent
	
	\begin{figure}[hbt]
	\centering
  	\includegraphics[width=0.3\textwidth]{2021_09_30_01.heic}
  	\caption{}
	\end{figure}
	
	\subsection*{B}\noindent
	
	I can't determine the specific values of c and d, but I can determine the relationship between c and d:
	
	For the total area between the probability density curve is 1:
	
	\begin{equation*}
		\begin{split}
		1=&2c\times 1\times\frac{1}{2}+3d\times(5-2)\times\frac{1}{2}\\
		=&c+\frac{9}{2}d
		\end{split}
	\end{equation*}
 
 	So we get $c+\frac{9}{2}d=1$.
	
	
	\subsection*{C}\noindent
	Given that that the probability that X is between 0 and 1 is four times as large as the probability that X falls between 2 and 5:
	\begin{equation*}
		\begin{split}
			&2c\times 1\times\frac{1}{2}=4\times 3d\times(5-2)\times\frac{1}{2}\\
			&c=18d
		\end{split}
	\end{equation*}
	
	from the answer for 3.B we know that $c+\frac{9}{2}d=1$.
	
	$
	\left\{
	\begin{array}
		&c+\frac{9}{2}d=1\\
		&c=18d
	\end{array}
	\right.	
	$
	\Rightarrow
	$
	\left\{
	\begin{array}
	&d=\frac{2}{45}\\
	&c=\frac{4}{5}	
	\end{array}
	\right.	
	$
	
	\subsection*{D}\noindent
	
	The area between 2 and 3 is $...=\frac{1}{9}$
	
	The area between 3 and 4 is $...=\frac{1}{15}$
	
	So:
	
	$P(2\le x\le 3)=\frac{1}{9}$
	
	$P(3\le x\le 4)=\frac{1}{15}<P(2\le x\le 3)$
	
	So I would bet that X would take on a value
between 2 and 3.
	
	\section*{4}
	
	Assume that the standard deviation of scores distribution is $\sigma$,
	
	X= the test scores
	
	$X\sim N(70,\sigma)\Rightarrow \frac{X-70}{\sigma}\sim N(0,1)$
	
	Since $P_{1.5}=0.9332$ and $P(X<85)=0.9332$
	
	$\frac{85-70}{\sigma}=1.5\Rightarrow \sigma=10$
	
	\section*{5}
	
	\subsection*{A}
	
	The probability that a current smoker is unemployed is:
	
	$P(W=0|S=2)=\frac{P(W=0\cap S=2)}{P(S=2)}=\frac{P(W=0\cap S=2)}{P(S=2\cap W=0)+P(S=2\cap W=1)}=\frac{0.14}{0.14+0.22}=0.3888889$
	
	\subsection*{B}
	
	The probability that an individual is a former smoker is:
	
	$P(S=1\cap W=0)+P(S=1\cap W=1)=0.17+0.17=0.34$
	
	\subsection*{C}
	
	$P(W=1)=P(W=1\cap S=0)+P(W=1\cap S=1)+P(W=1\cap S=2)=0.18+0.17+0.22=0.57$
	
	$P(W=0)=1-P(W=1)=0.43$
	
	$E(W)=1\times P(W=1)+0\times P(W=0)=0.57$
	
	\subsection*{D}
	
	$P(S=0)=P(S=0\cap W=0)+P(S=0\cap W=1)=0.12+0.18=0.3$
	
	$P(S=0)\times P(W=1)=0.171\ne P(S=0\cap W=1)$
	
	So smoking and employment are not independent.
	
	\subsection*{E}
	
	$P(S=0|W=1)=\frac{P(S=0\cap W=1)}{P(W=1)}=\frac{0.18}{0.57}=0.3157895$
	
	$P(S=1|W=1)=\frac{P(S=1\cap W=1)}{P(W=1)}=\frac{0.17}{0.57}=0.2982456$
	
	$P(S=2|W=1)=\frac{P(S=2\cap W=1)}{P(W=1)}=\frac{0.22}{0.57}=0.3859649$
	
	\section*{6}
	
	\subsection*{A}
	
	If the size is 5, there are $\frac{20!}{(20-5)!}=1860480$
	 ways;
	 
	If the size is 7, there are $\frac{20!}{(20-7)!}=390700800$
	 ways;
	
	If the size is 9, there are $\frac{20!}{(20-9)!}=60949324800$
	 ways;
	
	If the size is 11, there are $\frac{20!}{(20-11)!}= 6704425728000$
	 ways;
	
	\subsection*{B}
	
	5,
	
	
	\subsection*{C}
	
	5
	
	\subsection*{D}
	
	The probability of picking a committee of 9 representatives and all were Republicans is $\frac{C^9_{11}}{C^9_{20}}=\frac{55}{167960}=0.0003274589$, so the governor probably didn't choose the committee randomly.
	
	\section*{7}
	
	$0.05\times 0.25\times 0.4^2\times 0.3^2\times 
	\frac{5!}{2!2!1!}=0.00108$
	
	\section*{8}
	
	\subsection*{A}	
	
	$A_n= the\enspace National\enspace league\enspace team\enspace wins\enspace the\enspace n_{th}\enspace game$
	
	$P(A_1)=0.7$
	
	$P(A_2|A_1)=0.75, P(A_2|\overline{A_1})=0.4\Rightarrow P(A_2\cap A_1)=0.525,P(A_2\cap\overline{A_1})=0.21$
	
	$P(A_2)=P(A_2 \cap A_1)+P(A_2\cap \overline{A_1})=0.735$
	
	$P(A_1)\times P(A_2)\ne P(A_1\cap A_2)$
	
	So the games are not independent.
	
	\newpage
	
	\subsection*{B}
	
	When the series lasts 4 games:
	
	The probability that the National league team wins:
	
	$P=0.7\times(0.75)^3=0.2953125$
	
	The probability that the American League wins:
	
	$0.3\times 0.6^3=0.0648$
	
	The probability that the series lasts 4 games:$0.2953125+0.0648=0.3601125$
	
	\subsection*{C}
	
	When the series lasts 5 games:
	
	The probability that the National league team wins:
	
	$P1_5=P(\overline{A_1}\cap A_2\cap A_3\cap A_4 \cap A_5)=0.3\times 0.4\times(0.75)^3=0.050625$
	
	$P2_5=P(A_1\cap \overline{A_2}\cap A_3\cap A_4\cap A_5)=0.7\times 0.25\times 0.4\times(0.75)^2=0.039375$
	
	$P3_5=P(A_1\cap A_2\cap \overline{A_3}\cap A_4\cap A_5)=0.7\times 0.75\times 0.25\times0.4\times 0.75=0.039375$
	
	$P4_5=P(A_1\cap A_2\cap A_3\cap \overline{A_4}\cap A_5)=0.7\times 0.75\times 0.75\times 0.25\times0.4=0.039375$
	
	$P1_5+P2_5+P3_5+P4_5=0.16875$	

	The probability that the American League wins:
	
	$P1_5'=P(\overline{A_1}\cap \overline{A_2}\cap \overline{A_3}\cap A_4 \cap \overline{A_5})=0.3\times 0.6\times 0.6\times 0.4\times 0.25=0.0108$

	$P2_5'=P(\overline{A_1} \cap \overline{A_2} \cap A_3\cap \overline{A_4} \cap \overline{A_5})=0.3\times 0.6\times 0.4\times 0.25\times 0.6=0.0108$
		
	$P3_5'=P(\overline{A_1}\cap A_2\cap \overline{A_3}\cap \overline{A_4} \cap \overline{A_5})=0.3\times 0.4\times 0.25\times 0.6\times 0.6=0.0108$

	$P4_5'=P(A_1\cap \overline{A_2}\cap \overline{A_3}\cap \overline{A_4} \cap \overline{A_5})=0.7\times 0.25\times(0.6)^3=0.0378$
	
	$P1_5'+P2_5'+P3_5'+P4_5'=0.0702$

	The probability that the series lasts 5 games:
	
	$P1_5+P2_5+P3_5+P4_5+P1_5'+P2_5'+P3_5'+P4_5'=0.23895$

	
	\hspace*{\fill}
	
	When the series lasts 6 games:
	
	The probability that the National league team wins:
	
	$P1_6=P(\overline{A_1}\cap \overline{A_2}\cap A_3\cap A_4 \cap A_5\cap A_6)=0.3\times 0.6\times 0.4\times 0.75^3=0.030375$
	
	$P2_6=P(\overline{A_1}\cap A_2\cap \overline{A_3}\cap A_4 \cap A_5\cap A_6)=0.3\times 0.4\times 0.25\times 0.4\times 0.75^2=0.00675$
	
	$P3_6=P(\overline{A_1}\cap A_2\cap A_3\cap \overline{A_4} \cap A_5\cap A_6)=0.00675$
	
	$P4_6=P(\overline{A_1}\cap A_2\cap A_3\cap A_4 \cap \overline{A_5}\cap A_6)=0.00675$
	
	$P5_6=P(A_1\cap \overline{A_2}\cap \overline{A_3}\cap A_4 \cap A_5\cap A_6)=0.7\times 0.25 \times 0.6\times 0.4\times 0.75^2= 0.023625$
	
	$P6_6=P(A_1\cap \overline{A_2}\cap A_3\cap \overline{A_4} \cap A_5\cap A_6)=0.7\times 0.25\times 0.4\times0.25\times 0.4\times 0.75=0.00525$
	
	$P7_6=P(A_1\cap \overline{A_2}\cap A_3\cap A_4 \cap \overline{A_5}\cap A_6)=0.00525$
	
	$P8_6=P(A_1\cap A_2\cap \overline{A_3}\cap \overline{A_4} \cap A_5\cap A_6)=0.023625$
	
	$P9_6=P(A_1\cap A_2\cap \overline{A_3}\cap A_4 \cap \overline{A_5}\cap A_6)=0.00525$
	
	$P10_6=P(A_1\cap A_2\cap A_3\cap \overline{A_4} \cap \overline{A_5}\cap A_6)=0.023625$

		
	The probability that the American League wins:
	
	$P1_6'=P(A_1\cap A_2\cap \overline{A_3}\cap \overline{A_4} \cap \overline{A_5}\cap \overline{A_6})=0.7*0.75*0.25*0.6^3= 0.02835$
	
	$P2_6'=P(A_1\cap \overline{A_2}\cap A_3\cap \overline{A_4} \cap \overline{A_5}\cap \overline{A_6})=0.7\times 0.25\times 0.4\times 0.25\times 0.6^2=0.0063$
	
	$P3_6'=P(A_1\cap \overline{A_2}\cap \overline{A_3}\cap A_4 \cap \overline{A_5}\cap \overline{A_6})=0.0063$
	
	$P4_6'=P(A_1\cap \overline{A_2}\cap \overline{A_3}\cap \overline{A_4} \cap A_5\cap \overline{A_6})=0.0063$
	
	$P5_6'=P(\overline{A_1}\cap A_2\cap A_3\cap \overline{A_4} \cap \overline{A_5}\cap \overline{A_6})=0.0081 $
	
	$P6_6'=P(\overline{A_1}\cap A_2\cap \overline{A_3}\cap A_4 \cap \overline{A_5}\cap \overline{A_6})= 0.0018$
	
	$P7_6'=P(\overline{A_1}\cap A_2\cap \overline{A_3}\cap \overline{A_4} \cap A_5\cap \overline{A_6})= 0.0018$
	
	$P8_6'=P(\overline{A_1}\cap \overline{A_2}\cap A_3\cap A_4 \cap \overline{A_5}\cap \overline{A_6})=0.0081$
	
	$P9_6'=P(\overline{A_1}\cap \overline{A_2}\cap A_3\cap \overline{A_4} \cap A_5\cap \overline{A_6})= 0.0018$
	
	$P10_6'=P(\overline{A_1}\cap \overline{A_2}\cap \overline{A_3}\cap A_4 \cap A_5\cap \overline{A_6})=0.0081$
	
	The probability that the series lasts 6 games:
	
	$P1_6+\cdots+P10_6+P1_6'+\cdots+P10_6'=0.2142$
	
	\hspace*{\fill}
	
	When the series lasts 7 games(each team wins 3 games in the first 6 games):
	
	$P1_{33}=P(A_1\cap A_2\cap A_3\cap \overline{A_4}\cap \overline{A_5}\cap \overline{A_6})=0.7\times 0.75^2\times 0.25\times 0.6^2=0.0354375$
	
	$P2_{33}=P(A_1\cap A_2\cap \overline{A_3}\cap A_4\cap \overline{A_5}\cap \overline{A_6})=0.7\times 0.75\times 0.25\times 0.4\times 0.25\times 0.6=0.007875$
	
	$P3_{33}=P(A_1\cap A_2\cap \overline{A_3}\cap \overline{A_4}\cap A_5\cap \overline{A_6})=0.7\times 0.75\times 0.25\times 0.6\times 0.4\times 0.25=0.007875$
	
	$P4_{33}=P(A_1\cap A_2\cap \overline{A_3}\cap \overline{A_4}\cap \overline{A_5}\cap A_6)=0.7\times 0.75\times 0.25\times 0.6^2\times 0.4=0.0189$
	
	$P5_{33}=P(A_1\cap \overline{A_2}\cap A_3\cap A_4\cap \overline{A_5}\cap \overline{A_6})=0.7\times 0.25\times 0.4\times 0.75\times 0.25\times 0.6=0.007875 $
	
	$P6_{33}=P(A_1\cap \overline{A_2}\cap A_3\cap \overline{A_4}\cap A_5\cap \overline{A_6})=0.7\times 0.25\times 0.4\times 0.25\times 0.4\times 0.25=0.00175$
	
	$P7_{33}=P(A_1\cap \overline{A_2}\cap A_3\cap \overline{A_4}\cap \overline{A_5}\cap A_6)=0.7\times 0.25\times 0.4\times 0.25\times 0.6\times 0.4=0.0042$
	
	$P8_{33}=100110=0.7*0.25*0.6*0.4*0.75*0.25=0.007875$
	
	$P9_{33}=1 0 0 1 0 1=0.7*0.25*0.6*0.4*0.25*0.4=0.0042$
	
	$P10_{33}=1 0 0 0 1 1=0.7*0.25*0.6*0.6*0.4*0.75=0.0189$
	
	$P11_{33}=0 1 1 1 0 0=0.3*0.4*0.75*0.75*0.25*0.6=0.010125$
	
	$P12_{33}=0 1 1 0 1 0=0.3*0.4*0.75*0.25*0.4*0.25=0.00225$
	
	$P13_{33}=0 1 1 0 0 1=0.3*0.4*0.75*0.25*0.6*0.4=0.0054$
	
	$P14_{33}=0 1 0 1 1 0=0.3*0.4*0.25*0.4*0.75*0.25=0.00225$
	
	$P15_{33}=0 1 0 1 0 1=0.3*0.4*0.25*0.4*0.25*0.4 =0.0012$
	
	$P16_{33}=0 1 0 0 1 1=0.3*0.4*0.25*0.6*0.4*0.75=0.0054$
	
	$P17_{33}=0 0 1 1 1 0=0.3*0.6*0.4*0.75*0.75*0.25=0.010125$
	
	$P18_{33}=0 0 1 1 0 1=0.3*0.6*0.4*0.75*0.25*0.4=0.0054$
	
	$P19_{33}=0 0 1 0 1 1=0.3*0.6*0.4*0.25*0.4*0.75=0.0054$
	
	$P20_{33}=0 0 0 1 1 1=0.3*0.6*0.6*0.4*0.75*0.75=0.0243$
	
	The probability that the series lasts 7 games:
	
	$P1_{33}+\cdots+P20_{33}=0.1867375$
	
	\subsection*{D}
	
	0.3601125+0.23895+0.2142+0.1867375=1
	
	\subsection*{E}
	
	
	
	\section*{9}
	
	\subsection*{A}
	
	X= scores
	
	$X\sim N(440,60)\Rightarrow \frac{X-440}{60}\sim N(0,1)$
	
	$420<X<500\Rightarrow -\frac{1}{3}<\frac{X-440}{60}<1$
	
	$P_1-P_{-\frac{1}{3}}=0.4719034$
	
	\subsection*{B}
	
	$P_{1.644854}=0.95\Rightarrow \frac{X-440}{60}=1.644854$
	
	X=538.6912
	
	\subsection*{C}
	
	$X=500\Rightarrow \frac{X-440}{60}=1$
	
	$P_1=0.8413447$
	
	The probability that all there scores all below 500 is:
	
	$P_1^2=0.7078609$
	
	The probability that at least one of them scored more than 500 is:
	
	$1-P_1^2=0.2921391$
	
	\section*{10}
	
	X= amount of dye discharged
	
	$X\sim N(\Theta,0.4)\Rightarrow \frac{X-\Theta}{0.4}\sim N(0,1)$
	
	\hspace*{\fill}
	
	The probability that the shade is unacceptable is:
	
	$1-P(X\le 6)=1-P(\frac{X-\Theta}{0.4}\le \frac{6-\Theta}{0.4})=0.5\%$
	
	because $P_{2.575829}=0.995$
	
	$\frac{6-\Theta}{0.4}=2.575829$
	
	$\Theta=4.969668$
	
	
\end{document}