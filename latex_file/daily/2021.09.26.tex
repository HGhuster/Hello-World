\documentclass[11pt]{article}

\begin{document}

\section*{6.A}\noindent

A: test positive

B: have disease

$P(A|B)=90\%$

$P(\overline{A}|\overline{B})=95\%$

$P(B)=2\%$

\section*{6.B}\noindent

$P(\overline{A}|\overline{B})=95\%\rightarrow P(A|\overline{B})=5\%\rightarrow P(A\overline{B})/P(\overline{B})=5\%$

$P(A\overline{B})=5\%\times 98\%$

$P(A|B)=P(AB)/P(B)=90\%\rightarrow P(AB)=90\%\times 2\%$

$P(A)=P(AB)+P(A\overline{B})=0.067$

$P(B|A)=\frac{P(AB)}{P(A)}=\frac{0.018}{0.067}\approx26.87\%$

I use the law of total probability to solve this problem.

\section*{6.C}\noindent

My answer says the chances a positive result is right is 26.87\%.

Because the disease affects only 2\% of the population,so even if the test is still very accurate to the people who are not affected. The number of people who are false positive is still considerable to the number of people who a true positive. So the rate of false positives  is so high.


\section*{7.A}\noindent

$P(A|3times\enspace positive)=P(A\&3times\enspace positive)/P(3times\enspace positive)$

$P(A\&3times\enspace positive)=P^3(AB)=5.832\times 10^{-6}$

$P(3times\enspace positive)=P^3(A)=0.000300763=3.00763\times 10^{-4}$

$P(A|3times\enspace positive)\approx 1.94\%$

\section*{7.B}\noindent

Because with more times of test, the probability that all tests are false positive decrease, for example if the probability of false positive is a in one time(0<a<1),then the probability of false positive for all 3 times is $a^3$,$a^3<a$.

\end{document}
