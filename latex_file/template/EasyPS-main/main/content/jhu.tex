\begin{quote}
All knowledge is, in final analysis, history.

All sciences are, in the abstact, mathematics.

All judgements are, in their rationale, statistics.

\rightline{---Calyampudi Radhakrishna Rao
}
\end{quote}


It is still hard for me to recall my astonishment when I first read the quotation above.

When I was a freshman, I participated in the summer social practice of investigating the hybrid
power car market development, visited BYD's electric car factory, and conducted a
questionnaire survey in popular tourist attractions and public places in Nanjing. In this process,
I deeply felt the enormous social and economic benefits of technological innovation. Both
consumers and environmental protection benefit from it. However, at that time, the news of car
companies cheating government subsidies by fabricating sales of new energy vehicles
frequently appeared in China, which made me very sad. I communicated with my friends in
finance majors and learned that for companies listed on the stock market, the CSRC would
require them to disclose information to the public regularly. Suppose more car companies raise
more money through open capital markets. In that case, they can be under the public's
supervision, and the development of the new energy vehicle industry may be more positive. I
was impressed by the significant role of financial markets in improving the efficiency of using
funds, and I began to learn the relevant knowledge of finance by myself.

During my undergraduate, I learned nuclear engineering and technology at Huazhong
University of Science and Technology, ranking seventh in my major. I performed very well in
all mathematics classes, such as calculus (90), linear algebra (93), probability theory and
mathematical statistics (96), and complex function and integral transform (93). At the same
time, I took the initiative to act as a representative of many challenging courses, and I also
achieved good results in related classes, such as theoretical mechanics (97) and mechanical
principles (93). As a result, I was recommended postgraduate in a nuclear engineering major.
However, my heart is already yearning for finance, and I resolutely chose to take the exam and
explore the finance area.

While studying diligently, I also actively participated in academic research and Student
Union.In the sophomore summer vacation, with curiosity about scientific research, I joined the
computer simulation research project of material irradiation in my school, which gave me
preliminary academic research training. My main task is to learn how to use LAMMPS software
to simulate the radiation damage mechanism of the first wall material of Tokamak and compare
the computer simulation results with the experimental observation results. While studying the radiation damage of the material, the first wall material's radiation resistance is optimized by
changing the composition and structure in the computer simulation. Meanwhile, I actively
participated in the Student Union serving students and joined the Cultural Communication
Department. I am responsible for applying a budget of 3,000 yuan to the school every semester
to buy the books students wanted to read and for the weekend duty of Qiming study.
Simultaneously, I also organized a bi-weekly 25-person movie sharing session to enrich
everyone's cultural life.

When I prepared the graduation thesis, I chose to study the public acceptance of nuclear energy
from behavioral economics. Under my advisor's guidance, I independently collected and sorted
out the data through the questionnaires. Through using the different expressions of the same
question in the questionnaire and the asymmetry of people's nuclear radiation value function
obtained by data analysis, I verified that people's irrational fear of nuclear energy could be
partially explained from the perspective of frame effect and prospect theory in behavioral
economics. And I put forward some suggestions to improve public acceptance of nuclear energy
accordingly.