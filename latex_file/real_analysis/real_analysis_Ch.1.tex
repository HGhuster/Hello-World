\documentclass{article}
\title{Real analysis Ch.1}
\author{Dawei Wang}
\date{\today}
\usepackage{ctex}
\usepackage{amsmath}
\usepackage{amssymb}
\begin{document}
	\maketitle
	%	\tabletents
\section{皮亚诺公理}
\subsection{定义:自然数(非正式)}
自然数是集合
\[
N:={0,1,2,3,4...}
\]
的元素。其中,N是从零开始无休止地往前计数所得到的所有元素构成的集合,我们称N为自然数集。
\\ \hspace*{\fill}

为了定义自然数,我们将使用如下两个基本概念数0和增量运算。用n++来表示n的增量或紧跟在n之后的数。
\subsection{公理1.1}
0是一个自然数。

\subsection{公理1.2}
如果n是一个自然数,那么n++也是一个自然数。

\subsection{定义}
定义1为数$0++$,2为数$ (0++)++ $,3为数$ ((0++)++)++ $,等等。

\subsection{公理1.3}
0不紧跟在任何自然数后。换言之,对任意一个自然数n,$ n++\ne 0 $均成立。

\subsection{公理1.4}
对不不同的自然数而言,紧跟在它们之后的数字也一定是不同的,也就是说,如果n和m都是自然数,并且$ n\ne m $,那么$ n++\ne m++ $。逆否命题为,若$ n++=m++ $,则一定有$ n=m $

\subsection{公理:数学归纳法原理1.5}
令$ P(n) $表示自然数n得任意一个性质,如果$ P(0) $为真且$ p(n) $为真时一定有$ P(n++) $为真,那么对于任意自然数n,$ P(n) $一定为真。

\subsection{假设(非正式)}
存在一个数系N,我们称N中的元素为自然数,而且公理1.1~1.5对N均成立。

\subsection{注:}
尽管每一个自然数都是有穷的,但是由自然数所构成的集合却是无穷大的;也就是说虽然N是无穷大的,但是N由各个有穷的元素构成。不存在无穷大的自然数,无穷大不是自然数。(0是有穷的,假若n是有穷的,那么n++也是有穷的,因此根据公理1.5,所有自然数都是有穷的。)


\section{加法}
\subsection{定义:自然数的加法}
令m为一个自然数,我们定义m加上0为:$ 0+m:=m $。现在归纳的假设我们已经定义了如何把m加上n,那么我们把m加上$ n++ $定义为$ (n++)+m:=(n+m)++ $

\subsection{引理}
对任意自然数n,$ n+0=n $恒成立

\subsection{引理}
对任意自然数n和m,有$ n+(m++)=(n+m)++ $成立。

\subsection{命题:加法是可交换的}
对任意自然数n和m,有$ n+m=m+n $成立。

\subsection{命题:加法是可结合的}
对任意三个自然数$ a,b,c $,有$ (a+b)+c=a+(b+c) $成立。

\subsection{命题:消去律}
令$ a,b,c $为三个任意自然数并满足$ a+b=a+c $,那么$ b=c $成立。

\subsection{定义:正自然数}
称一个自然数n是正的,当且仅当它不等于0.

\subsection{引理}
令a表示一个正自然数,那么卡存在一个自然数b使得$ b++=a $。

\subsection{定义:自然数的序}
令n和m表示任意两个自然数。我们称n大于等于m,并记作$ n\ge m $,当且仅当存在自然数a使得$ n=m+a $。我们称n严格大于m,并记作$ n>m $,当且仅当$ n\ge m $且$ n\ne m $。
\\ \hspace*{\fill}

由于对任意n,有$ n++>n $,故不存在最大的自然数n,因为下一个数n++总是更大。

\subsection{命题:自然数的序的基本性质}
令$ a,b,c $为任意自然数,那么:

$ (a) $\qquad (序是自反的)$ a\ge a $。

$ (b) $\qquad (序是可传递的)如果$ a\ge b $并且$ b\ge c $,那么$ a\ge c $。

$ (c) $\qquad (序是反对称的)$ a\ge b $并且$ b\ge a $,那么$ a=b $。

$ (d) $\qquad (加法保持序不变)$ a\ge b $,当且仅当$ a+c\ge b+c $。

$ (e) $\qquad $ a<b $,当今仅当存在$ a++<b $。

$ (f) $\qquad $ a<b $,当接仅当存在正自然数d$ b=a+d $。

\subsection{命题:自然数的序的三岐性}
令a和b表示任意两个自然数,那么下面三种表述恰有一种表述为真:
$ a<b,a=b,a>b $。


\section{乘法}


\end{document}