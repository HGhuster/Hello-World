\documentclass{article}
\title{净现值和投资评价的其他方法}
\author{Dawei Wang}
\date{\today}
\usepackage{ctex}
\usepackage{amsmath}
\usepackage{amssymb}
\begin{document}
	\maketitle
\section{净现值法则(NPV rule)}

接受净现值大于0的项目,拒绝净现值为负的项目。

\hspace*{\fill}

净现值法的特点:

1.使用了现金流量

2.包含了项目的全部现金流量

3.对现金流量进行了合理的折现

\section{回收期法(payback)}
回收期法的决策过程。选择一个具体的回收期决策标准,比如两年,那些回收期等于或小于2年的项目都可行,大于2年的都不可行。

回收期法的问题:

1.回收期内现金流量的时间序列

2.关于回收期以后的现金支付

3.回收期法决策依据主观臆断

\section{折现回收期法(discounted payback period method)}
先对现金流量进行折现,然后求出达到初始投资所需要的折现现金流量的时间长短。

\section{内部收益率(internal rate of return,IRR)法}
内部收益率是使项目净现值为0的折现率。

若内部收益率大于折现率,项目可以接受;若内部收益率小于折现值,项目不能接受。

注:计算内部收益率不需要用到折现率,但运用内部收益率法也要用到折现率(作为比较项)

\section{内部收益率法存在的问题}
\subsection{独立项目与互斥项目的定义}
独立项目(independent project),就是对其接受或者放弃不受其他项目投资决策影响的投资项目。

互斥项目(mutually exclusive investment),不能同时接受的项目。

在首期付出现金的为投资型项目,在首期收到现金的为融资型项目。投资型项目是内部收益率应用的一般模型,内部收益率法则遇到融资型项目时出现悖反。

\hspace*{\fill}

多个收益率

现金流量出现k次变号可能会有k个内部收益率

\hspace*{\fill}

修正内部收益率

通过合并现金流(负的现金流折现到初期),使现金流的正负号只改变一次来处理多个内部收益率问题。

\subsection{互斥项目所特有的问题}
规模问题:NPV小的项目可能因为规模小IRR大,NPV大的项目可能因为规模大IRR小。

互斥项目的三种决策方法(不能把两个项目的内部收益率直接拿来比较):

1. 比较净现值

2. 计算增量净现值(大预算减小预算(以满足投资型项目的要求)的现金流量增量的净现值)

3. 比较增量内部收益率与折现率(大预算减小预算(以满足投资型项目的要求)的现金流量的内部收益率与折现率比较)

\section{盈利指数(profitability index,PI)法}
盈利指数(PI)=初始投资所带来的后续现金流量的现值/初始投资

1)独立项目:

对于独立项目,若PI>1,可以接受;若PI<1,必须放弃

2)互斥项目:

由于互斥项目规模不一致,所以不能直接比较PI。(可以计算增量现金流量的PI。)

3)资本配置:

当资金不足以支付所有净现值为证的项目时,需要进行资本配置(capital rationing)

在资金有限的情况下,不能仅仅依据单个项目的净现值进行排序,而应该根据现值与初始投资的比值进行排序。这就是盈利指数法则。
\end{document}