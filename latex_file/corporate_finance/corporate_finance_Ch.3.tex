\documentclass{article}
\title{财务报表分析与财务模型}
\author{Dawei Wang}
\date{\today}
\usepackage{ctex}
\usepackage{amsmath}
\usepackage{amssymb}
\begin{document}
	\maketitle
\section{财务报表分析}
\subsection{报表的标准化}
由于规模的差异,直接比较两家公司的财务报表几乎是不可能的。用百分比代替绝对值(把报表中的每个项目都表示成总资产的百分比),由此产生的报表就被称为共同比报表(common-size statement)。
\subsection{共同比利润表}
将利润表进行标准化的一种方法就是将所有的表项都以总销售额的百分比表示。

盈余的度量指标:

净利润

每股收益:净利润/已发行在外总股份数

EBIT:经营性利润,经营活动总收入减去经营活动总成本,EBIT排除资本结构(利息支付)和税收的影响,单独提取盈余差异部分。

EBITDA:EBIT+折旧和摊销(摊销理解为对无形资产的“折旧”,折旧应用于有形资产),EBITDA将两项非现金项目加入EBIT中。

\section{比率分析}
财务比率通常分为以下几个方面:

反映短期偿债能力的比率,即流动性比率

反映长期偿债能力比率,财务杠杆比率

反映资产管理情况的比率,周转率

反映盈利能力的比率

反映市场价值的比率

\subsection{短期偿债能力与流动性指标}
短期偿债能力比率是一组旨在提供企业流动性信息的财务比率,有时也被称为流动性指标。

流动比率=流动资产/流动负债

速动(酸性实验)比率=(流动资产-存货)/流动负债

现金比率=现金/流动负债

\subsection{长期偿债能力指标}
财务杠杆比率或者杠杆比率

负债比率(toatl debt ratio)=(总资产-总权益)/总资产

负债权益比=总负债/总权益

权益乘数=总资产/总权益

权益乘数=1+负债权益比

利息倍数(利息保障率times interest earned ratio)=EBIT/利息

现金对利息的保障倍数(cash coverage ratio)=EBITDA/利息	

有息负债/EBITDA (界限1、5)

\subsection{资产管理或资金周转指标}
资产管理比率或资产利用比率用于衡量资产的周转率,旨在说明公司能在多大程度上有效地运用资产获得收入。

存货周转率(inventory turnover)=产品销售成本/存货

存货周转天数(days' sales in inventory,平均的存货周转时间)=365天/存货周转率

应收账款周转率(receivables turnover)=销售额/应收账款

应收账款周转天数(days' sales in receivables)=365天/应收账款周转率

应付账款周转率(payables turnover)=产品销售成本/应付账款

总资产周转率(total asset turnover)=销售额/总资产

\subsection{盈利性指标}
销售利润率=净利润/销售额

息税、折旧摊销前利润率=EBITDA/销售额

资产收益率(return on assets,ROA)=净利润/总资产

权益收益率(return on equity,ROE)=净利润/总权益

\subsection{市场价值的度量指标}
EPS(每股收益)=净利润/发行在外的股票数

PE(price-earning ratio,市盈率)=每股价格/每股收益————企业成长机会越大,PE越高。

市值面值比(market-to-book ratio)=每股市价/每股账面价值

公司市值=每股股价×发行在外股份数

企业价值(EV,收购企业要花的钱)=公司市值+有息负债的市值-现金

企业价值乘数(EV乘数)=EV/EBITDA————企业成长机会越大,EV乘数越高。

\section{杜邦恒等式}
ROA和ROE之间的差异反映了债务融资或财务杠杆的运用。
\subsection{透视ROE}
ROE=ROA×权益乘数

杜邦恒等式:ROE=利润率×总资产周转率×权益乘数

杜邦恒等式说明,ROE受三个方面因素影响:

1.经营效率(以销售利润率度量)

2.资产运用效率(以总资产周转率度量)

3.财务杠杆(以权益乘数度量)

按杜邦恒等式,增加公司负债似乎有益于提高ROE,但增加负债的同时利息支出增加会降低销售利润,并进而引起ROE的降低。

\section{财务模型}
\subsection{一个简单的财务计划模型}
不需要外部资金时:

销售增长与财务政策之间的相互作用关系。当销售额增加时,总资产也增加,因为公司必须投资于净营运资本和固定资产以支撑更高的销售水平,随着资产的增加,资产负债表右边的负债与权益也要增加。此时股利或负债将成为资产负债表的调节变量。
\subsection{销售百分比法}
将利润表和资产负债表的项目分成两组,一组直接与销售额挂钩,另一组与销售额不直接相关。

\hspace*{\fill}

股利支付率(dividend payout ratio)=现金股利/净利润

留存比率(retention ratio)=留存收益的增加额/净利润

资本密集率(capital intensity ratio)=总资产/销售额————总资产周转率的倒数

若资本密集率固定

EFN()=资产/销售额×$ \Delta $销售额-自发增长的负债/销售额×$ \Delta $销售额-PM×预计销售额×(1-d)

其中PM为利润率,d表示股利支付率

\section{外部融资与增长}
\subsection{EFN与增长}
公司面临现金冗余或短缺取决于其增长(增长率高,现金短缺,增长率低,现金冗余)。

内部增长率(internal growth rate)——没有外部融资的情况下能实现的最大增长率:
\[
\frac{ROA\times b}{1-ROA\times b}
\]

可持续增长率(sustainable growth rate)——没有外部股权融资且保持负债权益比不变的情况下可能实现的最高增长率:
\[
\frac{ROE\times b}{1-ROE\times b}
\]

增长的决定因素:

任何导致ROE上升的因素都会通过使可持续增长率的分子更大和分母更小,从而导致可持续增长率上升,那些提高利润再投资率的因素也是如此。

一家公司的持续增长能力直接取决于以下四个因素。

1. 销售利润率

2. 股利政策

3. 融资政策

4. 总资产周转率

\subsection{可持续增长率的一个说明}
若总权益取自期初资产负债表,则可持续增长率用ROE×b来算是正确的。若期末则之前的描述是正确的。

growth rate g = retention rate b * Return on Equity ROE

image as followings:
year 1: your net income is NI, the portion that is not distributed is:  NI * b; this is the total amount to be invested at the end of year 1
next year: you invested the kept portion of NI*b, the return on it is NI*b*ROE, again you are going to kept some portion, NI*b*ROE*b, which will not be distributed. Considering the amount you already have had in year 1, now totally you have NI*b + NI*b*ROE*b, at the end of year 2 that can be invested in the following year

Thus, the growth rate g is: [(NI*b+NI*b*ROE*b) - NI*b] / NI*b =  b*ROE



\end{document}