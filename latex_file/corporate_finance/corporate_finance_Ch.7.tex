\documentclass{article}
\title{风险分析、实物期权和资本预算}
\author{Dawei Wang}
\date{\today}
\usepackage{ctex}
\usepackage{amsmath}
\usepackage{amssymb}
\begin{document}
	\maketitle
\section{敏感分析、场景分析和盈亏平衡分析}
\subsection{敏感性分析和场景分析}
用来检测某一特定NPV计算对特定假设条件变化的敏感度。

标准的敏感分析,是假定其他变量处于正常估计值,计算某一变量三种不同状态下可能估出的NPV。

\subsection{敏感性分析的作用}

1.可以表明NPV分析是否值得信赖

2.可以指出哪些方面需要多搜集信息

\hspace*{\fill}

不足:

更容易导致“安全错觉”;

只是孤立地处理每个变量的变化,而实际上不同变量之间可能是相互关联的。

\subsection{场景分析}

考察一些可能出现的不同场景,每个场景综合了各种变量的影响。

\subsection{盈亏平衡分析}
确定公司盈亏平衡时所需要达到的销售量,是敏感性分析方法的有效补充。

会计盈亏平衡点:净利润为零(纳税额为零)时的销售额

(固定成本+折旧)/(销售单价-单位变动成本)

\hspace*{\fill}

财务盈亏平衡点:使NPV为零的销售量

[EAC+固定成本×(1-t)-折旧×t]/[(销售单价-单位变动成本)×(1-t)]
 
其中EAC为初始投资的EAC,固定成本为每年需要支付的固定成本。
 
\hspace*{\fill}

财务盈亏平衡点和会计盈亏平衡点的区别在于会计盈亏平衡点忽略了初始投资的机会成本。
 
\section{蒙特卡罗模拟}
\subsection{构建基本模型}
\subsection{确定模型中每个变量的分布}
\subsection{通过计算机抽取一个结果}
\subsection{重复上述过程}
\subsection{计算NPV}

\section{实物期权}
\subsection{拓展期权}
业务好可以进行业务拓展,实现更大NPV。
\subsection{放弃期权}
业务不好可以中止业务,及时止损。
\subsection{择机期权}
可以进行投机。

\section{决策树}


\end{document}