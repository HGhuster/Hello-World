\documentclass{article}
\title{折现现金流量估价}
\author{Dawei Wang}
\date{\today}
\usepackage{ctex}
\usepackage{amsmath}
\usepackage{amssymb}
\begin{document}
	\maketitle
投资的净现值:

NPV=-成本+PV

\hspace*{\fill}

投资的终值:

$ FV=C_0\times (1+r)^T $

\hspace*{\fill}

投资的现值:

$ PV=\frac{C_T}{(1+r)^T} $

\hspace*{\fill}

一笔在T期后产生效益的投资项目的净现值:

$ NPV=-C_0+\frac{C_1}{1+r}+\frac{C_2}{(1+r)^2}+\cdots+\frac{C_T}{(1+r)^T}=-C_0+\sum_{i=1}^{T}\frac{C_i}{(1+r)^i} $

\hspace*{\fill}

名义年利率(annual percentage rate,APR)和实际年利率(effective annual rate,EAR)之间的关系:
$ EAR=(1+\frac{APR}{m})^m-1 $

m为每年计息次数

\hspace*{\fill}

连续复利

$ C_0\times e^{rT} $

\hspace*{\fill}

年金:

$ PV=C[\frac{1-\frac{1}{(1+r)^T}}{r}] $

\hspace*{\fill}

注:

1.递延年金:某人在第六年后的四年内每年都会收到500美元。如果利率为10\%,那么他的年金的现值为多少?

先计算第五年年初时年金的现值,再将其贴现到现在(5年期)。

\hspace*{\fill}

2.若年金的第一次支付发生在第0期,其价值如何计算?

分别计算第0期和其他期的和。

\hspace*{\fill}

3.不定期年金,支付频率超过1年。

计算每期间隔的实际利率。

\hspace*{\fill}

4.设两笔年金现值相等。写出现金流的时间点并分析。

\hspace*{\fill}

\section{增长年金}
增长年金现值的计算公式
$ PV=C[\frac{1}{r-g}-\frac{1}{r-g}\times(\frac{1+g}{1+r})^T]=C[\frac{1-(\frac{1+g}{1+r})^T}{r-g}] $
\end{document}