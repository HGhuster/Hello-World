\documentclass{article}
\title{投资决策}
\author{Dawei Wang}
\date{\today}
\usepackage{ctex}
\usepackage{amsmath}
\usepackage{amssymb}
\begin{document}
	\maketitle
\section{增量现金流量:资本预算的关键}
\subsection{现金流量而非“会计利润”}
与财务会计不同,公司理财注重现金流量,因此NPV是对现金流量的贴现(而不是对利润),从整体评价一家公司时,是对股利而不是利润折现(股利是股东收到的现金流量)。

在讲现金流量时,所运用的现金流量应该是因项目而产生的现金流量“增量”。这些现金流量是公司接受一个项目而引发的直接后果——现金流量的变化。
\subsection{沉没成本(sunk costs)}
沉没成本是已经发生的成本。由于沉没成本是在过去发生的,它不因接受或者摒弃某个项目的决策而改变。沉没成本不属于增量现金流量。

\subsection{机会成本(oppotunity costs)}
因为接受新项目,公司丧失了其他利用这项资产的机会所产生的成本叫做机会成本。

\subsection{副效应}
新增项目可能对公司原有项目产生副效应。

副效应可以分为侵蚀效应(erosion)和协同效应(synergy)。侵蚀效应是指新项目减少原有产品的销量和现金流。协同效应是指该新项目同时增加了公司原有项目的销量和现金流。		

\subsection{成本分摊}
在投资预算中,成本分摊只能是当该现金流作为一个项目的增量现金流时,才能计入该项目中。

\subsection{净营运资本计算的一个注解}
在如下情况下会产生对净营运资本(NWC)的投资:1.存货采购2.为不可预测的支出而在项目中保留的作为缓冲的现金3.当发生了赊销,产生的不是现金而是应收账款。对NWC的投资代表现金流出,因为从公司其他地方产生的现金被此项目占用了。

\section{通货膨胀与资本预算}
\subsection{利率与通货膨胀}
1+名义利率=(1+实际利率)×(1+通货膨胀率)

\subsection{现金流量与通货膨胀}
名义现金流量指实际收到或支出的美元,实际现金流量指该现金流量的实际购买力。

\subsection{折现:名义或实际}
名义现金流量应以名义利率折现。

实际现金流量应以实际利率折现。

\section{经营性现金流量(OCF)的不同算法(不考虑利息)}
\subsection{自上而下法}
经营性现金流(OCF)=销售收入-现金成本-税

\subsection{自下而上法}
OCF=净利润+折旧=(销售收入-现金支出-折旧)(1-t)+折旧

\subsection{税盾法}
OCF=(销售收入-现金成本)×(1-t)+折旧×t

\section{不同生命周期的投资:约当年均成本法,equivalent annual cost}
假设公司必须在两种不同生命周期的机器设备中做出选择。两种机器设备功能是一样的,但它们具有不同的经营成本和生命周期。简单地运用NPV法则就意味着我们应该选择其成本巨有最小现值的机器设备,然而,这种判断标准会造成错误的结果,因为成本较低的机器设备重置的时间可能早于另一种机器设备。

EAC=成本现值/年金系数

\hspace*{\fill}

注:

第一,要把成本现金流折现到0期;

第二,这种方法只有在两种机器都是可替换的时候才有效

\hspace*{\fill}

更常见的情形是,公司需要决定何时以新设备来替换旧设备。重置应该在旧设备的约当年均成本超过新设备的约当年均成本之前发生。

必须计算保留旧设备一年、两年、三年...的EAC来和新设备的EAC比较,选取成本低的。


\end{document}