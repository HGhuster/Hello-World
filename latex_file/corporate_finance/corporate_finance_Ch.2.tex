\documentclass{article}
\title{会计报表与现金流量}
\author{Dawei Wang}
\date{\today}
\usepackage{ctex}
\usepackage{amsmath}
\usepackage{amssymb}
\begin{document}
	\maketitle
\section{资产负债表}
 资产$\equiv$ 负债 +所有者权益 

\hspace*{\fill}

资产负债表中的资产按持续经营的企业的资产正常变现所需的时间长短顺序排列(取决于企业的行业性质和管理行为)。负债和所有者权益按偿付的先后顺序排列(取决于管理者对资本结构的选择)。

\hspace*{\fill}


分析资产负债表时,财务管理人员应注意三个问题:流动性、债务与权益、市价与成本。

\subsection{流动性(liquidity)}
流动性是指(在不引起价值大幅损失的前提下)资产变现的方便与快捷程度。

\hspace*{\fill}

流动资产(current assets)的流动性最强,它包括是指现金以及自资产负债表编制日起一年内能够变现的其他资产,包括应收账款(accounts receivable)和存货(inventory)。固定资产(fixed assets)是指流动性不高的一类资产,有形的包括房产、厂房、设备等。无形的包括商标、专利等。

\hspace*{\fill}

资产流动性越大,对短期债务的偿还能力越强。(高流动性通常意味着低收益率)。

\subsection{负债与权益}
负债(liabilities)是指企业所承担的在规定的期限内偿付现金的责任。

\hspace*{\fill}

所有者权益(stock-holders' equity)是对企业剩余资产的索取权。

\hspace*{\fill}

当企业将部分利润留存而不作股利发放时,留存收益增加,所有者权益的会计价值也随之提高。

\subsection{市价与成本}
企业资产的会计价值通常指置存价值或账面价值(book value实际上是成本)。

\hspace*{\fill}

市场价值(market value)是指有意愿的买者与卖者在资产交易中所达成的价格。

\hspace*{\fill}

财务管理者的目标是提升股票的市场价值。

\section{利润表(income statement)}
利润的会计定义为:

收入-费用$ \equiv $利润

\hspace*{\fill}

息税前利润(EBIT),反映扣除所得税和融资费用之前的利润。

\hspace*{\fill}

分析利润表时应注意:公认会计准则、非现金项目、时间与成本。

\subsection{公认会计准则}

\subsection{非现金项目(noncash item)}
折旧、递延税款(由会计利润和实际应纳所得税之间的差异引起)。当期税款向税务机关缴纳,而递延税款则不必。若本年的应税所得小于会计利润,以后年度的应税所得大于会计利润,这意味着本年的部分税款将在以后的年度付出,这就称为企业的负债,在资产负债表上表现为递延税款。从现金流量的角度看,递延税款不是一笔现金支出。

\subsection{时间与成本}
短期内成本有固定成本和非固定成本之分。
短期内特定的设备、资源和责任义务是固定的,但可以通过增加劳动力和原材料来改变产量。


长期来看,所有的成本都是变动的。费用分为产品成本和期间费用。产品成本指某一期间内所发生的全部生产成本,包括直接材料、直接人工和制造费用,产品销售出去后,在利润表上列示为已销产品成本。期间费用包括销售费用、财务费用和管理费用。

\section{税}
公司税额取决于税法,税法准则经常修改。

\subsection{公司税率}

\subsection{平均税率与边际税率}
平均税率等于应纳税额除以应税所得

\hspace*{\fill}

边际税率等于多赚一元需要多支付的税金。

\hspace*{\fill}

一般而言,与财务决策有关的是边际税率,因为新的现金流量总是按照边际税率计税,而财务决策通常涉及新的现金流量或现有现金流量的变动,边际税率可以告诉我们一项决策对应纳税额的边际影响。

\section{净营运资本}
净营运资本等于流动资产减流动负债。

\hspace*{\fill}

企业除了投资于固定资产(资本性支出),还要投资于净营运资本,这就是净营运资本变动额。通常情况下,经营与资本的变动额是正数。

\section{财务现金流量}
从财务的角度看,企业的价值就在于其产生现金流量的能力。

\hspace*{\fill}

资产的现金流量CF(A),等于流向债权人的现金流量CF(B)与流向权益投资者的现金流量CF(S)之和:

$ CF(A)\equiv CF(B)+CF(S) $

\hspace*{\fill}

企业资产所产生的现金流量等于

经营性现金流量-资本性支出-净营运资本增加额

\hspace*{\fill}

其中经营性现金流量=EBIT-税+折旧(反映经营性活动所带来的现金)

资本性支出=固定资产的取得-固定资产的出售=期末固定资产净额-期初固定资产净额+折旧

净营运资本增加=期末营运资本-期初营运资本

\hspace*{\fill}

向债权人支付的现金数量=支付的利息-净新借入额=支付的利息-(期末长期债务-期初长期债务)

向股东支付的现金流量=支付的股利-权益筹资净额=支付的股利-(发行的股票-回购的股票)

\section{会计现金流量表}
现金流量表由三个部分组成:经营活动产生的现金流量(由于生产和销售产品以及提供劳务等正常经营活动所带来的现金流量)、投资活动的现金流量、筹资活动产生的现金流量(当年对债权人和所有者的净支出(不包括利息费用))。

\hspace*{\fill}

利息费用在会计处理时,由于计算净利润时利息被当作一项费用扣减了。因此,会计现金流量表与财务现金流量之间的主要差异在于利息费用。

\end{document}