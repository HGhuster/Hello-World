\documentclass{article}
\title{宏观经济学概述}
\author{Dawei Wang}
\date{\today}
\usepackage{ctex}
\usepackage{amsmath}
\usepackage{amssymb}
\begin{document}
	\maketitle
\section{宏观经济学的产生与发展}
\subsection{西方经济学的演化规律}
古典经济学危机后,出现了对它的革命——边际革命,边际革命后马歇尔把这两种截然对立的思想综合为一体,称为新古典经济学。20世纪30年代又出现了对新古典经济学的信任危机,导致其后的凯恩斯革命,凯恩斯革命后经济学家萨缪尔森又把这两种思想综合为一体。

\subsubsection{古典经济学的危机和边际革命}

1. 经济学史前——价值论的萌芽

经济学起源于对价值来源问题的探讨。

经济学的史前阶段对价值来源、财富产生的认识,称为重商主义。重商主义认为价值(财富)产生于流通领域。

\hspace*{\fill}

2. 经济学的创世纪

亚当·斯密:经济学之父,著作《国民财富的性质和原因的研究》(《国富论》,经济学三部圣经之一),劳动价值论的提出者。

\hspace*{\fill}

3. 约翰·穆勒的综合

约翰·穆勒的《政治经济学原理》(第一本有里程碑意义的经济学教科书)全面吸收和综合了前人的理论。

\hspace*{\fill}

4. 古典经济学的危机

古典经济学的危机源于对一个很小的经济问题不能解释——新老葡萄酒的价值问题。一瓶刚出场的新葡萄酒的价格也就是十几块、几十块,而一瓶有百年历史的老葡萄酒的价格是几万甚至十几万,一瓶老酒的价格是一瓶新酒价格的一千倍甚至一万倍。原因何在?按照劳动价值论,商品的价值取决于凝结在商品中的无差别的人类劳动。从葡萄酒的生产过程看,凝结在一瓶葡萄酒中的古人劳动和现代人劳动几乎是一样的。为什么价值就相差成千上万倍呢?

假定不存在保存成本,如果认为老酒比新酒贵的理由在于老酒少,物以稀为贵,显然是站在劳动价值论的对立面——边际主义的立场上来解释这个问题。

\hspace*{\fill}

5. 边际革命

边际三杰:奥地利经济学家门格尔(Menger)、英国经济学家杰文斯(Jevons)、瑞士洛桑派的法国经济学家瓦尔拉斯(Leon  Walras)。

边际三杰几乎同时提出边际效用价值论,开始所谓的“边际革命”。这一理论反对劳动价值论,认为商品的价值不是取决于商品中所包含的客观劳动量,而是取决于人们对商品效用的主观评价,这是一种与古典经济学家截然对立的主观价值论。

边际主义者强调产品和生产要素市场上的定价与消费者主观的边际效用相关,由于存在边际效用递减,因此对各种产品效用的主观心里评价就会存在递减。主观、心理的评价决定商品的价值。这是在世界观上对古典经济学劳动价值论的革命。

\hspace*{\fill}

6. 对经济学中的革命的反思

劳动价值论被马克思所继承,称为马克思主义的三大来源之一,是1867年发表的《资本论》的基础。《资本论》被称为经济学的三部圣经之二。

\subsubsection{马歇尔的综合和新古典经济学的形成}

马歇尔(Alfred Marshall)在1890年出版的第二本有里程碑意义的经济学教科书《经济学原理》中提出具有首创意义的均衡价值论,使他称为经济学说史上第一个真正的集大成者。

马歇尔把主观的、心理的边际效用价值论归结为决定需求、消费、买方的力量。消费者在购买商品的时候会考虑一个问题,那就是这个东西值不值?所以买者决定是否购买一个商品就是由主观、心理的边际效用价值决定的。商品越少,消费者的边际效用越高,愿意支付的价格越高。而决定生产、供给、卖方背后深层次的原因是生产者对成本的衡量,归结为客观存在的劳动价值论的力量。当两种力量完全相等时,就形成相对静止、不再变动的均衡状态,产生均衡价格。

马歇尔把主观的、心理的边际效用价值论和客观的劳动价值论综合为一体,形成均衡价值论。均衡价值论将原本完全对立的世界观、价值论综合为一体。在方法论上,马歇尔在坚持李嘉图演绎推理的同时,广泛运用边际分析。因此马歇尔的综合是“世界观的而综合、价值论的综合和研究方法的综合”。

使边际革命者和古典经济学统一起来后,西方经济学出现了“新古典经济学”。

新古典经济学同样把自由放任作为最高准则,但已不像古典经济学那样只重视对生产的研究,而是转向了消费、需求,它把资源配置作为经济研究的中心,论述了价格如何使社会资源配置达到最优。

\subsubsection{新古典经济学的危机和凯恩斯革命}
新古典经济学一直坚信资本主义经济通过市场上得自由竞争总会自动调节达到充分就业得均衡状态,因此不可能发生普遍性生产过剩得经济危机。

微观经济学是建立在演绎基础上的一种理论,其特征为从基本前提假设出发,通过数学演绎推理,得出结论。

微观经济学的基本前提假设:

1. 理性人假设

理性人假设=经济人假设=利益最大化原则

经济学中的理性人以是否追求利益最大化为标准。追求利益最大化就是理性的,不追求利益最大化就是非理性的。

在市场经济中,光有追求利益最大化的主观动机是不够的,还需要其他的一些客观条件。

\hspace*{\fill}

2. 信息完全假设

这一假设条件的主要含义是指市场上每一个从事经济活动的个体都对有关的经济状况具有完全的信息。 对社会来讲,需求与供给双方是通过两个方面对价格对价格的作用而相互联系相互适应的,市场价格在这里作为联系供给和需求的指标,起着中介的作用。

价格机制完美地体现了20世纪自然科学的重要成果——旧三论(信息论、控制论、系统论)

\hspace*{\fill}

(1)价格机制是传递供求信息的经济机制

价格机制=市场机制

第二个基本假设——自由价格假设可以部分地从第一个基本假设——最大化原则推导出来。但最大化原则只是自由价格原则的必要非充分条件,即最大化原则只是自由价格的成因之一。这种对价格机制来讲处于第一位的,或者说最重要的作用显然要受到各种各样的客观条件约束,例如,垄断、工会、国家价格管制等。

完全竞争假设,建立在第二大基本前提假设之上的子命题:在完全竞争条件下,自由价格反映了全部的信息。

完全竞争假设包含了三个方面的含义:

第一,市场上有无数的买者和卖者;

第二,产品同质;

第三,要素自由流动。

(2)价格机制控制经济人、理性人、最大化的追求者(控制论)

价格机制=市场机制=看不见的手

价格机制控制我们成千上万人的行为,刺激追求利益最大化的消费者购买最物美价廉的商品,刺激追求利益最大化的生产者生产最供不应求的商品,并且按照最有效率、最能够实现利润最大化的方式进行生产,刺激追求利益最大化的劳动力阶层追求待遇高、福利好的职业。只要你追求利益最大化,你的行为就不可避免地要受到它的控制,价格机制体现了控制论的特点。

(3)价格机制协调整个社会的系统(系统论)


3. 市场出清假设

市场出清假设是建立在最大化原则与自由价格假设上的假设,与之前的两个基本假设有明确的因果关系。

出于主观为自己的动机的个人,在“看不见的手”(市场机制/价格机制)的作用下,产生了客观为他人的社会效果。以追求个人私利、利己为本心的行为,其结果是达到利他的社会公益。价格可以使个人理性和社会理性达到一致。

亚当·斯密的上述思想发展为一个更加精致的“定理”:给定价格等限制条件,消费者和厂商的最大化行为,将导致市场均衡状态。

市场均衡的含义:

第一,在市场均衡下,每个人或企业都实现了各自的利益最大化,这是市场均衡的最基本含义。

第二,市场均衡意味着所有经济人的行为都是相互协调的、相互兼容的;作为他们共同行为结果的交换比率即市场价格,使供给和需求达到相等,即市场“出清”。

瓦尔拉斯把局部均衡扩展到一般均衡。

现代西方经济学家运用数学工具证明瓦尔拉斯一般均衡体系只有在极其严峻的假设条件下才可能存在均衡解。

从前两个基本前提假设出发,将会推导出第三个。如果主观上追求自己的利益最大化,客观上有健全的信息传递机制,最终市场运行的结果将达到一般均衡,即市场出清。最后一个环节就是从福利经济学角度论证一般均衡、市场出清这样一种状态属于资源配置的帕累托最优。

瓦尔拉斯的一般均衡理论证明:供求相等的均衡不但可以存在于单个市场,而且还可以同时存在于所有市场。因而劳动力是充分就业的,价格的变动使劳动力市场供求相抵,不存在非自愿失业,充分就业是市场出清假设的内涵之一。

\hspace*{\fill}

大萧条导致社会思潮的巨大变迁,1936年凯恩斯出版了《就业、利息和货币通论》,简称《通论》。它是国家干预主义的开山之作,是经济学发展史上三部圣经之三,标志着现代宏观经济学的产生。

凯恩斯的《通论》正是当时新古典经济理论危机的产物。凯恩斯的革命非常彻底,对新古典经济学的三个基本前提假设全部抛弃,并反其道而行之,他认为:

第一:人不是趋利避害的完全的理性人。尤其是在突发经济状况下,人会丧失理性,产生从众心理、羊群行为。

第二:信息不完全。由于垄断的出现,导致价格机制不能快速准确地传递全部的供求信息。

由于前两个基本条件不成立,因此第三个基本条件就无法推导出来。

由于三大心理规律(消费边际倾向规律、资本边际效率规律和流动性偏好规律)地作用导致有效需求不足,从而必然产生大规模失业、生产过剩地经济危机,凯恩斯从理论上说明了资本主义市场经济的不稳定因素和非均衡趋势。在方法论上,凯恩斯回到了重商主义研究的宏观经济问题,开创了宏观经济的分析方法,即总量分析,还将实物经济和货币经济密切结合为一体,从而克服了传统经济学在分析过程中运用“二分法”的不一致性。

在政策上,凯恩斯反对新古典经济学的“自由放任”,凯恩斯的反危机政策有三个特点:国家调节和干预经济生活是前提;财务政策是其重心;举债支出是其手段。主张利用财政政策、货币政策手段恢复均衡,为国家干预经济提供了理论上的支撑。

\hspace*{\fill}

(四)萨缪尔森的综合和新古典综合——凯恩斯主义的形成

面对凯恩斯对新古典经济学的革命,萨缪尔森于1948年出版了《经济学》——第三本具有里程碑意义的经济学教科书,开始了经济学上的第三次综合。

1. 理论上的综合

萨缪尔森首先研究了两种理论的适用条件。他认为如果没有实现充分就业就应该用凯恩斯的理论,用国家干预财政、货币手段来恢复均衡。在经济实现了充分就业后就应该用新古典经济学来研究问题,用市场这只看不见的手来调剂供求,发挥作用。

萨缪尔森阐明了新古典经济学理论和其革命者——凯恩斯理论的逻辑联系:这是适用于两种不同条件下的理论,因此,两者可以并存,整个凯恩斯的和新古典的理论可以结合在一起,并首创“新古典综合”一词来概括这种理论体系上的结合。所以“新古典综合”是萨缪尔森的首创,代表了他“混合经济”的思想,即任何的经济都是政府干预与市场经济的混合。

2. 形式上的综合

从形式上看,“随着凯恩斯的理论命名为宏观经济学,新古典经济学则以微观经济学的新名称而出现”。

新古典综合论几乎就等同于凯恩斯主义,成为战后西方经济学的正统,并且成为政府干预经济的理论基础。

\subsection{微观和宏观经济学在经济学课程中的地位}
核心理论

核心理论是经济学全部的世界观和基础,是支持其他经济学课程的框架。核心理论包括两门课程:一个是微观经济学,前身是新古典经济学;另外一个是宏观经济学,包括凯恩斯经济学加其他流派。经济学的全部世界观都包含在这两门课程中。
\section{宏观经济学的研究对象}
\subsection{宏观经济学的研究对象}
1. 宏观经济学研究的对象就是宏观经济的波动和增长。

宏观经济学研究的第一个问题就是宏观经济学波动,主要解决怎么从失衡状态——供不应求或供过于求,恢复到均衡状态——供求相等的问题。宏观经济学波动问题,就时间跨度来说,是3年、5年,最多不超过10年。

\hspace*{\fill}

2. 宏观经济增长问题

宏观经济增长问题是均衡的长期化和动态化问题,即已经实现均衡,在一个很长的时期内(30—50年),怎么维持这种状态?宏观经济增长问题,就时间跨度来讲,是30年、50年,甚至100年的时间范围。

\subsection{度量宏观经济的量化指标}
一个量化的指标体系包括:

1.国民收入

国民收入用GNP表示,或者用Y表示,是度量宏观经济波动程度的第一个也是最重要的指标,运用这个指标可以进行纵向比较和横向比较。

纵向比较:反映同一个国家在不同时期经济发展的快慢。

横向比较:反映在同一个时期不同国家经济发展的快慢。

2. 物价水平

在收入既定的情况下,如何衡量收入的购买力,取决于物价水平P。名义收入是一种衡量,实际收入又是一种衡量。物价水平同样可以进行纵向比较和横向比较。

纵向比较:比较同一国家在不同时期通胀率的差异。

横向比较:比较不同国家在同一时期的通胀水平。

3. 就业率

就业率用N代表。

纵向比较:我国1980年前无失业,而90年代后失业率大幅上升。

横向比较:有的国家失业率始终很低,有的国家则正好相反。

4. 国际收支情况

国际收支情况用净出口(NX)代表。

净出口(NX)=出口(X)-进口(M)

\hspace*{\fill}

以上四个指标同时也是政府政策的四个最终目标:经济增长——针对Y,物价稳定——针对P,充分就业——针对N,国际收支平衡——针对NX。

四个目标中,最重要的是国民收入和物价水平,这两者构成了一个坐标系:P—Y坐标系。

就业量并非不重要,而是可以从国民收入——总产量指标中推算出来。

总产量=f(就业量,资本量)

就业是一种投入——劳动要素的投入,产出就是这个国家GNP的水平。通过GNP指标,可以倒推出在某个水平之下可以吸收的就业。所以就业量和GNP水平之间有投入和产出的关系,从第一个指标可以推算出第三个指标。

\subsection{宏观经济波动的根源}
衡量宏观经济波动的四个指标体现了波动的结果,是因变量,而导致波动的根源需要进一步的讨论。

政策变量(货币政策、财政政策、其他政策)、外生变量(战争、气候)、其他变量(消费、投资)——三大变量进入宏观经济学中,其结果就导致国民收入、价格、就业率、净出口出现不同的数值。宏观经济学研究的是导致这些结果的起因是什么,发生什么样的变化会导致这些结果的出现。所以在宏观经济学中,既要把根源找出,又要研究传导机制。(破译黑箱)

\subsection{宏观经济学的特征事实}
判断宏观经济理论模型——收入-支出模型、IS-LM模型、总需求——总供给模型——好坏的标准,主要看它们解释特征事实的能力。

特征事实事是指在宏观经济学中广泛存在的规律性,是经济学家根据时间序列的统计数据而得出的检验真理的试金石。

美国经济波动的八个特征事实。

(1)在经济的各部门之间的产量变动是相关的。

(2)工业生产、消费和投资是顺周期的,可以同时变动。其中政府购买也是顺周期的。(顺周期就是同GNP同方向变动)

(3)在经济周期的过程中,耐用品消费有强烈的顺周期性,而投资的变动性远远大于消费。投资比消费有更大的易变性。

(4)就业是顺周期的,失业是逆周期的。

(5)实际工资和平均劳动生产率是顺周期的,尽管实际工资只是轻微地顺周期。

实际工资=名义工资/P,平均劳动生产率是一个国家的总生产量/劳动力总数。

(6)货币供给和股票价格是顺周期的,而且是超前的。

(7)通货膨胀率和名义利率是顺周期的,而且是滞后的。

名义利率是一个国家银行里公布的利率。一个国家往往先出现经济过热,然后出现物价水平、名义利率的上升。

(8)实际利率是非周期性的。

利率是货币的价格,实际利率衡量了货币资产的实际价值。

按照费雪效应:实际利率=名义利率-通胀率

非周期性是与GNP没有显著的正向或反向关系。

\section{宏观经济学的研究方法}
\subsection{宏观经济的交易市场}
1. 产品市场

涉及衣食住行方方面面成千上万种产品交易的有形的交易市场就是产品市场。有些部门出售的是无形的劳务。有形的物品和无形的劳务都在产品市场上交易。

2. 货币市场

产品流的背后对应的是货币流,设计的是金融资产市场。例如,股票、债券、货币等也是我们财富的持有形式。

3. 劳动力市场

产品生产离不开生产要素。通常讲的生产四要素——土地、劳动、资本、企业家才能——被认为是狭义的要素,它们分别是广义的生产要素的一个典型代表:自然资源、人力资源、人造资源。

在宏观经济学里,对一个国家来讲,认为最重要的生产要素是劳动,所以研究的生产要素市场主要是劳动力市场。

\subsection{宏观经济的市场参与者}
1. 消费者

产品市场需求者+要素市场供给者。

2. 厂商或企业

产品市场的供给者+要素市场的需求者

3. 政府

政府对经济体系的举足轻重的作用:

(1)政府规模很大。

(2)政府机构庞大,涉及经济体系方方面面的事情。

4.外国

主要指外国的消费者。四个经济体又称为四部门,我们分析的思路是从两部门扩展大三部门、四部分。

\subsection{宏观经济的总需求——总供给分析}
把波动根源重新进行一次分类:财政政策、货币政策、消费支出等都是影响经济中总需求方的变量,用AD表示;另外劳动、技术、成本等影响总供给方的因素,用AS表示。

1. 总需求

总需求(agreement demand)是指在价格国民收入和其他经济变量给定的条件下,消费者、厂商、政府和外国愿意支付的数量。总需求分析只涉及产品市场和货币市场。(对国家的总体经济来讲,支出只要愿意就能实现)

2. 总供给

总供给(agreement supply)是指一国的全体厂商在现行价格、生产能力和总成本既定的条件下,愿意而且能够生产和出售的产品数量。总供给分析只涉及劳动力市场。(总供给强调的不仅是“愿意”而且是“能够”,因为生产还要受其他因素制约。)

\subsection{具体分析方法}
1. 文字表述法

由定义、假设、假说、预测构成。

(1)定义:对经济学研究中各种变量规定出明确的含义。

四类变量:

内生变量:可以一个经济体系内得到说明的变量(待求变量)。

外生变量:体系以外的因素决定的变量,影响内生变量(给定变量)。

存量:总量

流量:变动量

(2)假设:一个理论形成的适用条件、前提。

(3)假说:对两个或多个变量之间关系的阐述。

(4)预测:根据假说对未来进行的预测。

\hspace*{\fill}

2. 数学方程法

存在三种方程式:

(1)定义方程式:用方程式下定义。

(2)行为方程式:基本上是函数形式。

在某一个变化过程中,每一个起因都有一个结果与之一一对应,这样一种关系就叫函数关系,称结果是起因的函数,两者之间是一种明确的因果关系。

(3)平衡方程式:表示经济均衡的前提条件。

3. 几何图形法

4. 局部均衡分析和一般均衡分析法

从局部均衡到一般均衡,具体分析从一个市场的均衡向三个市场的均衡推进,是一个逐步放松假设的过程。我们的分析方法从一个市场——产品市场开始,过渡到三个市场——产品、货币、劳动力市场。这个过程中,把外生变量内生化;在一个市场分析中,先从两部门开始,再把经济主体一个一个加进来,从两部门扩展到三部门、四部门,这就是宏观经济学分析的一个总体思路。

5. 静态均衡分析和比较静态均衡分析法

静态均衡分析就是分析均衡的决定,在已知外生变量的条件下,决定内生变量。比较静态均衡分析就是分析均衡的移动。如果外生变量发生变化,那么均衡也会发生变化。先分析均衡的决定,再分析均衡的移动,也就是从静态均衡分析到比较静态均衡分析。

\end{document}