\documentclass{article}
\title{产品市场均衡:收入——支出模型}
\author{Dawei Wang}
\date{\today}
\usepackage{ctex}
\usepackage{amsmath}
\usepackage{amssymb}
\begin{document}
	\maketitle
总需求分析的通用假设前提:

1. 假定经济中存在着生产能力的闲置

生产能力指生产要素,包括:土地、劳动、资本、企业家才能。这里强调劳动和资本。生产能力闲置就表现为:

(1) 劳动力资源没有充分利用——存在失业。

(2) 厂房、机器等资本品没有充分利用——存在着开工率不足。

\hspace*{\fill}

2. 价格水平固定不变(存在着价格刚性)

价格刚性的两种表现:

(1) 在劳动力市场,即便存在失业,工资也不会降低,工资有向下刚性。(新凯恩斯主义:效率工资理论)

(2) 在产品市场,即便存在产能过剩,物价也不会降低,物价具有向下刚性。

\hspace*{\fill}

3. 在既定的价格水平上,总供给是无限的。

既然供过于求,又不能降价。这意味着在一个既定的价格水平上,产品市场的产品供给量是无限的;而在一个既定的工资水平上,劳动力的供给也是无限的,总供给曲线平行于横轴。平行的总供给曲线说明:由于存在着资源闲置,在固定的价格水平$ P_0 $下,供给方没有约束,没有供给瓶颈。在总产量线上的每一点,都有实现均衡的可能。

\hspace*{\fill}

4. 由于总供给无限,因此均衡的国民收入由总需求单方面决定——称为总需求分析

由于在总产量曲线上的每一点都有实现均衡的可能,因此最终均衡的实现由总需求的大小决定。总需求分析强调的是需求方的决定性作用。最终,总需求曲线和总产量曲线交在哪一点,总产量水平就在这一点稳定下来。

\hspace*{\fill}

上述通用假设描述的情况只有在经济大萧条时才能够出现。

通用前提假设集中体现了凯恩斯定律的含义。凯恩斯定律与古典经济学的萨伊定律完全对立。

萨伊定律:供给自己创造自己的需求。其含义为,在任何时候,经济中重要的是供给方,想卖什么,就会有人买什么;想卖多少,就会有人买多少。(此种情况和生产力水平较低有关)

凯恩斯定律:需求创造自己的供给。其含义为,在任何时候,经济中重要的需求方,想买什么,就会有人卖什么;想买多少,就会有人卖多少。(其背景是西方国家进行工业革命,机械化大生产使生产力大幅度提高。东西生产出来完全不成问题,生产出来之后,出现了生产过剩,由此提出了对需求方重视的问题。)

\section{从国民收入核算到宏观经济理论}
\subsection{收入核算(事后分析)}
一个国家的GDP已经生产出来,从事后的角度看,一个国家的总产量恒等于一个国家四个经济体的总支出,即核算恒等式:
\[
AE\equiv Y
\]
经济含义:任何时候的总支出都能够买到所需要的总产量。

收入核算是事后分析,从事后分析的角度得出,总支出等于总产量也等于总供给。

\subsection{均衡条件(事前分析)}
均衡条件为总需求等于总供给。
\[
AD=AS
\]
总供给也可以用Y表示,那么按照收入核算恒等式,关于总供给就可以得到:
\[
AS=Y\equiv AE
\]
综上:
\[
AD=AS=Y\equiv AE
\]

\subsection{总支出与总需求}
总支出AE和总需求AD之间的差别,只能从定义出发:

总需求AD:在价格、国民收入和其他经济变量既定的条件下,消费者、厂商、政府和外国愿意支出的数量。总需求分析只涉及产品市场和货币市场。

四个经济主体“愿意”支出就构成对一个国家强劲的总需求。所以总需求是一种意愿的支出,而总支出是一种实际的支出。

总支出$ AE=C+I+G+NX $

总支出是一个国家的四个经济主体实际花出的钱,把AE中“不愿意”的成分剔除,其余的可计入AD。AE分为四个部分。

\hspace*{\fill}

1. 消费者的消费支出C

C是购买耐用消费品、非耐用消费品以及劳务的支出。消费者的货币选票“愿意”才会投到商品上,所以C全部是愿意的,可计入总需求AD。

2. 国内总投资I

重置投资、固定资产投资、意愿存货统称为计划投资,记为$ I_{ji} $,总投资I=$ I_{ji}+\Delta inv $。计划投资$ I_{ji} $可计入总需求AD,而非意愿存货$ \Delta inv $不能计入总需求AD。

3. 政府购买支出G

这是政府兴办公共工程的开支,以及政府机构的建立、维持和运营的支出,很显然市政府愿意的支出,也可以计入总需求AD。

4. 外国消费者的消费支出

净出口NX等于出口减去进口。国外消费者愿意才会掏钱出来,也可以计入总需求AD。

\hspace*{\fill}

综上:
\[
AD=C+I_{ji}+G+NX
\]
\[
AE=C+(I_{ji}+\Delta inv)+G+NX
\]

总支出和总需求的差距就在于厂商的非意愿存货:$ AE-AD=\Delta inv $。一个国家的经济要实现均衡,总支出应该等于总需求,即$ \Delta inv=0 $。

\subsection{均衡调节机制}
不是每时每刻总供给都等于总需求。Y是变量,每时每刻围绕AD上下波动,如果供不应求或供过于求,有什么力量调节Y使之等于AD?由于基本前提里存在价格刚性,因此价格机制不起作用。


既然总供给和总需求之间就差一个非意愿存货$ \Delta inv $,考虑非意愿存货作为调节变量。

\hspace*{\fill}

(1)当Y($ \equiv AE $)>AD时,此时供过于求,总产量Y大于总需求AD,$ \Delta inv>0 $,存在积压。此时厂商减少存货投资,导致总投资水平下降,引起实际总产量Y减少,从而Y靠近AD,总供求趋向于相等,直到非意愿存货等于0的时候,厂商就不再减少总投资,即:
\[
C+I\downarrow+G+NX\equiv Y\downarrow
\]
直至(AE$ \equiv $)Y=AD,$ \Delta $inv=0。

(2)当Y($ \equiv AE $)<AD时,此时供不应求,总产量Y小于总需求AD,$ \Delta inv<0$,存在脱销。此时厂商增加存货投资,导致总投资水平上升,引起实际总产量Y增加,从而Y靠近AD,总供求趋向于相等,直到非意愿存货等于0的时候,厂商就不再减少总投资,即:
\[
C+I\uparrow+G+NX\equiv Y\uparrow
\]
直至(AE$ \equiv $)Y=AD,$ \Delta $inv=0。

(3)当Y($ \equiv AE $)=AD时,此时供求刚好相等,$ \Delta inv=0 $,既不存在脱销,也不存在积压,厂商既不增加投资,也不减少投资,保持投资不变。如果实际投资水平不变,总产量Y就不再变化,实现了均衡,即:
\[
C+I+G+NX\equiv Y
\]

非意愿存货调节机制的总结:

在宏观经济学中,非意愿存货也构成一种调节机制,$ \Delta inv=AE-AD=Y-AD=I-I_{ji} $,是实际值与计划值得差额,是平衡项。这说明价格并非唯一的调节机制,产量也构成一种调节机制。

\section{两部门产品市场均衡与国民收入的决定。}
市场上两个参与者:消费者、厂商。将厂商的计划投资$ I_{ji} $视为外生变量,即$ I_{ji}=I_0=const $

此时:

总需求AD=C+$ I_{ji} $=C+$ I_0 $

总供给AS=Y=$ Y_d $=C+S

所以两部门的均衡条件为:
\[
C+I_0=Y=Y_d=C+S
\]

\hspace*{\fill}

(1)均衡条件之一:$ C+I_0=Y $→待说明的变量C

(2)均衡条件之二:$ C+I_0=C+S $→待说明变量S

\subsection{消费函数和储蓄函数}
\subsubsection{消费函数}
消费函数主要取决于商品自身的价格水平,所以消费数量主要是商品自身价格水平的函数。而本章的通用前提假设是价格水平不变,价格水平不构成影响消费水平的因素,所以消费主要是收入水平的函数,并且是$ Y_d $(税后收入),而不是Y(税前收入)的函数。消费函数写成:
\[
C=C_0+cY_d
\]
在两部门经济中,因为没有政府,$ Y=Y_d $。但在三部门四部门经济中$ Y\ne Y_d $。

消费分为两部分:一是自发消费$ C_0 $,二是引致消费$ cY_d $。

\hspace*{\fill}

(1)自发消费$ C_0 $是当个人可支配收入为0的时候存在的消费,即当$ Y_d=0 $时,$ C=C_0 $。

自发消费$ C_0 $是一个外生变量,可分为两种情况:

第一,从短期来看,没有收入,但是要维持生存,也要消费。因此就短期看$ C_0>0 $。

第二,从长期来看,没有收入就没有消费,所以$ C_0=0 $。

\hspace*{\fill}

(2)引致消费时由于个人可支配收入增加所导致的消费的增加量。这里是两项之积,即$ c\cdot Y_d $。

其中,c是边际消费倾向MPC(marginal propensity to consume)。
\[
c=\Delta C/\Delta Y_d
\]
消费函数可能有三种变化:

第一种可能是c单调下降,边际消费倾向递减。

第二种是c单调上升,边际消费倾向递增。

第三种是c固定,边际消费倾向不变。

通常假定$ 0<c<1 $。

\subsubsection{储蓄函数}
储蓄函数是从消费函数派生出来的。由$ Y_d=C+S $,可得:
\[
S=Y_d-C=Y_d-(C_0+cY_d)=-C_0+(1-c)Y_d
\]

储蓄也是由两个部分构成的:一是自发储蓄$ -C_0 $,二是引致储蓄$ (1-c)Y_d $。

(1)自发储蓄$ -C_0 $是个人可支配收入为0时的储蓄。

短期$ -C_0<0 $;长期$ -C_0=0 $。

(2)引致储蓄$ (1-c)Y_d $是由于收入增加导致的储蓄的增加量。

\hspace*{\fill}

(1-c)是边际储蓄倾向MPS(marginal propensity to save)
\[
(1-c)=\Delta S/\Delta Y_d
\]
其范围为:$ 0<(1-c)<1 $

\subsection{决定均衡国民收入的方法之一:总需求——总供给法(AD-AS法)}
第一种是总需求——总供给法,又称$ AD-AS $法,是由消费函数决定国民收入。

1. 模型

总需求:$ AD=C+I_0=C_0+cY_d+I_0 $

总供给:AS=Y

\hspace*{\fill}

由于两部门经济不包括政府,因此$ Y=Y_d $。

故而均衡产量:$ Y^*=(C_0+I_0)/(1-c) $

\hspace*{\fill}

2. 图形

以纵轴代表总需求(AD),横轴代表总供给($ Y=Y_d $),画出一个坐标系,AD与直线Y=X的焦点的横坐标值即为均衡国民收入Y*。

Y*是稳定性均衡(其调节机制为$ \Delta inv $,属于产量调节机制),稳定性均衡是指一旦偏离均衡位置,有一种自发的机制使其自动回复。稳定性均衡代表了一种规律性的、可以重复出现的现象。非稳定性均衡表明在一个特殊的位置上也可以偶然呆住,但是稍微有一个微小的偏离就很难再恢复。

\hspace*{\fill}

3. 均衡点移动的比较静态分析

Y*是一个国家均衡的总产量水平,该均衡只表示产品市场的均衡、产品市场供求相等的状态,不代表劳动力市场的均衡状态,与能够实现劳动力市场充分就业的国民收入$ Y_f $未必重合在一起。

膨胀缺口:如果均衡的Y*要比能够实现充分就业的国民收入$ Y_f $大。均衡的Y*决定一个均衡水平,能实现充分就业的国民收入又决定一个能够实现充分就业的总需求水平,这两条总需求曲线在纵轴有一段截距的差额,这个差额就是一个膨胀缺口。

紧缩缺口:均衡的Y*小于充分就业的产量水平$ Y_f $,它们对应的总需求水平水平在纵轴上有一个差距,称为之紧缩缺口。

如何消除膨胀/紧缩缺口?改变投资和消费。

$ I_{ji},C_0,c $的增加都会使总需求曲线水平上移,反之反是。

\subsection{节俭的悖论}
在微观经济学里,储蓄上升的结果是增加个人财富,所以在微观经济学里得到的结论是:节俭是美德。在宏观经济学里,在宏观经济学的两部门产品市场均衡国民收入的决定中,得到了节俭的悖论——增加消费、减少储蓄的结果是均衡国民收入的增加,所以是消费致富论。

\hspace*{\fill}

1. 争论体现了合成谬误和分解谬误的存在。

合成谬误:对局部是正确的,对整体未必正确。

分解谬误:对整体是正确的,对局部未必正确。

\hspace*{\fill}

2. 根源与宏微观经济学的假设不同。

消费致富论的适用条件:深度萧条下,生产能力闲置,价格刚性,总需求决定总供给,总供给不会成为约束总需求的条件。

而到了经济繁荣,一个国家的生产能力、资源充分利用,总攻击开始约束总需求。在这种情况下,扩张需求就会出现过的货币去追逐有限的商品,当商品的数量已经达到极限,再扩张需求就会出现通货膨胀。

宏观经济学里总需求决定国民收入,或者说总支出决定总产量。自变量为消费,因变量是收入。消费的变化是起因,国民收入的变化是结果。

微观经济学里,一个消费者的收入等于它购买的各种产品的数量乘以各个产品价格的加总。收入的变化决定每种产品消费数量的变化。收入是自变量,是起因,消费数量的变化是因变量,是结果。因果关系和宏观经济学里是倒置的。

\subsection{决定均衡国民收入的方法至二:投资——储蓄法(I-S法)}
决定均衡国民收入的方法至二,是投资——储蓄法,又叫储蓄函数决定均衡的国民收入。

1. 模型

\begin{equation}
	Y=AD
\end{equation}
\begin{equation}
	AD=C+I_{ji}
\end{equation}
\begin{equation}
	Y=Y_d=C+S
\end{equation}
将(2)(3)代入(1)得到
\[
C+I_{ji}=C+S
\]
\[
I_{ji}=S
\]
\[
I_{0}=-C_0+(1-c)Y_d
\]
最终得到
\[
Y^*=(C_0+I_0)/(1-c)
\]

\hspace*{\fill}

2. 图形

纵轴代表储蓄,也代表投资,横轴代表国民收入Y。投资曲线是一条平行于横轴的直线。储蓄曲线是单调上升,斜率小于45°的直线。由于是短期储蓄函数,因此自发储蓄是小于0的,所以储蓄函数在纵轴的负半周上有一个$ -C_0 $的截距。

Y*是稳定性均衡:

Y*以左存在:计划投资$ I_{ji} $大于储蓄导致$ \Delta inv<0 $。存在脱销,厂商会追加投资,导致投资水平I上升,投资增加等于总支出的增加,进而导致总产量水平的增加。即:

\[
C+I\uparrow\equiv Y\uparrow
\]

\hspace*{\fill}

3. 均衡点移动的比较静态均衡分析。
(1)Y*与能够实现充分就业的国民收入$ Y_f $之间的距离
\[
Y^*=(C_0+I_0)/(1-c)
\]
从Y*的表达式看,决定Y*的有三个因素:$ C_0 $、$ I_0 $和c。

$ I_{ji} $、$ C_0 $、c的增加都能增加国民收入。

\subsection{总需求——总供给法与投资——储蓄法的异同}
异:

最关键的是纵轴代表的力量是不一样的。AD-AS法中,横、纵两轴分别代表对总需求和总产量。I-S法中,纵轴同时代表总需求方的I和总供给方的S。

同:

(1)在通用前提假设之下

第一,存在着过剩的生产能力(供过于求);

第二,价格向下刚性;

第三,在供过于求有不能降价的情况下,供给是无限的,不会约束需求;

第四,均衡的国民收入由总需求单方面决定。

(2)同在产品市场

产品市场的分析能够用收入-支出模型,如果分析产品市场和货币市场的同时均衡,这个模型就不再适用了。

(3)都由消费函数决定

虽然I-S法是由储蓄函数来决定国民收入大小的,但是储蓄函数也是由消费函数派生出来的,起本质作用的还是消费函数。

(4)$ I_{ji} $为既定外生变量

(5)都是存货调节机制

\section{三部门产品市场均衡国民收入的决定}
三部门条件下,有消费者的消费支出C、厂商的投资支出$ I_{ji} $(被视为外生变量$ I_{ji}=I_0 $)、有政府的收入TA、支出G、转移支付TR。

两部门情况下有$ Y=Y_d $,三部门该条件下不成立。
\subsection{政府的收入与支出行为}
\subsubsection{政府的支出行为}
(1)政府的购买支出G

G为政府兴办公共工程的开支以及政府机构的建立、维持和运营的费用,直接购买有形的物品和无形的劳务,应直接计入国民收入Y,对Y有直接影响。处理时视作外生变量$ G=G_0 $

(2)转移支付TR

政府支出还包括转移支付以及公债利息,这里忽略公债利息不计。转移支付即没有购买有形的物品,也没有购买无形的劳务。转移支付对Y有间接影响(因其构成个人可支配收入),不直接计入Y。处理时视为外生变量$ TR=TR_0 $。

\subsubsection{政府的收入行为}
政府的税收形式有两种:固定税制和变动税制。

(1)固定税制

固定税制:税收水平时一个一次性支付的固定数量,与收入水平无关。

T是外生变量$ T=T_0 $。

(2)变动税制

变动税制:税收水平随收入的变化而变化,是收入水平的函数,而且是收入水平的一个比例$ T=tY $

其中,T是Y(税前收入)的函数,而不是$ Y_d $(税后收入)的函数。

t为税率(比例所得税),$ t=\Delta T/\Delta Y $,取值范围:$ 0<t<1 $。

进一步可将税收的构成分为两部分:
\[
T=T_0+tY
\]
一部分是自发的$ T_0 $,与收入无关,取决于政府意愿,是一个既定外生变量;另一部分是和收入相关的tY,即收入(比例)所得税。

按支出法,税收对Y没有直接影响,不直接计入Y,但是有间接影响(通过个人可支配收入)。

\subsection{固定税制条件下,三部门产品市场均衡国民收入的决定}
\[
Y=AD
\]
\[
AD=C+I_{ji}+G
\]
\[
C=C_0+cY_d,0<c<1
\]
\[
Y_d=Y+TR_0-T
\]
\[
T=T_0
\]
\[
I_{ji}=I_0
\]
\[
G=G_0
\]
综上:
\[
Y=AD=C_0+I_0+G_0+cTR_0-cT_0+cY
\]
令截距项$ A_0=C_0+I_0+G_0+cTR_0-cT_0 $

则:
\[
Y^*=(C_0+I_0+G_0+cTR_0-cT_0)/(1-c)=A_0/(1-c)
\]

三部门固定税制条件下,均衡的国民收入Y*取决于六个外生变量:$ C_0 $、$ I_0 $、$ G_0 $、$ TR_0 $、$ T_0 $、$ c $。

\hspace*{\fill}

2. 45°线法——萨缪尔森交叉图

\subsection{固定税制条件下的乘数}
1. 乘数(multiplier)的定义
乘数的本质是一个边际量,用k表示。乘数=边际量=$ \Delta $因变量/$ \Delta $自变量。

乘数有六个:$ k_{C_0} $、$ k_i $、$ k_g $、$ k_{tr} $、$ k_{T_0} $、$ k_c $

\[ k_{C_0}=\partial Y^*/\partial C_0=1/(1-c) \]
\[ k_{I_0}=\partial Y^*/\partial I_0=1/(1-c) \]
\[ k_g=\partial Y^*/\partial G=1/(1-c) \]
\[ k_{TR}=\partial Y^*/\partial TR=c/(1-c) \]
\[ k_{T_0}=\partial Y^*/\partial T_0=-c/(1-c)<0 \]
\[ k_c=\partial Y^*/\partial c=Y_d/(1-c) \]

平衡预算乘数:

预算盈余(budget surplus,BS)的概念:

BS=政府的收入-政府的支出=T-(G+TR)

A. BS=0

这是一个完美的状态,即政府的收入完全等于政府的支出,政府完全实现了收支相等。

B. $ \Delta BS=0 $

政府手指的变动量等于0。现在政府的预算存在赤字或盈余,但只要保持政府的赤字不再增加或者盈余不再减少即可。

\[
\Delta BS=\Delta T-(\Delta G+\Delta TR)=0
\]
假定:$ \Delta TR=0 $

则若要保持平衡预算不变:$ \Delta T_0=\Delta G $

政府增加收入的同时增加政府购买支出就可以实现平衡预算。

若政府既要干预经济,又要保持自身的预算赤字不增加,在这种情况下均衡的国民收入的变动会怎样?

$ \Delta G $是正向的力量,对Y有扩张作用,其影响为:
\[
\Delta Y_G=\Delta G+c\Delta G+c^2\Delta G+c^3\Delta G+\cdots
\]
$ \Delta T_0 $是反方向的力量,对Y有收缩的作用,其影响为:
\[
\Delta Y_{T_0}=-c\Delta T_0-c^2\Delta T_0-c^3\Delta T_0-\cdots
\]

合力对Y的影响为:
\[
\Delta Y=\Delta Y_G+\Delta Y_{T_0}=\Delta G=\Delta T_0
\]
因此:
\[
k_{ping}=\Delta Y/\Delta G=\Delta Y/\Delta T_0=1
\]
平衡预算乘数意味着政府有两个目标:既要兼顾自身的预算平衡,又要考虑对Y的影响。


\subsection{变动税制条件下,三部门产品市场均衡国民收入的决定}
1. 模型
\[
Y=AD
\]
\[
AD=C+I_{ji}+G
\]
\[
C=C_0+cY_d,0<c<1
\]
\[
Y_d=Y+TR_0-T
\]
\[
T=T_0+tY,0<t<1
\]
\[
I_{ji}=I_0
\]
\[
G=G_0
\]
得到:
\[
Y=AD=C_0+I_0+G_0+cTR_0-cT_0+c(1-t)Y
\]
解得均衡国民收入:
\[
Y^*=(C_0+I_0+G_0+cTR_0-cT_0)/(1-c)=A_0/[1-c(1-t)]
\]
其中:$ A_0=C_0+I_0+G_0+cTR_0-cT_0 $

说明$ C_0 $、$ I_0 $、$ G_0 $、$ TR_0 $、$ T_0 $、$ c $、$ t $这七个外生变量的变化都将导致一国GDP的变化。

\hspace*{\fill}

2. 45°线法

固定税制和变动税制下的总需求函数有相同的截距,但变动税制下的总需求函数曲线的斜率小,因此和45°线交在更左的地方。

\subsection{变动税制条件下的乘数}
(1)政府购买支出乘数:
\[ k_g=\partial Y^*/\partial G=1/[1-c(1-t)] \]

(2)自发消费乘数:
\[ k_{C_0}=\partial Y^*/\partial C_0=1/[1-c(1-t)] \]

(3)投资乘数:
\[ k_{I_0}=\partial Y^*/\partial I_0=1/[1-c(1-t)] \]

(4)转移支付乘数:
\[ k_{tr}=\partial Y^*/\partial TR=c/[1-c(1-t)] \]

(5)政府的固定税收乘数:
\[ k_{T_0}=\partial Y^*/\partial T_0=-c/[1-c(1-t)]<0 \]

(6)边际消费倾向乘数:
\[ k_c=\partial Y^*/\partial c=Y_d/[1-c(1-t)] \]

(7)税率乘数:
\[ k_t=\partial Y^*/\partial t=-cY/[1-c(1-t)] \]

(8)平衡预算乘数
\[
\Delta BS=\Delta T-(\Delta G+\Delta TR)=0
\]
假定$ \Delta TR=0 $

故$ \Delta T=\Delta G $

由于:$ T=T_0+tY $

对上式两边全微分、最终税收的变动量为:
\[
\Delta T=\Delta T_0+\Delta tY+t\Delta Y+\Delta t\Delta Y
\]

忽略最后一项:

\[
\Delta T=\Delta T_0+\Delta tY+t\Delta Y
\]

A. 如果$ \Delta T_0=\Delta G $

$ \Delta G $是正向的力量,对Y有扩张作用,其影响为:
\[
\Delta Y_G=\Delta G\times k_g=\Delta G/[1-c(1-t)]
\]

$ \Delta T_0 $是反向的力量,对Y有收缩的作用,其影响为:
\[
\Delta Y_{T_0}=\Delta T_0\times k_{T_0}=-\Delta T_0c/[1-c(1-t)]
\]

合力对Y的影响为:

\[
\Delta Y=\Delta Y_G+\Delta Y_{T_0}=\Delta G(1-c)/[1-c(1-t)]=\Delta T_0(1-c)/[1-c(1-t)]
\]

\[
k_{ping}=\Delta Y/\Delta G=\Delta Y/\Delta T_0=(1-c)/[1-c(1-t)]<1
\]

$ k_{ping} $的经济含义:政府在考虑“预算平衡”的条件下,即把增加的政府收入中的部分(固定税收部分)用于政府购买支出,由此导致国民收入的增加量小于政府购买支出的增加量,也小于政府固定税收的增加量。

\hspace*{\fill}

B. 如果$ \Delta T=\Delta G $

合力对Y的影响为:

\[
\Delta Y=\Delta T=\Delta G
\]
\[
k_{ping}=\Delta Y/\Delta G=\Delta Y/\Delta T=1
\]

\subsection{不同财政政策工具对预算盈余的影响}
\subsubsection{财政政策工具}
财政政策工具有支出政策,包括政府购买支出G和转移支付TR。财政政策工具还有收入政策,包括自发税收$ T_0 $以及税率t。

\subsubsection{政府的预算盈余}
政府的预算盈余 BS=政府收入-政府支出=$ T-(G+TR)=T_0+tY-(G+TR
) $

考虑Y理论上有一个达到充分就业的$ Y_f $,所以它将对应于一个充分就业的预算盈余BS*。
\[
BS^*=T_0-G-TR+tY_f
\]
\[
\Delta BS=\Delta T-(1-\Delta G+\Delta TR)=\Delta T_0+\Delta tY-\Delta G-\Delta TR+t\Delta Y
\]
其中$ \Delta T_0,\Delta tY,\Delta G,\Delta TR $为直接影响:财政工具的变动,立刻导致预算盈余总量的变动。这四个政策工具对预算盈余的影响非常直接,不需要任何的中间环节。

$ t\Delta Y $为间接影响:体现为财政政策工具的变动,首先导致均衡国民收入的变动,四个财政政策工具对均衡国民收入都有一个乘数效应,进而均衡国民收入的变动又导致预算盈余总量的变动。这个影响是间接的,有一个产生乘数效应的中间环节。

\subsubsection{$ \Delta G $对$ \Delta BS $的影响}
\begin{equation*}
	\begin{split}
	\Delta BS&=\Delta T-(\Delta G+\Delta TR)\\
	&=\Delta T-\Delta G\\
	&=\Delta T_0+\Delta tY+t\Delta Y-\Delta G\\
	&=t\Delta Y-\Delta G\\
	&=t\times\Delta G\times k_g-\Delta G\\
	&=-\Delta G(1-c)(1-t)/[1-c(1-t)]
	\end{split}
\end{equation*}
\[
\Delta BS/\Delta G=-(1-c)(1-t)/[1-c(1-t)]<0
\]
\subsubsection{$ \Delta TR $对$ \Delta BS $的影响}
\begin{equation*}
\begin{split}
\Delta BS&=\Delta T-(\Delta G+\Delta TR)\\
&=\Delta T-\Delta TR\\
&=\Delta T_0+\Delta tY+t\Delta Y-\Delta TR\\
&=t\Delta Y-\Delta TR\\
&=t\times\Delta TR\times k_{tr}-\Delta TR\\
&=-\Delta TR(1-c)/[1-c(1-t)]
\end{split}
\end{equation*}
\[
\Delta BS/\Delta TR=-(1-c)/[1-c(1-t)]<0
\]

\subsubsection{$ \Delta T_0 $对$ \Delta BS $的影响}
\begin{equation*}
\begin{split}
\Delta BS&=\Delta T-(\Delta G+\Delta TR)\\
&=\Delta T\\
&=\Delta T_0+\Delta tY+t\Delta Y\\
&=\Delta T_0+t\Delta Y\\
&=t\times\Delta T_0\times k_{T_0}+\Delta T_0\\
&=\Delta T_0(1-c)/[1-c(1-t)]
\end{split}
\end{equation*}
\[
\Delta BS/\Delta T_0=(1-c)/[1-c(1-t)]>0
\]

\subsubsection{$ \Delta t $对$ \Delta BS $的影响}
\begin{equation*}
\begin{split}
\Delta BS&=\Delta T-(\Delta G+\Delta TR)\\
&=\Delta T\\
&=\Delta T_0+\Delta tY+t\Delta Y\\
&=\Delta tY+t\Delta Y\\
&=\Delta tY+t\times\Delta t\times k_t\\
&=\Delta t(Y+t\times k_t)=\Delta t(1-c)Y/[1-c(1-t)]
\end{split}
\end{equation*}
\[
\Delta BS/\Delta t=(1-c)Y/[1-c(1-t)]>0
\]

\section{四部门产品市场均衡国民收入的决定}
\subsection{四部门}
四部门即四个经济主体(四个市场参与者):第一个是消费者,有消费者就有消费支出C;第二个是厂商,有厂商的计划投资$ I_{ji}=I_0 $;第三是政府,政府的收入自发税收$ T_0 $和税率t与政府的购买支出G和转移支付TR都对经济产生影响;第四是国外消费者,国外消费者花的钱就是净出口NX,净出口等于出口X减去进口M。出口X取决于外国消费者对本国产品的需求,取决于外国消费者的收入、偏好,本国产品价格,汇率等因素,故可将出口视为一个外生变量,$ X=X_0 $。而进口是本国消费者对国外产品的需求,取决于本国消费者的行为。
\subsection{进口函数}
进口函数:$ M=M_0+mY $

第一项是自发进口$ M_0 $和收入无关。

第二项是引致进口mY。引致进口是国民收入的函数,不是个人可支配收入的函数(政府也要进口国外产品)。

m是边际进口倾向,是一个边际量。$ m=\Delta M/\Delta Y,0<m<1 $

\subsection{四部门产品市场均衡国民收入的决定}
1. 模型
\begin{equation*}
	\begin{split}
	&Y=AD\\
	&AD=C+I_0+G_0+NX\\
	&C=C_0+cY_d,0<c<1\\
	&Y_d=Y+TR_0-T\\
	&T=T_0+tY,0<t<1\\
	&NX=X_0-M=X_0-(M_0+mY)
	\end{split}
\end{equation*}
\[
Y=AD=C_0+I_0+G_0+cTR_0-cT_0+X_0-M_0+[c(1-t)-m]Y
\]
\begin{equation*}
	\begin{split}
	Y^*&=(C_0+I_0+G_0+cTR_0-cT_0+X_0-M_0)/[1-c(1-t)+m]\\
	&=A_0/[1-c(1-t)+m]
	\end{split}
\end{equation*}

\hspace*{\fill}

\subsection{开放经济条件下的各种乘数}
1. $ k_{C_0}=k_i=k_g=k_{X_0}=1/[1-c(1-t)+m] $

2. 自发进口乘数$ k_{M_0}=-1/[1-c(1-t)+m] $

3. 转移支付乘数$ k_{tr}=c/[1-c(1-t)+m] $

4. 自发税收乘数$ k_{T_0}=-c/[1-c(1-t)+m] $

5. 边际消费倾向乘数$ k_c=Y_d/[1-c(1-t)+m] $

6. 税率乘数$ k_t=-cY/[1-c(1-t)+m] $

7. 边际进口倾向乘数$ k_m=-Y/[1-c(1-t)+m] $

8. 平衡预算乘数

(1)若$ \Delta T_0=\Delta G $
\[
k_{ping}=\Delta Y/\Delta G=\Delta Y/\Delta T_0=(1-c)/[1-c(1-t)+m]<1
\]
(2)若$ \Delta T=\Delta G $
\[
k_{ping}=\Delta Y/\Delta G=\Delta Y/\Delta T=(1-c)/[1-c+m]<1
\]


\end{document}