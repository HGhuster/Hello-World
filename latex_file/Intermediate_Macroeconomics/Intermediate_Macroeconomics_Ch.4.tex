\documentclass{article}
\title{产品市场和货币市场的同时均衡}
\author{Dawei Wang}
\date{\today}
\usepackage{ctex}
\usepackage{amsmath}
\usepackage{amssymb}
\begin{document}
	\maketitle
\section{产品市场均衡:IS曲线}
\subsection{投资的决定}
在投资的决定里,总投资等于自发投资$ I_0 $加上引致投资-bR。
\[
I=I_0-bR
\]

1. 自发投资$ I_0 $

自发投资是为了获得最大利益而进行的投资,其大小取决于资本的边际效率(资本收益率)。资本边际效率取决于宏观经济形势的变化,可将其视为一个既定的外生变量。

\hspace*{\fill}

2. 引致投资-bR

投资的成本分两种:若是贷款来的资金,则利率R是一种显性成本;若是用自有资金进行投资,则利率R是一种隐性成本。投资的显性成本和隐性成本都取决于利率R。

b是用来衡量投资对利率变动的敏感程度的指标,叫作投资的利率弹性。与微观经济学中不同,b虽然叫做利率弹性,但其本质还是一个边际量:$ -b=\Delta I/\Delta R $。

\hspace*{\fill}

投资由一个无法解释的外生变量变成了一个内生变量。投资被表述成三个变量$ I_0 $、$ b $、$ R $的函数,这三个变量的变化都将导致投资的变化。

\subsection{引入投资函数条件下,三部门产品市场均衡国民收入的决定}
1. 模型
\begin{equation*}
	\begin{split}
	&Y=AD\\
	&AD=C+I_{ji}+G_0\\
	&C=C_0+cY_d,0<c<1\\
	&Y_d=Y+TR_0-T\\
	&T=T_0+tY,0<t<1\\
	&I_{ji}=I_0-bR,-b<0
	\end{split}
\end{equation*}
\[
Y=AD=C_0+I_0+G_0+cTR_0-cT_0-bR+c(1-t)Y
\]
\begin{equation*}
	\begin{split}
	Y^*&=(C_0+I_0+G_0+cTR_0-cT_0-bR)/[1-c(1-t)]\\
	&=(A_0-bR)/[1-c(1-t)]	
	\end{split}
\end{equation*}

\hspace*{\fill}

2. 45°线法

\hspace*{\fill}

3. 均衡点的移动

(1)波动根源和传导机制

$ C_0 $、$ I_0 $、$ G_0 $、$ TR_0 $、$ T_0 $、$ b $、$ R $、$ c $、$ t $这9个外生变量都将导致一国GDP的变化。特别关注R,R是联系两个市场——产品市场和货币市场的桥梁和纽带,在收入——支出模型中是一个外生变量。波动根源关注的是利率的变化。

利率和投资反向变化。利率下降,总投资增加,进而总需求和总产量增加,反之反是。

(2)坐标转换:从AD-Y坐标系到R-Y坐标系

初始的波动发生在总需求总供给坐标系,总需求-总供给坐标系只讨论产品市场均衡,而在利率-总产量坐标系能够讨论产品和货币市场的同时均衡,而在利率-总产量坐标系能够讨论产品和货币市场的同时均衡。由于两个坐标系的横轴一样,因此它们可以垂直放置。

将能够实现产品市场均衡的收入和利率的组合点连起来就得到IS曲线。

IS曲线的含义:要实现产品市场的均衡,利率和国民收入必须相互配合、一一对应的、反方向变动的点的轨迹就是IS曲线,它描述了产品试产实现均衡的道路。线上的每一点都是能够实现产品市场均衡的国民收入的组合点。

\hspace*{\fill}

4. IS曲线的表达式
把均衡收入水平的表达式$ Y^*==(A_0-bR)/[1-c(1-t)] $变形:
令$ \alpha=1/[1-c(1-t)] $
\[
Y^*=(A_0-bR)\alpha=\alpha A_0-\alpha bR
\]

\subsection{决定IS曲线位置(左右平移)的因素}
当Y=0时,$ R=A_0/b $——IS曲线在纵轴的截距;当R=0时,$ Y=\alpha A_0 $——IS曲线在横轴的截距。表达式是:$ Y=\alpha A_0-\alpha bR $。

当$ A_0 $增加,IS曲线将发生一个向右的平移。导致$ A_0 $增加的因素包括:$ C_0 $、$ I_0 $、$ G_0 $和$ TR_0 $的上升,或者$ T_0 $的下降。IS曲线右移的幅度是$ \alpha\Delta A_0 $,反之反是。

\subsection{决定IS曲线斜率的因素}
IS曲线的斜率表示为-$ \alpha $b。

1. b

b上升会导致IS曲线在纵轴的截距$ A_0/b $下降。

\hspace*{\fill}

2. t

t上升导致$ \alpha $下降横轴截距$ \alpha A_0 $减小。

\subsection{IS曲线的特殊情况}

导致IS曲线移动的因素$ A_0 $是五项的代数和,涉及政策工具;b是投资需求的利率弹性;决定$ \alpha $的是c和t。

投资需求的利率弹性b取决于厂商投资需求对利率变动的敏感程度,涉及心理因素。

在此讨论b变化的两种特殊情形:

\hspace*{\fill}

1. 第一种情形:b→0

就b的表达式来看:-b=$ \Delta I/\Delta R=0 $,所以推出$ \Delta I=0 $。

此时IS曲线垂直。

\hspace*{\fill}

2. 第二种情形:b→∞

b的表达式是:$ -b=\Delta I/\Delta R=\infty $,进一步推出,要么分母的变动量$ \Delta R $等于0,要么分子的变动量$ \Delta I $趋向于正无穷。

此时IS曲线水平。

\subsection{IS曲线以外点的经济含义}
1. IS曲线以左点的经济含义

在AD-Y坐标系中,IS曲线以左点对应的点存在AD>Y,即过度的产品需求EDG(excess demand of goods),$ \Delta inv<0 $,存在脱销,厂商追加投资,扩大AD,进而扩大Y。在R-Y坐标系中,IS曲线以左的点同样存在过度产品需求EDG,存在一个向右拉动的力量,这是厂商扩张生产的力量。

\hspace*{\fill}

2. IS曲线以右点的经济含义

在AD-Y坐标系中,IS曲线以右的点对应的点存在AD<Y,即过度的产品供给ESG(excess supply of goods),$ \Delta inv>0 $,存在积压,厂商要减少投资,减少AD,进而减少Y。

\subsection{推导IS曲线的另一种方法——四象限法}
IS曲线的I代表投资,S代表储蓄,IS曲线的含义就是储蓄转化为投资,或者投资等于储蓄,I=S。这是收入——支出模型中I-S法的均衡条件,意味着从I=S可以推出利率R与国民收入Y之间的关系,因而叫IS曲线。

1. 用I-S法推导三部门变动税制条件下的储蓄函数。
\begin{equation*}
	\begin{split}
	&Y=AD\\
	&AD=C+I+G_0\\
	&Y=Y_d+T-TR_0=C+S+T-TR_0
	\end{split}
\end{equation*}
变形:
\[
I=S+T-TR_0-G_0
\]
将右边视为$ S_{zong} $
\[
S_{zong}=S_{geren}+S_{zhengfu}(BS)
\]
\begin{equation*}
	\begin{split}
	S_{geren}&=Y_d-C\\
	&=Y_d-(C_0+cY_d)\\
	&=-C_0+(1-c)Y_d\\
	&=-C_0+(1-c)[Y+TR_0-(T_0+tY)]\\
	&=-C_0+(1-c)(TR_0-T_0)+(1-c)(1-t)Y
	\end{split}
\end{equation*}
\begin{equation*}
	\begin{split}
	S_{zong}&=S_{geren}+S_{zhengfu}\\
	&=-C_0+(1-c)(TR_0-T_0)+(1-c)(1-t)Y\\
	&+[(T_0+tY)-TR_0-G_0]\\
	\end{split}
\end{equation*}
\begin{equation*}
	\begin{split}
	S_{zong}=-C_0-cTR_0+cT_0-G_0+[1-c(1-t)Y]
	\end{split}
\end{equation*}
均衡条件:
\[
I=S_{zong}
\]
故:
\[
I_0-bR=-C_0-cTR_0+cT_0-G_0+[1-c(1-t)Y]
\]
\[
Y^*=(A_0-bR)/[1-c(1-t)]
\]

\subsection{IS曲线中蕴含的因果关系}
从IS曲线的表达式:$ Y=\alpha A_0-\alpha bR $,Y、R互为因果关系,无所谓谁为因谁为果。

\section{货币市场均衡:LM曲线}
利率是货币资产的价格,关于利率的决定就要涉及货币市场。LM的曲线则体现了货币市场的均衡。货币市场均衡要从货币供给和货币需求来研究。
\subsection{资产的种类和选择}
按照流动性从大到小来进行划分:

1. 货币

货币的职能:交换媒介、价值尺度、贮藏手段。

宏观经济学里的货币主要作为一种交易媒介出现。$ M=C+D$,C指通货(currency),D指在商业银行里的活期存款(deposit)。
这两类金融资产的特点是具有完全的流动性,可以随时用来支付,就收益来讲,它们的收益很低。

\hspace*{\fill}

2. 金融资产

在宏观经济学中,把除了货币以外的其他流动性金融资产统称为债券。债券包括定期存款、股票、国库券以及现实生活中各种各样的债券。特点是流动性差收益高。

\hspace*{\fill}

3. 实物资产

在宏观经济学的研究中,实物资产和利率没有关系,可以忽略。

\hspace*{\fill}

4. 重要假定

假定人们只在货币、债券这两种流动性强的资产形式中进行选择。债券是货币的唯一替代物。

\subsection{货币需求}
凯恩斯认为对货币的需求分为三种动机:

\hspace*{\fill}

1. 交易动机(transaction motive)

由于收入与支出有时滞,因此人们需要保留一部分货币在手中,以应付日常交易的需要。

\hspace{\fill}

2. 预防动机(precautionary motive)

未来的收入和支出具有不确定性,所以人们需要保留一部分货币在手中,以应付不能预料的收入延期和支出增加。

\hspace*{\fill}

前两种动机可归为一类,统称为交易动机的货币需求,用$ L_t $来表示,L(liquidity)体现了人们对货币资产流动性强优点的偏好。这两类动机的货币需求的共性是:与收入同方向变动,但与利率无明显相关关系。它们的表达式可以写成:
\[
L_t=kY
\]
k叫作货币需求的收入弹性,虽然名为弹性,本质上是一个边际量。
\[
k=\lim_{\Delta Y\rightarrow 0}\Delta L_t/\Delta Y=dL_t/dY,0<k<1
\]

3. 投机动机(speculation motive)

投机动机的货币需求涉及人们转换资产组合的动机:改变持有两种资产的组合比例,来获得最大收益。

(1)债券的价格和利率

\hspace*{\fill}

(2)投机动机的货币需求与R

投机动机的货币需求是指人们需要保留在手中,以便在有利可图的时候进行投资或投机,用$ L_s $来表示。

若现行利率低,则:1.持有货币的机会成本低;2:债券价格过高,未来有较大下跌风险。因此人们选择抛售债券持有货币,此时投机动机的货币需求$ L_s $就高。因此投机动机的货币需求和利率反向变动。

用h来衡量投机动机货币需求对利率变动的敏感程度,h叫作货币需求的利率弹性。$ -h=\Delta L_s/\Delta R<0 $是边际量。

$ L_s=W_0-hR $,$ W_0 $是经济中流动性资产的实际值,是一个金融资产的总约束。

\hspace*{\fill}

4. 总的货币需求函数
\[
L=L_t+L_s=kY+W_0-hR
\]

横轴截距:kY+$ W_0 $

纵轴截距:$ R=(kY+W_0)/h $

\hspace*{\fill}

(1)决定L曲线位置(左右平移)的因素:k,Y;

(2)决定L曲线斜率的因素:h。

\subsection{货币供给}
货币供给写成:$ M/P=M_0/P $,M是名义货币供给量,$ M=M_0 $说明名义货币供应量是一个中央银行决定的外生变量。$ M/P=M_0/P $的含义是:实际货币供给量是一个由中央银行决定的外生变量。

货币供给图形中,纵轴代表利率R,横轴代表货币供给M/P。由于货币供给量是一个由央行决定的外生变量$ M_0/P $,因此它是一条不随利率水平变化的、垂直于横轴的直线。

\subsection{货币市场的均衡}
1. 均衡条件:
(1) 模型
\begin{equation*}
	\begin{split}
	&L=M/P\\
	&L=kY-hR\\
	&M/P=M_0/P
	\end{split}
\end{equation*}
故:
\[
R^*=-M_0/(hP)+(k/h)Y
\]

(2) 图形

\hspace*{\fill}

2. 均衡点的移动

(1) 波动的根源和传导机制

k、h、$ M_0 $、P、Y——五个外生变量中,考察收入Y的变化。Y在收入——支出模型中是内生变量、对货币市场而言收入Y是波动的根源,因为Y体现了产品市场对货币市场的反作用。

波动传导机制:初始收入上升导致交易动机的货币需求增加,进而导致总货币需求上升,在货币供给不变的条件下,货币的资产价格利率上升。

(2) 坐标转换:从R-L,M/P坐标系到R-Y坐标系。

\subsection{决定LM曲线位置(左右平移)的因素}
纵轴截距:$ -M_0/(hP) $

横轴截距:$ M_0/kP $

如果$ M_0/P $上升,则右移,反之反是。

\subsection{决定LM斜率的因素}
k、h

k上升,纵轴截距不变,斜率上升。

h上升,横轴截距不变,斜率下降。

\subsection{LM曲线以外点的经济含义}
LM曲线上的点都是能够实现货币市场均衡的国民收入与利率的组合点。

1. LM曲线以左点的经济含义

LM曲线以左的点存在着过度的货币供给ESM(excess supply of money),所以利率存在着下行压力。

2. LM曲线以右的经济含义

LM曲线以右的点存在着过度的货币需求EDM(excess demand of money),故利率存在着上行压力。

\subsection{推导LM曲线的另一种方法——四象限法}

\subsection{LM曲线的特殊情况}
决定LM曲线左右移动的因素是$ M_0/P $,决定其斜率的是k和h。$ M_0 $是由央行决定的外生变量,k取决于个人经济生活中一些制度因素,h是货币需求的利率弹性,涉及人们心理因素的变化,下面的两个特例即分析h的两种变化。

\hspace*{\fill}

1. 古典特例:h→0

其经济含义为投机动机的货币需求对利率的变动是不敏感的。这意味着货币的投机需求是一个常数。

此时LM曲线垂直与横轴,其位置取决于k、$ M_0/P $。此时LM曲线的表达式是:$ Y=M_0/(kp) $。

2. 凯恩斯特例:h→∞

其经济含义为投机动机的货币需求对利率的变动极度敏感。

现在的利率从$ R_1 $极小幅度地下降至$ R_2 $,由于投机动机的货币动机对于利率的变动是非常敏感的,在一个很低的水平上,变动量趋向于正无穷。

在一个较低的利率水平上,投机动机的货币需求趋向于正无穷。此时央行增加货币供应量,由于投机动机的货币需求趋向于正无穷,因此央行在市场投放的货币将全部被人们所持有观望。此时货币政策失效,很难通过货币增发降低市场利率。(凯恩斯陷阱/流动性陷阱),货币资产价格利率R有向下刚性。

\hspace*{\fill}

LM曲线可分为三个区域:凯恩斯区域(水平)、古典区域(垂直)、中间区域(倾斜)。

在古典区域,货币政策有效,财政政策失效(增加政府支出G,IS曲线右移,然而均衡的国民收入不变,因为LM曲线垂直)。

\subsection{LM曲线中蕴含的因果关系}
单纯从LM曲线表达式看:$ R=-M_0/(hp)+(k/h)Y $,利率和收入具有对应关系,互为因果。从图形看LM曲线单调上升,图形只要有单调性就存在反函数,起因结果两者之间互为因果、一一对应。


\section{产品和货币市场同时均衡:IS-LM模型}
\subsection{对IS-LM模型的分歧}
IS-LM模型是分析产品市场和货币市场同时均衡的工具,也是凯恩斯主义宏观经济学分析的经典范式。

但由于在凯恩斯的《通论》里并没有IS-LM模型,而是希克斯根据凯恩斯的思想总结出来的(也得到了凯恩斯本人的认可)。因此对于IS-LM模型能不能代表凯恩斯思想这个问题,经济学发展史上曾经出现过分歧。

萨缪尔森赞同IS—LM模型代表凯恩斯的思想,琼·罗宾逊反对。
\subsection{凯恩斯的基本理论框架}
凯恩斯理论体系的基础是价格刚性,这是他理论中不容置疑的公理,他排除了价格机制自发调节经济的作用。

从这个前提假设出发,也很难得出最终结果一定是非均衡的,需要国家干预的结论。凯恩斯又提出了三大心理规律,作为在公理基础上的各种各样的定理。

在宏观经济学流程图里,在一个封闭经济条件下,要增加总需求进而增加总产量,路径实际上有三条:第一条:刺激消费,第二条:刺激投资,第三条:增加政府购买支出。

\hspace*{\fill}

首先是增加消费,政府刺激消费的措施可以是:减税/增加转移支付,进而增加个人可支配收入,起到刺激消费的作用。但是消费的增加有没有上限呢?凯恩斯的第一大心理规律认为边际消费倾向是递减的。

1. 三大心理规律之一——边际消费倾向递减规律

意即边际消费倾向对个人可支配收入的一阶导小于0(消费对个人可支配收入的二阶导小于零)。

由此说明一个问题:消费是不足的,它不可能无限扩大,这里凯恩斯通过三大心理规律之一——边际消费倾向递减规律,否定了消费对增加总需求进而增加总产量实现充分就业的作用。

\hspace*{\fill}

2. 三大心理规律之二——资本的边际效率递减规律

资本边际效率递减的原因是:在经济萧条的背景下,资本的边际效率就是投资收益率,由于人们预期的投资收益率低,预期的资本边际效率低,所以有钱也不投资。不投资导致总需求乃至总产量下降,从而导致经济更加萧条。经济越萧条,预期未来的投资收益率将会更低。从而形成了恶性循环。

资本边际效率递减的后果是自发投资曲线虽然单调上升,但是它将导致一种递减的速率单调上升,自发投资的增加存在一个上限,不可能无限制地增加。凯恩斯三大心理规律之二——资本的边际效率递减规律,否定了自发投资无限增大的可能性。

\hspace*{\fill}

3. 三大心理规律之三——流动性陷阱

投资的组成是自发投资加上引致投资,自发投资不能无限增大,那么有没有可能通过不断降低利率,增加投资的机会成本来刺激投资呢?

当h趋于正无穷,对于一条水平的LM曲线,增发多少货币,人们就持有多少货币,货币需求和货币供给同步增加,不可能通过增发货币来降低利率。这说明货币资产价格利率也存在向下刚性,它的下降是无限制地,故而货币政策失效。

将三大心理规律二和三结合在一起,得到的结论是:投资是不足的,政府影响投资政策的作用也是有限的。

\hspace*{\fill}

三大心理规律结合在一起,我们得出:由于消费和投资的不足,都不可能无限制地增加,所以通过这两个路径影响总需求也不可能无限制地增加,通过这两个路径影响总需求地作用是有限的,总需求是不足的,导致两个市场同时均衡条件下地总产量水平Y*必然会小于充分就业下地总产量水平$ Y_f $。故而政府干预必不可少。

\subsection{IS-LM模型中的均衡收入和均衡利率}
联立IS曲线和LM曲线:
\begin{equation*}
	\begin{split}
	&Y=\alpha A_0-\alpha bR\\
	&kY-hR=M_0/P
	\end{split}
\end{equation*}
解得:
\begin{equation*}
	\begin{split}
	&Y^*=\frac{A_0+(b/h)(M_0/P)}{1-c(1-t)+bk/h}\\
	&R^*=\frac{kA_0-[1-c(1-t)](M_0/P)}{h[1-c(1-t)]+bk}
	\end{split}
\end{equation*}

在利率和国民收入地坐标空间中,IS曲线代表了产品市场均衡的道路,LM曲线代表了货币市场均衡的道路,所以这两个线的交点E上,是能够同时实现两个市场同时均衡的利率和收入的组合点$ Y^* $和$ R^* $。

初始经济未必在E点,第一象限被IS曲线和LM曲线分成四个区域,四个区域里的点各自如何移动?

四个失衡区域的含义:

IS曲线以左的点,受到水平向右的拉力,因为非意愿存货小于零,这是过度的产品需求EDG导致厂商扩张产量的力量,反之反是。

LM曲线以上的点,收到垂直向下的拉力,这是过度的货币供给ESM导致货币资产价格利率下降的力量。

只有E点,既不存在水平方向的力,也不存在垂直方向的力。

通过对这个体系的动态学分析,得到的结论是:合力一定趋向均衡,没有发散的可能。

\hspace*{\fill}

对调整速度的假设

货币市场是金融资产交易的市场,它实现供求相等的速度,即出清的速度比产品市场要快,产品市场涉及衣食住行方方面面、成千上万种有形的产品以及无形的劳务,所以出清的速度比货币市场慢。这就意味着垂直方向的作用力大于水平方向的作用力。

如果垂直方向的力大于水平方向的力,趋向均衡的路径将会发生什么样的事情呢?

均衡外的点会在到LM曲线后沿着LM曲线向E点逼近。

\subsection{财政政策乘数$ k_g $和货币政策乘数$ k_m $}
特别关注政策变量的变化对GDP的影响。下面分析两个市场同时均衡的条件下的财政政策和货币政策乘数。

 (1) 财政政策乘数$ k_g $
\begin{equation*}
	\begin{split}
	k_g&=\partial Y/\partial G\\
	&=\frac{1}{1-c(1-t)+bk/h}
	\end{split}
\end{equation*}

(2) 货币政策乘数$ K_m $
\begin{equation*}
	\begin{split}
	k_m&=\partial Y/\partial (M_0/P)\\
	&=\frac{b/h}{1-c(1-t)+bk/h}\\
	&=\frac{1}{(h/b)[1-c(1-t)]+k}
	\end{split}
\end{equation*}

财政、货币政策乘数是考虑财政政策和货币政策的运用对均衡国民收入带来的影响,同时也应考虑财政政策和货币政策的运用对均衡利率带来的影响。

(3) 
\begin{equation*}
	\begin{split}
	\partial R/\partial G&=\frac{k}{h[1-c(1-t)]+bk}\\
	&=\frac{1}{(h/k)[1-c(1-t)]+b}
	\end{split}
\end{equation*}
此式说明政府购买支出增加,均衡利率水平也将会增加,两者同方向变化。如果政府购买支出增加将会导致利率上升,利率上升导致投资下降,这种效应称为财政政策的挤出效应,此表达式恰好证明了财政政策“挤出效应”的存在。

(4)
\begin{equation*}
	\begin{split}
	\partial R/\partial (M_0/P)&=\frac{-[1-c(1-t)]}{h[1-c(1-t)]+bk}\\
	&=\frac{-1}{h+bk/[1-c(1-t)]}
	\end{split}
\end{equation*}
货币供给量的增加会导致均衡利率下降,体现了货币政策的运用对利率的影响。

\subsection{影响财政政策和货币政策的因素}
影响财政、货币政策的因素有五个:c、t、b、k、h。

就边际消费倾向c来讲,根据各国的统计数据的统计数据实证分析,发现各个国家的边际消费倾向基本稳定,因此实证分析推翻了凯恩斯三大心理规律之一——边际消费倾向递减。

税率和货币需求的收入弹性k涉及经济生活中的制度因素,我们认为基本稳定。

投资需求的利率弹性b和货币需求的利率弹性涉及心理因素,讨论IS曲线特例的时候有b趋向于0和b趋向于正无穷两种情形,得到IS曲线的两个特例,垂直的和水平的。

关于LM曲线特例的讨论也有两种情况:h趋向于0(古典),h趋向于正无穷(凯恩斯),分别对应垂直的LM曲线和水平的LM曲线。

\hspace*{\fill}

(1)t下降,将会导致$ \alpha $上升,在纵轴截距不变的情况下,IS曲线变得更加平坦,财政政策乘数变大,货币政策乘数变大。政府购买支出对利率的效应也变大,货币政策对利率的效应变小。

(2)b上升,影响IS曲线在纵轴的截距$ A_0/b $下降,在横轴截距不变的情况下,IS曲线将变得更加平坦。财政政策乘数减小,财政政策对均衡利率的影响减小。货币政策乘数变大,货币政策对均衡利率的影响变小。

(3)k下降不影响IS曲线斜率,它决定了LM曲线在横轴的截距,$ M_0/(kP) $,在纵轴截距不变的情况下,LM曲线变得更加平坦了。财政政策乘数变大,财政政策对均衡利率的影响变小。货币政策乘数变大,货币政策对均衡利率的影响变大。

(4)h影响LM曲线在纵轴的截距$ -M_0/(hP) $,h上升,在横轴截距不变的条件下,LM曲线变得更加平坦。财政政策乘数变大,财政政策对均衡利率的影响变小。货币政策乘数变小,货币政策对均衡利率的影响变小。

\subsection{关于财政政策效力的命题}
导致IS曲线变得平坦的原因有两个:t下降和b上升。但t下降将导致财政政策乘数变大,b上升将导致财政政策乘数下降。(笼统地讲IS曲线越平坦财政政策越有效是错误的,必须区分导致IS曲线变平坦的原因。)

导致LM曲线变平坦的原因有两个:k下降和h上升。两个起因都导致财政政策乘数变大。(笼统地讲LM曲线越平坦财政政策越有效是对的。)

\hspace*{\fill}

关于财政政策效力的正确表述是:

命题1 

在LM曲线斜率不变的条件下,由税率(t)下降导致的IS曲线越平坦,财政政策的效力越大。

\hspace*{\fill}

命题2

在LM曲线斜率不变的条件下,由投资需求的利率弹性上升(b上升)导致的IS曲线越平坦,财政政策的效力就越小。

\hspace*{\fill}

命题3

在IS曲线斜率不变的条件下,LM曲线越平坦(无论是k下降还是h上升导致),财政政策的效力越大。

\subsection{关于货币政策效力的命题}
导致LM曲线变得平坦的因素有两个:k下降和h上升。两个起因分别导致两种截然相反的结果。k下降导致货币政策乘数变大,h上升导致货币政策乘数减小。(不能笼统地说LM曲线越平坦,货币政策的效力越大或越小,必须区分原因)。

导致IS曲线变平坦的因素有两个:t下降和b上升,两个起因都导致货币政策乘数变大,所以可以笼统地讲IS曲线越平坦货币政策的效力越大。

\hspace*{\fill}

关于货币政策效力正确的表述是:

\hspace*{\fill}

命题4 在LM曲线斜率不变的条件下,IS曲线越平坦,货币政策效力越大。

\hspace*{\fill}

命题5 在IS曲线斜率不变的条件下,由于货币需求的收入弹性(k)下降导致的LM曲线越平坦,货币政策效率越大。

\hspace*{\fill}

命题6 在IS曲线斜率不变的条件下,由于货币需求的利率弹性(h)上升导致的LM曲线越平坦,货币政策效率越小。

\end{document}