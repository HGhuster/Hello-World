\documentclass{article}
\title{宏观经济学指标的度量}
\author{Dawei Wang}
\date{\today}
\usepackage{ctex}
\usepackage{amsmath}
\usepackage{amssymb}
\begin{document}
	\maketitle
\section{国民收入的核算}
国民收入既可以用GNP和GDP表示,也可以用Y(yield,产出)来表示。

\subsection{核算对象——国民生产总值}
GNP(Gross National Products)代表国民生产总值。

GDP(Gross Domestic Products)代表国内生产总值。

\hspace*{\fill}

1. 二者的相同之处

都是指一个国家在一定时期内所生产的最终产品市场价值的总和。

(1)“一定时期内”(一年内)——计算期

第一,说明国民收入是流量不是存量。

第二,不包括已有的商品的交易。例:古董的交易,由于古董被生产出来的当年已经被计入当年的GNP和GDP,重复交易的产值不再计入。但是经纪人撮合买卖双方的劳务是发生在当期的,因此经纪人的佣金应计入GNP和GDP。

(2)“最终产品”

是指最后供人们使用的产品,与之对应的是中间产品,区分最终产品和中间产品主要是为了避免重复计算。

计算最终品的三个优点:

第一,可以避免重复计算。

第二,可以避免由于经济结构的不同带来的数据的不可比性。

第三,最终产品不仅包括有形的产品,还包括无形的劳务。

(3)“市场价值”

表明最终产品要经过市场交换。

优点:在流通环节,统计便利,操作、计量方便。

缺点:

有漏损,不经过交换的经济活动所创造的价值不被计入GDP。

地下经济、黑市交易不能统计进去。

(4)“总和”

强调一个国家的GDP等于每种产品i的产量$ Q_i $ 乘以价格$ P_i $的加总。

(5)“生产”


第一,是所生产的而不是所销售的,即不完全经过交换的。

只要是生产出来的,无论卖没卖出去,全部都计入GDP。

第二,必须把与生产无关的、既不提供物品也不提供劳务的活动所带来的价值排除在GNP的计算之外。(例如出售股票、债券的收益。)能否计入GNP,要把握住两点:一是看有没有提供物品,二是看有没有提供劳务。

\hspace*{\fill}

2. 两者的不同之处

(1)GDP:以领土作为统计指标,强调无论劳动力和其他生产要素是属于本国还是外国,只要是在本国领土上生产的产品和劳务的价值都计入GDP。

GDP=本国要素在国内的收入+外国要素在国内的收入

GDP是指一个国家在本国领土上,在一定时期内,所生产的最终产品市场价值的总和。

(2)GNP:以人口为统计标准,无论劳动力和生产要素在国内还是在国外,只要是本国常住居民所生产的产品和劳务的价值都计入GNP。

常住居民包括:常住本国的本国公民、暂住外国的本国公民、常住本国的外国公民。三类人中,前两类是绝大多数,第三类是极小部分,若略去不计。

GNP=本国要素在国内的收入+本国要素在国外的收入。

\hspace*{\fill}

GDP-GNP=外国要素在本国的收入-本国要素在外国的收入

现在国际上更通用的指标是GDP。GNP由于界定个人是否为本国常住居民,工作量很大、很繁琐。而统计所在地的指标则比较简便。

\subsection{核算方法之一:支出法}
支出法的内容是指一个国家的总支出(aggregate expenditure)就等于这个国家的GDP,即
\[
GDP=AE
\]
支出法的含义:通过核算一定时期内,整个社会购买最终产品的总支出来计算GDP。整个社会包括:消费者、厂商、政府、外国四个经济主体。

1. 个人消费支出C(consume)

消费者的个人消费支出包括:耐用品消费(使用期在1年以上)支出、非耐用品消费支出(使用期在1年以内)和劳务支出。消费者的个人消费支出既包括游行的物品也包括无形的劳务。

\hspace*{\fill}

2. 私人国内总投资(investment)

厂商的支出是私人国内总投资。消费和投资都是流量,都是时期分析。

总投资=重置投资(折旧)+净投资

净投资=固定资产投资+存货投资

存货投资=意愿存货+非意愿存货

重置投资是指用于维护原有资本存量完整的投资支出,也就是补偿资本存量中已损耗部分的投资。重置投资的多少取决于原有资本存量的数额,构成与寿命等情况,它不会导致原有资本存量的增加。所以国民生产总之GNP减去重置投资就是国民生产净值NNP。

GNP-重置投资=NNP

住宅建设是计入到投资这一项的,而且是记在固定资产投资这一项下。

\hspace*{\fill}

3. 政府购买支出G(government purchase)

政府支出=政府购买支出+转移支付(TR,transfer payment)+公债利息等。

政府购买支出=政府兴办公共工程的开支+政府机构的建立、维持和运营的费用。

4. 净出口(net export)
外国消费者对本国产品的需求(消费支出)就体现为本国的出口,用X来表示。计入GDP的是净出口NX:

净出口NX=出口X-进口M

\subsection{核算方法之二——收入法}
GDP蛋糕如何分配的角度。工人贡献了自己的劳动,得到了工资;资本家贡献了资本,得到了利息;土地所有者贡献了土地,得到了地租。

整个社会主体可分为两类:
第一类——公共部门(政府),收入体现为政府收入,即税收(用TA表示)。

第二类——私人部门(消费者、厂商),收入体现为个人可支配收入(用$ Y_d $代表)。

个人可支配收入的去向分为两部分:消费和储蓄(用S代表):

$ Y_d=C+S $

从收入角度计量,在统计私人部门和政府部门分到的蛋糕的时候,可能会出现这种情况,政府部门把自己分到的蛋糕又给私人部门切了一块(转移支付TR,来源于税收,转化为个人可支配收入)。

综上所述,我们用收入法得到一个收入核算恒等式:

\[
Y(GDP)=Y_d+TA-TR=C+S+TA-TR
\]

或者:
\[
Y(GDP)+TR=Y_d+TA
\]

\subsection{国民收入核算恒等式}
1. 国民收入核算恒等式
\[
C+I+G+NX\equiv Y(GDP)\equiv Y_d+TA-TR\equiv C+S+TA-TR
\]

\hspace*{\fill}

2. 计算GDP的不同方法

还可以用生产法来计算生产各环节的增值总和。

\hspace*{\fill}

3. GDP的各种等量指标

在均衡分析里,GDP等价于很多指标:总产量Y、GNP、总供给AS、总收入、国民收入。

\[
GNP=GDP=AS=Y
\]

\subsection{国民收入核算中的其他总量}

在均衡分析中认为国民生产总值、国民收入等都是相同的概念,而在核算中要求对这些总量区别对待。

国民生产总值(GNP):通过支出法或收入法得到

国民生产净值(NNP):GNP-资本消耗(折旧)

国民收入(NI):国民生产净值-企业间接税-其他

个人收入(PI):国民收入-公司利润-社会保险税+政府和企业给个人的转移支付+利息+红利

个人可支配收入(PDI):个人收入-个人所得税及非税支付

\subsection{GDP(或GNP)指标的缺陷}
1.存在低估

由于GDP强调的是“市场价值的总和”。无市场价格的物品就被排除在外。在市场不健全的情况下,有些该计入的未计入,自给自足那一块也漏损掉了。

\hspace*{\fill}

2. 反映的只是产品数量,无法反映产品质量的改进。

这是GDP指标一个很重要的缺陷,衡量的是产值。产值可能一样,但是提供服务的质量是有差别的。

如果用一个产业的产值占GDP的比重来衡量一个产业的重要性,就会发现,有一些传统产业由于交易方式比较落后,产品价值高,所以占GDP比重比较大。但这并不能说明这个产业更重要,有可能是由于历史原因,造成这个产业的产品价格居高不下。

\hspace*{\fill}

3. 只计算最终产品的市场价值,而没有考虑生产该产品的社会成本。

从另一个角度看,追求GNP的增长,可视为短期利益,对环境的保护可视为长期利益,长短期利益的权衡很重要。

\hspace*{\fill}

4. 没有考虑闲暇对人们福利的影响,也是一种低估。

闲暇本身是福利的体现,GDP是衡量综合国力的指标,想要成为体现人们福利的指标,就必须把闲暇因素考虑在内。

\section{价格指数}
\subsection{名义GDP和实际GDP}
(1)名义GDP:以当年价格计算的GDP。

\hspace*{\fill}

(2)实际GDP:选定一个基期,以基期的不变的价格来计算GDP。

\hspace*{\fill}

区分名义GDP和实际GDP的目的是:把计算中,价格水平上升带来的导致GDP计算结果变化的不确定因素剔除出去,从而剔除了价格水平的波动对GDP数值造成的影响。由此引出第三个概念——GDP缩减指数。

\hspace*{\fill}

(3)GDP缩减指数(GDP deflator,又称GDP折算指数):等于名义GDP除以实际GDP再乘以100。

\[
GDP\enspace deflator=\frac{\sum P^i_i\times Q^i_i}{\sum P^i_b\times Q^i_i}\times 100
\]

\subsection{消费者价格指数CPI}

衡量价格指数的指标,除了GDP缩减指数以外,还有消费者价格指数CPI(consumer price index)。CPI衡量一个国家消费者生活成本的变动情况,通过设定一个消费品系列或者“消费品篮子”,然后比较两个时期消费品价格的变化所带来的影响。
\[
CPI=\frac{\sum P^i_i\times Q^i_b}{\sum P^i_b\times Q^i_b}\times 100
\]

就范围来讲GDP缩减指数强调的是当期数量$ Q^i_i $,而CPI强调的是基期数量$ Q^i_b $。GDP缩减指数涉及的i是全部产品和劳务,CPI涉及的i是跟消费相关的特定的一组产品组合。

就国别来讲,GDP缩减指数涉及领土的概念,仅仅包括在本国领土内生产出来的产品。而CPI包括所有的消费品,也会有进口品。

\subsection{通货膨胀率}

衡量价格指数的第三个指标,通货膨胀率(用$ \pi $来代表)。
\[
\pi=\frac{P_i-P_{i-1}}{P_{i-1}}
\]
其中$ P_i $为第i年的价格指数,$ P_{i-1} $为第i-1年的价格指数。

通货膨胀率的计算至少有两种选择:第一种选择GDP缩减指数,第二种选择CPI。

\end{document}