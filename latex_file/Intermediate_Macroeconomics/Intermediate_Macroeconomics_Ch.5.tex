\documentclass{article}
\title{宏观经济政策}
\author{Dawei Wang}
\date{\today}
\usepackage{ctex}
\usepackage{amsmath}
\usepackage{amssymb}
\begin{document}
	\maketitle
\section{货币政策}
在利率和收入的坐标空间内,假定初始的IS曲线和LM曲线决定的均衡国民收入是Y*,Y*是能够实现产品和货币两个市场同时均衡的收入水平,常见的情形是Y*比能够实现劳动力市场充分就业的$ Y_f $小,存在一个紧缩缺口,要实现劳动力市场的充分就业,就要把现有的国民收入从Y*提升到$ Y_f $。

实现充分就业的渠道有两种:

1.在LM不变的情况下,使IS曲线发生一个向右的平移,这是运用财政政策;

2.在IS不变的情况下,使LM曲线发生一个向右的平移,这是运用货币政策;

3.同时运用财政政策和货币政策,IS和LM曲线同时发生向右的平移。

\subsection{影响货币供应量的货币政策工具}
央行通过运用货币政策工具来影响货币供应量。

1. 关于银行的几个概念:

(1)准备金

准备金是商业银行持有的为应付储户的提款所需要的货币。

完全的准备金等于储户的全部存款。

(2)法定准备金R

法定准备金是中央银行规定商业银行必须持有的最低数量的准备金,用R来表示。

法定准备金比率用rd表示,rd=法定准备金/储户的全部存款。

(3)超额准备金

超额准备金是指超出法定准备金的那部分。

准备金=法定准备金+超额准备金

(4)高能货币H

高能货币(high-power money),又称“基础货币”或者“强力货币”。

H=C+R


\subsection{货币政策的传导机制}

\subsection{货币政策效力分析之一}

\subsection{货币政策效力分析之二}

\subsection{货币政策的缺陷}


\section{财政政策}
\subsection{财政政策工具}

\subsection{财政政策传导机制}

\subsection{财政政策效力分析之一}

\subsection{财政政策效力分析之二}

\subsection{财政政策自身的特点——自动的稳定器}

\subsection{财政政策的缺陷}


\section{产出的构成和政策组合}
\subsection{产出的构成}

\subsection{财政政策的选择}

\subsection{财政-货币政策组合的效果}

\subsection{中国宏观经济政策小结}


\section{总需求曲线}

\subsection{总需求与总需求曲线}

\subsection{总需求曲线的传导}

\subsection{财政政策对总需求曲线位置的影响}

\subsection{货币政策对总需求曲线斜率的影响}

\subsection{小结}


\end{document}