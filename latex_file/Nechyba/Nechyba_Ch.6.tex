\documentclass{article}
\title{Nechyba Ch.6 尽力做到“最好”}
\author{Dawei Wang}
\date{\today}
\usepackage{ctex}
\usepackage{amsmath}
\usepackage{amssymb}
\begin{document}
	\maketitle
\section{选择:结合经济处境与偏好}
机会成本=边际替代率

预算线的斜率表示$ x_1 $用$ x_2 $表示的机会成本——市场允许消费者把$ x_1 $转化为$ x_2 $的比率;

无差异曲线的斜率代表边际替代率——消费者在多得到1单位$ x_1 $时愿意放弃的$ x_2 $数量。

最优消费束处消费者对相对价值的估价(MRS)与市场的价格比率(预算线斜率)达到一致。

\hspace*{\fill}

市场如何使我们在边际上都是一样的

在进入市场之后,优于市场提供给所有消费者的价格比率都是一样的,因此所有消费者的预算线斜率一致,在所有消费者都选择各自的最优消费束后,消费者在边际上变得一致(MRS=预算线斜率)——在离开市场时,他们对商品有着相同的偏好(最优化选择后每个人的MRS都等于市场价格比率)。

\hspace*{\fill}

市场如何消除我们进行贸易的任何需要

一旦消费者离开市场后,他们在边际上对商品的评价在边际上完全一样,意即大家都有了相同的MRS,因此他们之间不存在进一步交易并使得每个人都变得更好的可能性,这是因为在市场中每个人已经尽力做到最好了。

\hspace*{\fill}

帕累托最优:如果不存在改变一种情形的方式,使得一些人变好而同时没有任何人变得更差,经济学家称该情形是有效的,反之无效。

在人们的边际替代率相同时,他们之间不能通过相互交易使彼此都获益。

\subsection{角点解}
有时候我们的偏好与处境使得尽力做到最好意味着我们选择不消费某一种特定的产品。

\hspace*{\fill}

如果在一个消费者偏好的特定模型中所有的商品都是“必要的”,则角点解是不可能产生的;

一旦无差异曲线与坐标轴相交,有些商品就不是“必要的”,因此存在在某个经济条件下角点解为最优选择的可能。

\subsection{多解——选择集和偏好的非凸性}
最优解的唯一性由两个假设保证:

1. 预算线是直线;

2. 所有的偏好都被假定满足“平均优于极端”的假设。

\hspace*{\fill}

选择集的非凸性:

预算中的“扭结”,严格来讲,在无差异曲线图满足“平均优于极端”的假设时,并不是多重“最优”消费束可能性的必要条件。相反,必要的是一个称为选择集的“非凸性”的性质。、

\hspace*{\fill}

一个点集是凸的(convex)是指一条连接一个集合内两点的直线本身也包含在这个集合内部。相反,一个点集是非凸的(non-convex)是指连接一个集合内两点的直线有一部分位于该集合外部。

\hspace*{\fill}

偏好的非凸性——偏好的非凸性也能导致多(最优)解。

\hspace*{\fill}

给定一个人的经济处境,在他或者她的选择集与偏好都没有展示出非凸性时,我们可以确定个体经济人有唯一的“最优”选择。更精确地,我们要求偏好为严格凸——平均严格优于极端(在无差异曲线包含线性组成部分或者“平坦点”的时候,多重最优消费束是可能的)。

\section{在数学模型中最优化}
\subsection{最优化问题}
\[
\max\limits_{x_1,x_2} u(x_1,x_2)\qquad s.t.\qquad p_1x_1+p_2x_2=I
\]

两种解最优化的思路:

1.把受约束的最优化转化为不受约束的最优化;

2.解受约束最优化问题的拉格朗日方法。

3.机会成本=边际替代率(MRS);$ MRS=\frac{p_1}{p_2} $

\subsection{角点解}

\subsection{非凸性与一阶条件}

当最优化问题中所有的商品都必要时——无差异曲线不与坐标轴相交——我们已经证明了问题的最优解都满足拉格朗日问题的一阶条件。

在非凸性出现时,受约束最优化问题的一阶条件是一点为真正最优解的必要条件但非充分条件。

可以通过检查二阶条件确保拉格朗日法只产生真正的最优解。

\subsection{从观测的选择估计偏好}






\end{document}