\documentclass{article}
\title{Nechyba Ch.9 商品需求与劳动和资本供给}
\author{Dawei Wang}
\date{\today}
\usepackage{ctex}
\usepackage{amsmath}
\usepackage{amssymb}
%\usepackage{graphicx} %插入图片的宏包
%\usepackage{float} %设置图片浮动位置的宏包
%\usepackage{subfigure} %插入多图时用子图显示的宏包
\begin{document}
	\maketitle
作为消费者,个人成为了商品和服务的需求者;

作为工人,个人成为了劳动的供给者;

作为储蓄者,个人成为了资金的供给者;

作为借贷者,个人成为了资金的需求者。

\section{推导需求与供给曲线}

\subsection{A 对商品和服务的需求}

区分三种需求曲线:

收入需求曲线:反映(外生的)收入和商品需求量之间的关系;

自价格需求曲线:反映一种商品的价格与其自身需求量之间的关系;

交叉价格需求曲线:反映一种商品的价格和另一种商品的需求量之间的关系。

\hspace*{\fill}

收入需求关系:收入与需求之间的关系有时用恩格尔曲线来表示。

恩格尔定律:随着收入的增加,人们在食物上的总支出增加(食物是正常品),人们在食物上的支出占收入的比例也下降(食物是必需品)。

\hspace*{\fill}

自价格需求关系:传统的需求曲线。

\hspace*{\fill}

交叉价格需求关系:

\subsection{劳动供给}

劳动供给曲线绘制了个人在不同的工资率下选择向市场提供的劳动量。

休闲作为一种正常品,是导致供给曲线向下倾斜的必要条件,但不是充分条件。

当休闲是低档品,此时财富效应和替代效应在以休闲为横轴的方向上同向变化,从而工资增加,休闲减少,工作时间增加。此时并不能说休闲是吉芬品,因为吉芬品是在收入为外生的时候定义的(此时收入为内生)。

\hspace*{\fill}

在Cobb-Douglas情况下,价差价格需求曲线的形状由于收入效应和替代效应恰好抵消为垂线。

\end{document}