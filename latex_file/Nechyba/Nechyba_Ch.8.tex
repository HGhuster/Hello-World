\documentclass{article}
\title{Nechyba Ch.8 劳动和资本市场的财富与替代效应}
\author{Dawei Wang}
\date{\today}
\usepackage{ctex}
\usepackage{amsmath}
\usepackage{amssymb}
\usepackage{graphicx} %插入图片的宏包
\usepackage{float} %设置图片浮动位置的宏包
\usepackage{subfigure} %插入多图时用子图显示的宏包
\begin{document}
	\maketitle
考虑收入是内生的情况,即收入并不是固定的预算而是通过售卖我们的禀赋得到。

\section{财富效应、替代效应与禀赋}
当商品价格变化时,由于我们所拥有的某件东西的价值发生了变化,因此我们的预算线也改变了。这种效应叫作“财富效应”,因为当价格变化时消费者会经历一个财富变化,因而影响消费者拥有的财富价值。

\hspace*{\fill}

财富效应:可以把一种商品价格的上涨看作另一种商品价格的下降。(从预算线角度)。

因此当收入为内生时,一种商品价格上涨可能导致其消费变多(财富效应),此时不能认定其为吉芬品!!!

\hspace*{\fill}

研究工人劳动供给和工人工资之间关系的劳动经济学家已得出结论:即对于一个工人当工资上涨时,最初当其工资较低时劳动供给会增加,随着工资继续上涨,该普通工人的劳动供给最终会减少。

工资(税后)上涨的替代效应会使工人增加他们的工作时间,而随着工资上涨财富效应更可能使工人减少他们的工作时间(假定闲暇是正常品)。

因此政府通过减税来增加政府收入是没有保证的。反之亦然。

\hspace*{\fill}

实际利率取决于资本市场中供给和需求的力量。

更高的利率是否会引起储蓄的增长,也不确定。



\end{document}