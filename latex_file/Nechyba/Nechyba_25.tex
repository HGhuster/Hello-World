\documentclass{article}
\title{Nechyba Ch.25 寡头垄断}
\author{Dawei Wang}
\date{\today}
\usepackage{ctex}
\usepackage{amsmath}
\usepackage{amssymb}
\usepackage{graphicx} %插入图片的宏包
\usepackage{float} %设置图片浮动位置的宏包
\usepackage{subfigure} %插入多图时用子图显示的宏包
\begin{document}
	\maketitle
寡头垄断市场结构是指由于某些进入壁垒,使得市场上只有少数几个企业,外部不能对它们构成竞争威胁。与垄断的情形类似,进入壁垒可能是技术性的,或者法律性的。如果进入壁垒消失,只要有正的利润,新企业就会不断进入,直至完全变成完全竞争市场。

由于垄断市场中只有少数几个企业,我的企业产量多少会影响其他企业所收取的价格,或者我的企业的定价决策会决定其他企业所能收取的价格。垄断企业因此处于一种策略性的环境中,它们的决策直接影响它们所在的经济环境。

\section{寡头垄断的竞争与合谋}

假定两个企业是同质的,并且具有不变的边际成本。

在寡头垄断市场上的两家企业本质上需要制定两个基本决策:(1)生产多少;(2)销售价格。

正确的策略性变量究竟是价格还是产量取决于寡头垄断企业面临的情形,我们称之为企业运作的经济环境。因此我们建立两种形式的模型:产量竞争(quantity competition)模型和价格竞争(price competition)模型。

寡头垄断模型的一个特点是寡头垄断企业或同时、或序贯地确定它们地策略性决策。在同时决策的情况下我们使用纳什均衡的概念,在序贯决策的情况下使用子博弈完美纳什均衡的概念。

我们讨论四种不同的模型:(1)价格竞争,企业同时决定价格策略决策;(2)价格竞争,企业序贯决定价格策略竞争;(3)产量竞争,企业同时决定产量策略决策;(4)产量竞争,企业序贯决定产量策略决策。

如果寡头垄断企业简单地联合起来像一个单一垄断企业一样行动(消除垄断企业之间的竞争),寡头垄断企业在原则上会做得更好。

\subsection{寡头价格(伯兰特)竞争}

伯兰特认为,即使只有两家寡头垄断企业,如果它们进行价格竞争,会导致寡头垄断企业市场价格等同于完全竞争时的价格(价格等于边际成本)。

\hspace*{\fill}

同时定价策略决策
	
在伯兰特的的模型中两个企业同时做决策,并具有不变的边际成本(无固定成本)。当寡头垄断企业考虑公布价格时,不得不考虑对手将公布什么价格和消费者对不同价格的组合将如何反应。一个合理的推论为:如果宣布不同的价格,消费者都会奔向价格较低者,另一个企业将不能出售任何一单位商品。

因此,寡头垄断厂商将试图避免两种情形发生:第一,不会为了吸引顾客而定价过低以至于导致负的利润。由于假设没有固定成本和不变的边际成本,这意味着寡头垄断厂商不会定价低于边际成本。第二,寡头垄断厂商不会定价高于对手,因为这样意味着失去所有顾客。因而,宣布一个等于边际成本的价格是我们对对方策略的“最优反应”,这个结果就是纳什均衡。

 




\end{document}