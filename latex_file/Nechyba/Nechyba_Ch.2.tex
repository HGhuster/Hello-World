\documentclass{article}
\title{Nechyba Ch.2 消费者的经济环境}
\author{Dawei Wang}
\date{\today}
\usepackage{ctex}
\usepackage{amsmath}
\usepackage{amssymb}
\begin{document}
	\maketitle
\section{A 消费者选择集与预算约束}
给定约束时的所有选项为选择集(choice set)。

任何一个行动的机会成本是采取这个行动所放弃的下一个最好的行动;机会成本度量了做选择时的权衡取舍。

外生收入的变大会导致选择集变大,但是机会成本不变——机会成本由价格比率决定。

\hspace*{\fill}

内生的收入:禀赋


禀赋(endowment):禀赋是指一个消费者所拥有的一束商品,并可以用来与其他商品进行交换。禀赋的一个定义特征是:因为消费者拥有禀赋束,ta可以选择无视商品的市价消费掉那束商品。

\hspace*{\fill}

当预算约束来自禀赋时,对消费者而言可得的货币不是固定的。可得的货币取决于禀赋束中商品被赋予的价格。当禀赋束中的商品价格变化时预算线围绕禀赋点旋转。

\section{B 消费者选择集与预算方程}
\subsection{两商品选择集与预算集}
选择集:
\[
C(p_1,p_2,I)=\{(x_1,x_2)\in\mathbb{R}^2_+|p_1x_1+p_2x_2\le I\}
\]

预算集:
\[
B(p_1,p_2,I)=\{(x_1,x_2)\in\mathbb{R}^2_+|p_1x_1+p_2x_2=I\}
\]
\subsection{多于两个商品的选择集与预算集}
选择集:
\[
C(p_1,p_2,\cdots,p_n,I)=\{(x_1,x_2,\cdots,x_n)\in\mathbb{R}^n_+|p_1x_1+p_2x_2+\cdots+p_nx_n\le I\}
\]

预算集:
\[
B(p_1,p_2,\cdots,p_n,I)=\{(x_1,x_2,\cdots,x_n)\in\mathbb{R}^n_+|p_1x_1+p_2x_2+\cdots+p_nx_n=I\}
\]
\subsection{产生于禀赋的选择集}
\[
C(p_1,p_2,e_1,e_2)=\{(x_1,x_2)\in\mathbb{R}^2_+|p_1x_1+p_2x_2\le p_1e_1+p_2e_2\}
\]



\end{document}