\documentclass{article}
\title{Nechyba Ch.3 劳动与金融市场的经济环境}
\author{Dawei Wang}
\date{\today}
\usepackage{ctex}
\usepackage{amsmath}
\usepackage{amssymb}
\begin{document}
	\maketitle
我们在放弃一些禀赋来获得货币进行消费;我们的内生收入取决于我们所拥有的禀赋的价格。
\section{A 工人与储蓄者的预算}
\subsection{作为工人的选择集}
一个小时休闲的机会成本是在工人在该小时所获得的工资。

\hspace*{\fill}

工人的禀赋点(无论工资是多少都可以完全消费掉)是其休闲时间。

\hspace*{\fill}

当选择集是从禀赋而不是从固定的货币量导出时,若价格变化,预算线会围绕着禀赋点旋转。

\hspace*{\fill}

政府政策的劳动力市场政策(税收、补贴)会使工人预算线产生扭结。

\subsection{作为储蓄者时的预算约束}
作为储蓄者的决策是是否通过储蓄而不是马上消费或限制自己对未来收入的借贷程度来延迟当期的享乐。

现在的储蓄可以在未来出售后消费,贷款则实际上是卖出未来的资产以供今天消费。

\section{B 工人和储蓄者的选择集与预算方程}
\subsection{作为工人的选择集:}
\[
C(w,L)=\{(c,L)\in\mathbb{R}^2_+|c\le w(L-l)\}
\]

预算线
\[
B(w,L)=\{(c,L)\in\mathbb{R}^2_+|c= w(L-l)\}
\]

消费n种商品的选择集
\[
C(p_1,p_2,\cdots,p_n,w,L)=\{(p_1,p_2,\cdots,p_n,L)\in\mathbb{R}^{n+1}_+|p_1x_1+p_2x_2+\cdots p_nx_n\le w(L-l)\}
\]

\subsection{为未来规划时的选择集}
记$ e_1 $、$ e_2 $为当期与第二期的收入(禀赋):

选择集:
\[
C(e_1,e_2,r)=\{(c_1,c_2)\in\mathbb{R}^2_+|(1+r)c_1+c_2\le(1+r)e_1+e_2\}
\]

多期选择集:
\begin{equation*}
	\begin{split}
	&c_n+(1+r)c_{n-1}+(1+r)^2c_{n-2}+\cdots+(1+r)^{n-1}c_1\\
	=&c_n+(1+r)e_{n-1}+(1+r)^2e_{n-2}+\cdots+(1+r)^{n-1}e_1
	\end{split}
\end{equation*}

\end{document}