\documentclass{article}
\title{microeconomic}
\author{Dawei Wang}
\date{\today}
\usepackage{ctex}
\usepackage{txfonts}
\usepackage{amsmath}
\usepackage{amssymb}
\usepackage{graphicx} %插入图片的宏包
\usepackage{float} %设置图片浮动位置的宏包
\usepackage{subfigure} %插入多图时用子图显示的宏包
\begin{document}
\maketitle
\section{比较优势}
\subsection{从比较优势中得到的insight}
要实现合作中的集体产出达到生产可能性边界,合作的各方应做自己擅长的事情(分工合作+贸易)
\subsection{对于比较优势理论的一些思考}
在几十年前,中国生产的主要是低端产品为世界提供了大量廉价的低技术含量产品,大家互惠互利。随着时代发展,中国的产业在逐渐升级,一些高技术含量的产品我们也可以生产,但是在这个过程中与其他国家产生了较大矛盾(类似于近几年开始的中美贸易战),分工合作和贸易在造福双方的同时如何有效处理这种冲突?



\end{document}
