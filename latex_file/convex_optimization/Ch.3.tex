\documentclass{article}
\usepackage{geometry}
\usepackage[dvipsnames,svgnames]{xcolor}
\usepackage{tcolorbox}
\usepackage{mathrsfs}
\usepackage{amsfonts,amssymb}
\usepackage{euscript}
\usepackage{amsmath}
\usepackage{color}
\usepackage{soul}
\usepackage{CJKutf8}
\usepackage{indentfirst}

\usepackage{graphicx} %插入图片的宏包
\usepackage{float} %设置图片浮动位置的宏包
\usepackage{subfigure} %插入多图时用子图显示的宏包

\title{Convex functions}

\definecolor{highlight}{rgb}{1,0,0}

\newtcolorbox{defi}[2][]{colback=Salmon!20, colframe=Salmon!90!Black,title=\textbf{#2}}

\newtcolorbox{theo}[2][]{colback=JungleGreen!10!Cerulean!15,colframe=CornflowerBlue!60!Black,title=\textbf{#2}}
\begin{document}
\maketitle

\section*{1 Basic properties and examples}

\subsection*{1.1 Definition}

A function $f: \mathbf{R}^n \rightarrow \mathbf{R}$ is convex if $\mathbf{dom} f$ is a convex set and if for all x,
$y \in \mathbf{dom}f$, and $\theta$ with $0 \le\theta\le1$, we have

\[
f(\theta x+(1-\theta)y)\le\theta f(x)+(1-\theta)f(y)
\]

A function f is strictly convex if strict inequality above whenever $x \ne y$
and $0 < \theta < 1$. We say f is concave if -f is convex, and strictly concave if -f is
strictly convex.

All affine (and therefore
also linear) functions are both convex and concave. Conversely, any function that
is convex and concave is affine.

A function is convex if and only if it is convex when restricted to any line that
intersects its domain. In other words f is convex if and only if for all $x \in \mathbf{dom}f$ and
all v, the function $g(t) = f(x+tv)$ is convex (on its domain, $\{t|x+tv \in \mathbf{dom}f\}$).

A convex function is
continuous on the relative interior of its domain; it can have discontinuities only
on its relative boundary.


\subsection*{1.2 Extended-value extensions}

It is often convenient to extend a convex function to all of $\mathbf{R}^n$ by defining its value
to be ∞ outside its domain. If f is convex we define its extended-value extension
$\widetilde{f} : \mathbf{R}^n \rightarrow R \cup \{\infty\}$ by

\begin{equation*}
        \widetilde{f}=
        \left\{
            \begin{array}{lr}
            f(x)\qquad &x\in \mathbf{dom}f\\
            \infty\qquad &x\notin \mathbf{dom}f
            \end{array}
        \right.
\end{equation*}

The extension $\widetilde{f}$ is defined on all $\mathbf{R}^n$, and takes values in $\mathbf{R} \cup \{\infty\}$. We can recover
the domain of the original function f from the extension $\widetilde{f}$  as $\mathbf{dom}f =\{x|\widetilde{f}<\infty\}$.

\subsection*{1.3 First-order conditions}

Suppose $f$ is differentiable (i.e., its gradient $\nabla f$ exists at each point in $\mathbf{dom}f$,
which is open). Then $f$ is convex if and only if $\mathbf{dom}f$ is convex and

\[
f(y)\ge f(x) +\nabla f(x)^T(y-x)    
\]

holds for all $x,y\in\mathbf{dom}f$.

The affine function of y given by $f(x) +\nabla f(x)^T(y-x)$ is, of course, the first-order
Taylor approximation of f near x. The inequality states that for a convex
function, the first-order Taylor approximation is in fact a global underestimator of
the function. Conversely, if the first-order Taylor approximation of a function is
always a global underestimator of the functio

the inequality shows
that if $\nabla f(x) = 0$, then for all $y\in\mathbf{dom}f$, $f(y) \ge f(x)$, i.e., x is a global minimizer
of the function $f$.

Strict convexity can also be characterized by a first-order condition: $f$ is strictly
convex if and only if domf is convex and for $x,y\in\mathbf{dom}f$, $x \ne y$, we have

\[
f(y)> f(x) +\nabla f(x)^T(y-x)    
\]

For concave functions we have the corresponding characterization: $f$ is concave
if and only if domf is convex and

\[
f(y)\le f(x) +\nabla f(x)^T(y-x)    
\]

for all $x,y\in\mathbf{dom}f$.

\subsection*{1.4 Second-order conditions}

We now assume that f is twice differentiable, that is, its Hessian or second derivative
$\nabla^2f$ exists at each point in $\mathbf{dom}f$, which is open. Then f is convex if and
only if $\mathbf{dom}f$ is convex and its Hessian is positive semidefinite: for all $x \in \mathbf{dom}f$,

\[
    \nabla^2f\succeq 0    
\]

For a function on \textbf{R}, this reduces to the simple condition $f''(x) \ge 0$ (and $\mathbf{dom}f$
convex, i.e., an interval), which means that the derivative is nondecreasing. The
condition $\nabla^2f\succeq 0$ can be interpreted geometrically as the requirement that the
graph of the function have positive (upward) curvature at x.

Similarly, f is concave if and only if domf is convex and $\nabla^2f\preceq 0$ for
all $x \in \mathbf{dom}f$. Strict convexity can be partially characterized by second-order
conditions. If $\nabla^2f\succ 0$ for all $x \in \mathbf{dom}f$, then $f$ is strictly convex. The
converse, however, is not true: for example, the function $f:\mathbf{R} \rightarrow \mathbf{R}$ given by
$f(x) = x^4$ is strictly convex but has zero second derivative at $x = 0$.

\subsection*{1.5 Examples}

We start with some functions on \textbf{R}, with variable x.

\begin{enumerate}
    \item Exponential. $e^{ax}$ is convex on \textbf{R}, for any $a \in \mathbf{R}$.
    \item Powers. $x^a$ is convex on $\mathbf{R}_{++}$ when $a \ge 1$ or $a \le 0$, and concave for $0 \le a \le 1$.
    \item Powers of absolute value. $|x|^p$, for $p \ge 1$, is convex on textbf{R}.
    \item Logarithm. log x is concave on $\mathbf{R}_{++}$.
    \item Negative entropy. x log x (either on $\mathbf{R}_{++}$, or on $\mathbf{R}_+$, defined as 0 for $x = 0$) is convex.
\end{enumerate}

We now give a few interesting examples of functions on $\mathbf{R}^n$.

\begin{enumerate}
    \item Norms. Every norm on $\mathbf{R}^n$ is convex.
    \item Max function. $f(x) = max\{x_1,\cdots,x_n\}$ is convex on $\mathbf{R}^n$.
    \item Quadratic-over-linear function. The function $f(x, y) = x^2/y$, with
    \[
    \mathbf{dom} f=\mathbf{R\times R_{++}}=\{(x,y)\in \mathbf{R}^2|y>0\}    
    \]
    is convex
    \item Log-sum-exp. The function $f(x)=log(e^{x_1} +\cdots+ e^{x_n})$ is convex on $\mathbf{R}^n$.
    This function can be interpreted as a differentiable (in fact, analytic) approximation
    of the max function, since
    \[
    max\{x_1,\cdots,x_n\}\le f(x)\le max\{x_1,\cdots,x_n\}+log n 
    \]
    for all x.
    \item Geometric mean. The geometric mean $f(x)=(\Pi_{i=1}^nx_i)^{1/n}$ is concave on $\mathbf{dom}f=\mathbf{R}^n_{++}$
    \item Log-determinant. The function $f(X)=log(detX)$ is concave on $\mathbf{dom}f =\mathbf{S}^n_{++}$.
\end{enumerate}



\subsection*{1.6 Sublevel sets}

The $\alpha-sublevel$ set of a function $f : \mathbf{R}^n \rightarrow \mathbf{R}$ is defined as

\[
C_\alpha=\{x\in\mathbf{dom}f|f(x)\le \alpha\}    
\]

Sublevel sets of a convex function are convex, for any value of $\alpha$.

The converse is not true: a function can have all its sublevel sets convex, but
not be a convex function. For example, $f(x)=-e^x$ is not convex on R (indeed, it
is strictly concave) but all its sublevel sets are convex.

If f is concave, then its $\alpha-superlevel$ set, given by $\{x\in\mathbf{dom}f|f(x)\ge\alpha\}$, is a
convex set.

\subsection*{1.7 Epigraph}

The graph of a function $f : \mathbf{R}^n \rightarrow \mathbf{R}$ is defined as

\[
    \{(x, f(x))|x \in \mathbf{dom}f\}
\]

which is a subset of $\mathbf{R}^{n+1}$. The epigraph of a function $f : \mathbf{R}^n \rightarrow \mathbf{R}$ is defined as

\[
\mathbf{epi} f=\{(x, t) | x \in \mathbf{dom}f, f(x) \le t\},
\]

which is a subset of $\mathbf{R}^{n+1}$.

The link between convex sets and convex functions is via the epigraph: A
function is convex if and only if its epigraph is a convex set. A function is concave
if and only if its hypograph, defined as

\[
    \mathbf{hypo} f=\{(x, t)|t\le f(x)\}
\]

is a convex set.

\subsection*{1.8 Jensen's inequality and extensions}

\[
    f(\theta x + (1-\theta)y) \le \theta f(x)+(1-\theta)f(y),
\]

is sometimes called Jensen's inequality. It is easily extended to convex combinations
of more than two points: If $f$ is convex, $x_1,\cdots, x_k \in \mathbf{dom}f$, and $\theta_1,\cdots, \theta_k \ge 0$
with $\theta_1+\cdots+\theta_k=1$, then

\[
f(\theta_1x_1+\cdots+\theta_kx_k)\le \theta_1f(x_1)+\cdots+\theta_kf(x_k)
\]

As in the case of convex sets, the inequality extends to infinite sums, integrals, and
expected values. For example, if $p(x) \ge 0$ on $S \subseteq \mathbf{dom}f$,$\int_S p(x)dx=1$, then

\[
f(\int_Sp(x)xdx)\le\int_S f(x)p(x)dx    
\]

provided the integrals exist. In the most general case we can take any probability
measure with support in $\mathbf{dom}f$. If x is a random variable such that $x \in \mathbf{dom}f$
with probability one, and $f$ is convex, then we have

\[
    f(\mathbf{E}x)\le \mathbf{E}f(x)    
\]

\subsection*{1.9 Inequalities}

\[
\sqrt{ab}\le \frac{a+b}{2}    
\]

\[
    -log(\frac{a+b}{2})\le\frac{-log a-logb}{2}
\]

\[
    \sum_{i=1}^nx_iy_i\le(\sum_{i=1}^n|x_i|^p)^{1/p}(\sum_{i=1}^n|y_i|^q)^{1/q} \qquad p>1,\frac{1}{p}+\frac{1}{q}=1
\]

\section*{2 Operations that preserve convexity}

\subsection*{2.1 Nonnegative weighted sums}

Combining nonnegative scaling and addition, we see that the set of convex functions is itself a
convex cone: a nonnegative weighted sum of convex functions,

\[
f=w_1f_1+\cdots+w_mf_m    
\]

is convex. Similarly, a nonnegative weighted sum of concave functions is concave. A
nonnegative, nonzero weighted sum of strictly convex (concave) functions is strictly
convex (concave).

These properties extend to infinite sums and integrals. For example if f(x, y)
is convex in x for each $y \in \mathcal{A}$, and $w(y) \ge 0$ for each $y \in \mathcal{A}$, then the function g
defined as

\[
g(x)=\int_{\mathcal{A}} w(y)f(x,y)dy    
\]

is convex in x (provided the integral exists).

\subsection*{2.2 Composition with an affine mapping}

Suppose $f:\mathbf{R}^n \rightarrow \mathbf{R}, A \in \mathbf{R}^{n\times m}$, and $b \in \mathbf{R}^n$. Define $g: \mathbf{R}^m \rightarrow \mathbf{R}$ by

\[
    g(x) = f(Ax + b)
\],

with $\mathbf{dom} g = \{x | Ax + b \in \mathbf{dom}f\}$. Then if f is convex, so is g; if f is concave,
so is g.

\subsection*{2.3 Pointwise maximum and supremum}

If $f_1$ and $f_2$ are convex functions then their pointwise maximum $f$, defined by

\[
    f(x) = max\{f_1(x),f_2(x)\}
\]

with $\mathbf{dom} f=\mathbf{dom}f_1\cap \mathbf{dom}f_2$, is also convex.

It is easily shown that if $f_1,\cdots,f_m$ are convex, then their pointwise maximum

\[
    f(x) = max\{f_1(x),\cdots, f_m(x)\}
\]

is also convex.

The pointwise maximum property extends to the pointwise supremum over an
infinite set of convex functions. If for each $y \in \mathcal{A}$, f(x, y) is convex in x, then the
function g, defined as

\[
g(x)=\sup_{y\in \mathcal{A}} f(x,y)
\]

is convex in x.

Here the domain of g is

\begin{center}
    $\mathbf{dom} g=\{x|(x,y)\in\mathbf{dom}$ for all $y\in\mathcal{A}, \sup_{y\in\mathcal{A}}f(x,y)<\infty\}$
\end{center}

Similarly, the pointwise infimum of a set of concave functions is a concave function.

\[
\mathbf{epi}g=\cap_{y\in\mathcal{A}}\mathbf{epi}f(\cdot,y)    
\]

Thus, the result follows from the fact that the intersection of a family of convex
sets is convex.

\subsection*{2.4 Composition}



\subsection*{2.5 Minimization}


\subsection*{2.6 Perspective of a function}


\section*{3}

\section*{4}

\section*{5}

\section*{6}

\end{document}